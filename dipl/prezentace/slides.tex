\documentclass{beamer} % [handout] pro zakazani pauz
\usepackage{amsmath,amssymb}
\usetheme{boxes}
\usecolortheme{orchid}

\usepackage[czech]{babel}
\usepackage[T1]{fontenc}
\usepackage[utf8]{inputenc}

\uselanguage{czech}
\languagepath{czech}
\deftranslation[to=czech]{Definition}{Definice}
\deftranslation[to=czech]{Theorem}{Věta}

\usepackage{braket}
\usepackage{tikz-cd}
\usepackage{tikz}
\usetikzlibrary{matrix, calc, arrows, fit}

% Special symbols
\newcommand\rbelow{\prec}
\newcommand\crbelow{\rbelow\mkern-8mu\rbelow}
\newcommand\comp{\ensuremath{\gamma}}
\def\InclUp{\rotatebox[origin=c]{90}{$\subseteq$}}
\def\DotsUp{\rotatebox[origin=c]{90}{\dots}}
\newcommand\p[1]{\ensuremath{\mathcal{ #1 }}}

\newenvironment{diagram}{\begin{center}\begin{tikzcd}}{\end{tikzcd}\end{center}}

% Categories
\newcommand\categoryStyle[1]{\ensuremath{\mathbf{#1}}}
\newcommand\Frm{\categoryStyle{Frm}}
\newcommand\Loc{\categoryStyle{Loc}}
\newcommand\StoneFrm{\categoryStyle{StoneFrm}}
\newcommand\kBDStoneFrm{\ensuremath{\kappa\text{--}\categoryStyle{BDStoneFrm}}}
\newcommand\ExtrStoneFrm{\categoryStyle{ExtrDStoneFrm}}
\newcommand\Top{\categoryStyle{Top}}
\newcommand\StoneSp{\categoryStyle{StoneSp}}
\newcommand\Bool{\categoryStyle{Bool}}
\newcommand\ComplBool{\categoryStyle{ComplBool}}
\newcommand\kComplBool{\ensuremath{\kappa\text{--}\categoryStyle{ComplBool}}}
\newcommand\CRegFrm{\categoryStyle{CRegFrm}}
\newcommand\RegKFrm{\categoryStyle{RegKFrm}}

% Functors
\newcommand\R{\ensuremath{\mathfrak{R}}}
\newcommand\C{\ensuremath{\mathcal{R}}}
\newcommand\Bo{\ensuremath{\mathfrak{B}}} % booleanization
\newcommand\Bc{\ensuremath{\mathbb{B}}}   % boolean/complemented elements
\newcommand\BcO{\ensuremath{\Bc_{\omega}}}
\newcommand\BcK{\ensuremath{\Bc_{\kappa}}}
\newcommand\BcI{\ensuremath{\Bc_{\infty}}}
\newcommand\J{\ensuremath{\mathfrak{J}}}
\newcommand\JO{\ensuremath{\J_{\omega}}}
\newcommand\JK{\ensuremath{\J_{\kappa}}}
\newcommand\JI{\ensuremath{\J_{\infty}}}
\newcommand\Sp{\ensuremath{\mathcal{S}}}

\title{N\v ekter\'e bezbodov\' e aspekty souvislosti}
\author{Tom\'a\v s~Jakl}
\date{2. z\'a\v r\'i 2013}

\begin{document}
\frame{\titlepage}

\frame{
    \frametitle{Předběžnosti z bezbodové topologie}
    \begin{itemize}
        \item Úplný svaz $L$ je \emph{frame} pokud pro všechny $A\subseteq L$ a $b\in L$ platí:
        \begin{align*}
            (\bigvee A)\wedge b = \bigvee \Set{ a\wedge b | a\in A}
        \end{align*}

        Příklad: Nechť $(X,\tau)$ topologický prostor, pak $\Omega((X,\tau)) = \tau$ při uspořádaní inkluzí je frame.

        \item Nechť $L$ a $M$ jsou framy. Zobrazení $f\colon L\to M$ je \emph{frameový homomorfismus} pokud zachovává $\bigvee$, $\wedge$, $0$ a $1$.

        \medskip
        \pause

        \item Pseudokomplement: $a^* = \bigvee \Set{ x | x\wedge a = 0 }$
        \item $a\rbelow b \stackrel{def}{\equiv} a^*\vee b = 1$
        \item $\crbelow$ je největší interpolativní relace obsažená v $\rbelow$.
        \item Frame je \emph{úplně regulární} pokud $a = \bigvee \Set{ x | x\crbelow a}$ pro všechna $a$.
        \item Frame $L$ je \emph{kompaktní} pokud ke každému $M\subseteq L$, t.ž. $\bigvee M = 1$, existuje konečné $F\subseteq M$, že $\bigvee F = 1$.
    \end{itemize}
}

\frame{
    \frametitle{Kompaktifikace}

    \begin{theorem}
        Nechť $L$ je úplně regulární a nechť $\C L$ je frame regulářních ideálů $I$, tzn. ideálů pro které platí, že kdykoliv $a \in I$, tak existuje $b\in I$ t.ž. $a \crbelow b$.

        \medskip
        Potom frameový homomorfismus $\comp_L\colon \C L\to L$ definovaný předpisem $I\mapsto \bigvee I$ je kompaktifikace $L$.
    \end{theorem}

    \bigskip
    \pause

    \begin{theorem}
        Pro $L$ úplně regulární platí:
        \begin{itemize}
            \item $L$ je nesouvislý právě tehdy když $\C L$ je nesouvislý.
            \item Pokud je $L$ extremálně nesouvislý, pak $\C L$ je také extremálně nesouvislý.
        \end{itemize}
    \end{theorem}
}

\frame{
    \frametitle{Stoneova dualita}
    \begin{itemize}
        \item Frame je \emph{Stoneův} právě když je kompaktní a 0-dimenzionální.
    \end{itemize}

    \medskip

    \begin{theorem}[Stoneova dualita pro bezbodovou topologii]
        Funktory $\BcO$ a $\JO$ tvoří ekvivalenci kategorie Booleových algeber a Booleových homomorfismů a kategorie Stoneových framů a frameových homomorfismů.
    \end{theorem}

    \medskip
    \pause

    \begin{columns}[b]
    \begin{column}{6.5cm}
        \begin{theorem}[Klasická Stoneova dualita]
            Za předpokladu Boolean Ultrafilter Theoremu je kategorie Booleových algeber ekvivalentní s kategorií duální ke kategorii Booleových algeber.
        \end{theorem}
    \end{column}
    \begin{column}{3cm}
        \vspace{3em}
        \hspace{3em}
        \begin{tikzcd}[ampersand replacement=\&, scale = 0.7, transform canvas={scale=0.7}]
            \StoneFrm \ar[bend left=15]{rr}{\BcO} \& \& \Bool{} \ar[bend left=15]{ll}{\JO} \ar{ldd}{\Sp} \\
            \\
            \& \StoneSp \ar{uul}{\Omega}
        \end{tikzcd}
    \end{column}
    \end{columns}
}

\frame{
    \frametitle{Části Stoneovy duality}

    Buď $\kappa$ nekonečný kardinál.

    \begin{itemize}
        \item Element Stoneova framu je $\kappa$--\emph{generovaný} pokud je supremem množiny obojetných prvků kardinality menší než $\kappa$.

        \item Stoneův frame je $\kappa$--\emph{bazicky nesouvislý} pokud pro každý jeho $\kappa$--generovaný element $a$ platí, že 
        \begin{align*}
            a^{**}\vee a^* = 1.
        \end{align*}

        \item Frameový homomorfismus $f$ mezi Stoneovými framy je $\kappa$--\emph{bazicky úplných} pokud platí, že
        \begin{align*}
            f(a^*) = f(a)^*
        \end{align*}
        pro každý $\kappa$--generovaný element $a$.
    \end{itemize}

    \pause

    \begin{theorem}
        Kategorie $\kappa$--úplných Booleových algeber a $\kappa$--úplných Booleových homomorfismů je ekvivalentní kategorii $\kappa$--bazicky nesouvislých Stoneových framů a $\kappa$--bazicky úplných frameových homomorfismů.
    \end{theorem}
}

\frame{
    \frametitle{Extremálně nesouvislé Stoneovy framy}

    \begin{columns}[c]
    \begin{column}{6.5cm}
        Obraz podkategorie úplných Booleových algeber funktorem \JO{} je kategorie framů a frameových homomorfismů splňujících:
        \begin{align*}
            a^{**}\vee a^* = 1 \quad\text{a}\quad f(a^*) = f(a)^* \tag{N.O.}
        \end{align*}
        pro všechna $a$.

        \bigskip

        \visible<2>{
        \begin{theorem}
            Kategorie extremálně nesouvislých Stoneových framů a frameových homomorfismů splňujících (N.O.) je ekvivalentní kategorii úplných Booleových algeber a úplných Booleových homomorfismů.
        \end{theorem}
        }
    \end{column}
    \begin{column}{3cm}
    \vspace{-3em}
    \begin{diagram}[row sep=0.7cm, ampersand replacement=\&, scale = 0.7, transform canvas={scale=0.7}]
        \StoneFrm
            \ar[yshift=0.2em]{r}{\scalebox{1.5}\BcO}
        \& \Bool
            \ar[yshift=-0.2em]{l}{\scalebox{1.5}\JO} \\
        \\
        \omega_1\text{--}\categoryStyle{BDStoneFrm}
            \ar[yshift=0.2em]{r}{\scalebox{1.5}{$\Bc$}_{\omega_1}}
            \ar{uu}{\InclUp}
        \& \omega_1\text{--}\ComplBool
            \ar{uu}{\InclUp}
            \ar[yshift=-0.2em]{l}{\scalebox{1.5}{$\J$}_{\omega_1}} \\

        {} \ar{u}{\InclUp} \& {} \ar{u}{\InclUp} \\
        \DotsUp \& \DotsUp \\


        \kBDStoneFrm
            \ar[yshift=0.2em]{r}{\scalebox{1.5}\BcK}
            \ar{u}{\InclUp}
        \& \kComplBool
            \ar{u}{\InclUp}
            \ar[yshift=-0.2em]{l}{\scalebox{1.5}\JK} \\

        \DotsUp
            \ar{u}{\InclUp}
        \& \DotsUp
            \ar{u}{\InclUp} \\
        {} \& {} \\

        \ExtrStoneFrm
            \ar[yshift=0.2em]{r}{\scalebox{1.5}{\BcI}}
            \ar{u}{\InclUp}
        \& \ComplBool
            \ar{u}{\InclUp}
            \ar[yshift=-0.2em]{l}{\scalebox{1.5}\JI}
    \end{diagram}
    \end{column}
    \end{columns}
}

\frame{
    \frametitle{Vztah kompaktnosti a Stoneovy duality}

    \begin{itemize}
        \item V Booleově algebře $a \crbelow b$ právě když $a \leq b$.
    \end{itemize}

    \bigskip

\begin{center}
\begin{tikzpicture}[descr/.style={fill=white,inner sep=2.5pt}, ampersand replacement=\&]
    \tikzset{
      mymx/.style={
        matrix of nodes,
        row sep=3.5em,
        column sep=3.5em,
      },
      lbl/.style={
        below,
        auto,
      }
    }
    \tikzstyle{bigbox} = [draw=gray, thick, rounded corners, rectangle]

    \pgfmathsetmacro{\H}{1.2}
    \pgfmathsetmacro{\W}{0.6}

    \matrix (m) [mymx]
        { \RegKFrm \& \& \CRegFrm \\
          \StoneFrm \& \Bool \& \\
          \ExtrStoneFrm \& \ComplBool \& \categoryStyle{DMRFrm} \\
        };
    \path[->,font=\scriptsize,every node/.style=lbl]
        (m-2-1)     edge node {\InclUp} (m-1-1)
        (m-3-1)     edge node {\InclUp} (m-2-1)
        (m-3-3)     edge node {\InclUp} (m-1-3)
        (m-3-2)     edge node {\InclUp} (m-2-2)
        (m-3-2)     edge node {$\subseteq$} (m-3-3)

        % Stone duality
        (m-2-2) edge[transform canvas={yshift=1.5pt}]  node[swap] {\JO}  (m-2-1)
        (m-2-1) edge[transform canvas={yshift=-1.5pt}] node[swap] {\BcO} (m-2-2)

        (m-3-2) edge[transform canvas={yshift=1.5pt}]  node[swap] {\JI}  (m-3-1)
        (m-3-1) edge[transform canvas={yshift=-1.5pt}] node[swap] {\BcI} (m-3-2)

        % Compact. Reflection
        (m-1-1.355) edge node[swap] {$\subseteq$} (m-1-3.185)
        (m-1-3.175) edge node[swap] {\C} (m-1-1.5);


%   \path[<->, thick]
%       (m-3-1) edge node {} (m-3-2);


    \node[bigbox, dotted] [fit = (m-2-1) (m-3-2)] {};
    \draw[gray, rounded corners, loosely dashed]
           ($(m-1-1) + (-\H,   0)$)
        -- ($(m-1-1) + (-\H, +\W)$)
        -- ($(m-1-3) + (+\H, +\W)$)
        -- ($(m-3-3) + (+\H, -\W)$)
        -- ($(m-3-3) + (-\H, -\W)$)
        -- ($(m-1-3) + (-\H, -\W)$)
        -- ($(m-1-1) + (-\H, -\W)$)
        -- ($(m-1-1) + (-\H,   0)$);
\end{tikzpicture}
\end{center}
}

\frame{
    \frametitle{Booleanizace}

    Buď $L$ frame, pak \emph{Booleanizace} $L$ (značíme ji $\Bo L$) je Booleovský frame/úplná Booleova algebra sestávající z elementů splňujících
    $$a = a^{**}.$$

    Na frameových homomorfismech máme zobrazení:
    $$(\Bo f)(x) = f(x)^{**}.$$

    \pause

    \begin{block}{Tvrzení}
        $\Bo\colon \ExtrStoneFrm\to \ComplBool$ je funktor.
    \end{block}

    \pause

    \begin{theorem}
        Funktory $\Bo$ a $\JI$ tvoří ekvivalenci kategorií \ExtrStoneFrm{} a \ComplBool.
    \end{theorem}
}

\frame{
    \frametitle{Charakterizace extremální nesouvislosti}

    \begin{itemize}
        \item Nechť $f\colon L\to M$ je hustý a na frameový homomorfismus mezi úplně regulárními framy, pak říkáme, že $M$ je hustý \emph{v} $L$.
        \item Pokud navíc $\C L \cong \C M$, říkáme, že $M$ je \emph{superhustý v} $L$.
    \end{itemize}

    \bigskip
    \pause

    \begin{theorem}
        Úplně regulární frame je extremálně nesouvislý/De Morganovský právě tehdy když každý frame v něm hustý je v něm superhustý.
    \end{theorem}

    \begin{theorem}
        Kompaktní metrizovatelné prostory nemají žádné superhusté sublokály.
    \end{theorem}
}

\end{document}
