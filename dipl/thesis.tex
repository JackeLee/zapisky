\documentclass[12pt,a4paper,fleqn]{report}
\usepackage[utf8]{inputenc}
% \usepackage{graphicx}
\usepackage{amsthm}
\usepackage{amsfonts}
\usepackage{amssymb}
\usepackage{a4wide}
\usepackage{titlesec}

\titlelabel{\thetitle.\quad}

\def\afterDotSpace{.5em}

\newcounter{thmCounter}[section]
\renewcommand \thechapter{\Roman{chapter}}
\renewcommand \thesection{\arabic{section}}
\renewcommand{\thethmCounter}{\thesection.\arabic{thmCounter}}

\def\blockHelperA{\par\medskip\noindent\refstepcounter{thmCounter}}
\def\blockHelperB{\arabic{section}.\arabic{thmCounter}.}
\def\blockHelperCA{\hspace{\afterDotSpace}\ignorespaces}
\def\blockHelperCB{\it\hspace{\afterDotSpace}\ignorespaces}

\def\blockEnv#1{\blockHelperA\hbox{\bf {\blockHelperB} #1.}\blockHelperCA}
\def\blockEnvS#1{\blockHelperA\hbox{\bf #1.}\blockHelperCA}
\def\blockPropEnv#1{\blockHelperA\hbox{\bf {\blockHelperB} #1.}\blockHelperCB}
\def\blockPropEnvS#1{\blockHelperA\hbox{\bf #1.}\blockHelperCB}
\def\endBlockEnv{\par\medskip}

\def\num{\blockHelperA\hbox{\bf {\blockHelperB}}\hspace{\afterDotSpace}}

\newenvironment{block}{\blockEnv}{\endBlockEnv}
\newenvironment{block*}{\blockEnvS}{\endBlockEnv}
\newenvironment{blockProp}{\blockPropEnv}{\endBlockEnv}
\newenvironment{blockProp*}{\blockPropEnvS}{\endBlockEnv}

\newtheoremstyle{newthmstyle}% name of the style to be used
    {3pt}% measure of space to leave above the theorem. E.g.: 3pt
    {3pt}% measure of space to leave below the theorem. E.g.: 3pt
    {\itshape}% name of font to use in the body of the theorem
    {}% measure of space to indent
    {\bfseries}% name of head font
    {.}% punctuation between head and body
    {\afterDotSpace}% space after theorem head; " " = normal interword space
    {\thmnumber{#2.}\thmname{ #1}\thmnote{ \rm [#3]}}% Manually specify head

\newtheoremstyle{newthmstyleNormal}{3pt}{3pt}{}{}{\bfseries}{.}{\afterDotSpace}
    {\thmnumber{#2.}\thmname{ #1}\thmnote{ \rm [#3]}}

\theoremstyle{newthmstyle}
\newtheorem{name}[thmCounter]{Note}
\newtheorem{lemma}[thmCounter]{Lemma}
\newtheorem{theorem}[thmCounter]{Theorem}
\newtheorem{proposition}[thmCounter]{Proposition}

\theoremstyle{newthmstyleNormal}
\newtheorem{definition}[thmCounter]{Definition}

\newcommand\Frm{\ensuremath{\mathrm{Frm}}}
\newcommand\StoneFrm{\ensuremath{\mathrm{StoneFrm}}}
\newcommand\Top{\ensuremath{\mathrm{Top}}}
\newcommand\StoneSp{\ensuremath{\mathrm{StoneSp}}}
\newcommand\N{\ensuremath{\mathbb{N}}}
\newcommand\R{\ensuremath{\mathfrak{R}}}
\newcommand\J{\ensuremath{\mathfrak{J}}}
\newcommand\p[1]{\ensuremath{\mathcal{ #1 }}}
\newcommand\Bo{\p{B}} % booleanization
\newcommand\Bc{\ensuremath{\mathbb{B}}} % boolean/complemented elements
\newcommand\closure[1]{\overline{#1}}

% TODO spacing before and after thms and others
% TODO macro for \downarrow with better spacing

\title{Some point--free aspects of connectedness}
\author{Tom\'a\v s Jakl}

\begin{document}
\maketitle
\tableofcontents

\chapter{Introduction}

\chapter{Preliminaries}
\section{Category Theory}
\section{Topology and point--free Topology}
Basic of \Top{} and \Frm.
Spaciality.
Separation axioms (regularity, complete regularity, normality, ...?).
TODO place somewhere StoneSp and StoneFrm and \J.

\chapter{Connectedness and Compactification}
\section{Connectedness and variants of disconnectedness}
\section{Compactness and compactification}
\section{Properties of compactification with respect to disconnectedness}
\begin{lemma}
    The following are equivalent:

    \begin{enumerate}
        \item Closure of each open sublocale is open.
        \item $\closure{\mathfrak{o}(a)} = \mathfrak{o}(a^{**})$.
        \item $a^{**} \vee a^* = 1$.
    \end{enumerate}
\end{lemma}

\begin{proposition}
    Let $L$ be a extremally disconnected frame, then $\R L$ is also extremally disconnected.
\end{proposition}

\chapter{Stone duality}
\section{Stone correspondence for \StoneFrm}
\begin{lemma}
    Let $B$ be a Boolean algebra and $I \in \J B$. Then $I$ is complemented iff $I =\;\downarrow a$ for some $a \in B$.
\end{lemma}
\begin{proof}
    Let $I$ be a complemented ideal. Since $I \vee I^c = 1_{\J B}$ there exists $a \in I$ and $b \in I^c$ such that $a \vee c = 1$. From $I \wedge I^c = \{0\}$ we have $a \wedge b = 0$ and $I \wedge \downarrow b = \{0\}$.
    From uniques of complements we get $I^c = \downarrow b$, indeed $I \vee \downarrow b = 1_{\J B}$ and $I \wedge \downarrow b = \{0\}$. Using the same arguments we get $I = \downarrow a$.

    Converse implication is trivial since $\downarrow a \vee \downarrow a^c = 1_{\J B}$.
\end{proof}

\section{Stone duality for \StoneSp} % TODO maybe StoneSp
\section{Parts of duality}
\subsection{Complete Boolean algebras}
\num TODO definition of Booleanization

\begin{lemma}
    \begin{enumerate}
        \item For all $J \in \J L$: $J^* =\;\downarrow (\bigvee J)^*$.
        \item For all $a \in L$: $(\downarrow a)^* = \downarrow a^*$.
        \item Let $L$ be Boolean, then for all $J \in \J L$: $J^{**} =\;\downarrow \bigvee J$.
    \end{enumerate}
\end{lemma}

\begin{lemma}
    Let $B$ be a Boolean frame, then $B \cong \Bo\J B$.
\end{lemma}

\begin{lemma}
    Let $L$ be a compact DeMorgan/extremally disconnected frame, then $L \cong \R\Bo L$.
\end{lemma}

\subsection{$\kappa$--complete Boolean algebras}
\num definition of $\StoneFrm^\kappa$.

\begin{lemma}
    Let $L$ be a $\StoneFrm^\kappa$ for some $\kappa$, then $\Bc L$ is $\kappa$--complete Boolean algebra.
\end{lemma}

\begin{lemma}
    Let $L$ be a $\kappa$--complete Boolean algebra, then $\J B \in \StoneFrm^\kappa$.
\end{lemma}

\chapter{Compactification and Metrizability}
\section{Uniformity and metrizability}
\section{Non--metrizability of compactification}
\begin{lemma}
    For $L$ normal frame: $\sigma(x)\vee\sigma(x) = \sigma(x\vee y)$.
\end{lemma}

\chapter{Conclusion}

\end{document}
