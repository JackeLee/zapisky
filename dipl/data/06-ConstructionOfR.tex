\chapter{Construction of \R}

\section{Summarization}

Recall Theorem~\ref{p:universalCompactification}. The compactification of a complete regular frames is obtained as the frame of all regular ideals. The construction of a Stone frame in the Stone correspondence (Theorem~\ref{p:stonefrmBoolIso}) is given in similar fashion. The functor $\JO$ maps a Boolean algebra to the frame of all ideals of the Boolean algebra. We can equivalently say that any Stone frame is obtained from a Boolean algebra by taking the frame of all regular ideals since in any Boolean algebra the relations $\leq$, $\rbelow$ and $\crbelow$ coincide.

In classical topology, the compactification and the Stone representation of Boolean algebras are the same construction as well.
The compactification of a completely regular topological space is homeomophic to the space of all ultrafilters of that space. Similarly for the Stone duality, each Stone space is constructed as the space of all ultrafilters of some Boolean algebra.

The whole (point-free) situation is depicted in the diagram below.

\begin{center}
\begin{tikzpicture}[descr/.style={fill=white,inner sep=2.5pt}]
    \tikzset{
      mymx/.style={
        matrix of nodes,
        % nodes=block,
        row sep=3.5em,
        column sep=3.5em,
      },
      lbl/.style={
        % above,
        below,
        auto,
        % pos=0.15,
        % sloped, % make the text follow the path
        %execute at begin node={$}, % begin math mode, for the minus signs etc.
        %execute at end node={$}, % end math mode
      }
    }
    \tikzstyle{bigbox} = [draw=gray, thick, rounded corners, rectangle]

    \pgfmathsetmacro{\H}{1.2}
    \pgfmathsetmacro{\W}{0.6}

    \matrix (m) [mymx]
        { \RegKFrm & & \CRegFrm \\
          \StoneFrm & \Bool & \\
          \categoryStyle{KDMRFrm} & \categoryStyle{CBool} & \categoryStyle{DMRFrm} \\
        };
    \path[->,font=\scriptsize,every node/.style=lbl]
        (m-2-1)     edge node {\InclUp} (m-1-1)
        (m-3-1)     edge node {\InclUp} (m-2-1)
        (m-3-3)     edge node {\InclUp} (m-1-3)
        (m-3-2)     edge node {\InclUp} (m-2-2)
        (m-3-2)     edge node {$ \subseteq^{\text{obj}}$} (m-3-3)

        % Stone duality
        (m-2-2) edge[transform canvas={yshift=1.5pt}]  node[swap] {\JO}  (m-2-1)
        (m-2-1) edge[transform canvas={yshift=-1.5pt}] node[swap] {\BcO} (m-2-2)

        % Compact. Reflection
        (m-1-1.355) edge node[swap] {$\subseteq$} (m-1-3.185)
        (m-1-3.175) edge node[swap] {\C} (m-1-1.5);

    \path[<->, thick]
        (m-3-1) edge node {} (m-3-2);


    \node[bigbox, dotted] [fit = (m-2-1) (m-3-2)] {};
    \draw[gray, rounded corners, loosely dashed]
           ($(m-1-1) + (-\H,   0)$)
        -- ($(m-1-1) + (-\H, +\W)$)
        -- ($(m-1-3) + (+\H, +\W)$)
        -- ($(m-3-3) + (+\H, -\W)$)
        -- ($(m-3-3) + (-\H, -\W)$)
        -- ($(m-1-3) + (-\H, -\W)$)
        -- ($(m-1-1) + (-\H, -\W)$)
        -- ($(m-1-1) + (-\H,   0)$);
\end{tikzpicture}
\end{center}

where the symbol $\subseteq^{\text{obj}}$ denotes an inclusion of objects of one category to another. (New categories in the diagram: \categoryStyle{CBool} denotes the category of complete Boolean algebras and \emph{all} Boolean homomorphisms. The category \categoryStyle{KDMRFrm} is the category of all extremally disconnected Stone spaces, or in other words all compact De Morgan regular frames, and \emph{all} frame homomorphisms. \categoryStyle{RegKFrm} denotes the category of compact regular frames, \categoryStyle{CRegFrm} the category of completely regular frames.)

In the diagram, Stone duality is drawn in the area surrounded by the dotted rectangle and the compactification is drawn in the area surrounded by the dashed curve. As we know from the discussion above, the construction on objects is exactly the same for both marked parts of the diagram. However, the Stone correspondence is an equivalence of categories, whereas the compactification is just a coreflection.

Taking the frame of all regular ideals of a bounded pseudocomplemented lattice is a general construction of compact frames. A natural question arises: Is it possible to extend the category \ComplBool{} to a wider subcategory of \categoryStyle{DMRFrm} and again obtain the equivalence of categories carried by \R{}? From the following diagram we can see why it is not possible.

\begin{diagram}
    \categoryStyle{KDMRFrm} \ar[<->, thick]{r}{} & \categoryStyle{CBool} \ar{r}{\subseteq^{\text{obj}}} & \categoryStyle{DMRFrm} \ar[bend right=20,swap]{ll}{\restr{\C}{\categoryStyle{DMRFrm}}} \\
    \ExtrStoneFrm \ar[yshift=-0.2em,swap]{r}{\BcI}\ar{u}{\InclUp} & \ComplBool \ar[yshift=0.2em,swap]{l}{\JI} \ar{u}{\InclUp} \ar{ru}{\subseteq}
\end{diagram}

Objects of \categoryStyle{CBool} are in correspondence with objects of \categoryStyle{KDMRFrm}. On the other hand, \R{} provides a coreflection of the category \categoryStyle{DMRFrm} onto \categoryStyle{KDMRFrm}. Since \ComplBool{} is a subcategory of \categoryStyle{CBool}, we cannot hope to expand the correspondence by extending the category \ComplBool{} into a wider subcategory of \categoryStyle{DMRFrm}.

\section{Booleanization}
In this section we will prove that Booleanization is functorial in the part of the Stone correspondence for extremally disconnected Stone frames, and that it constitutes an equivalence of categories.

\begin{definition}\label{d:Booleanization}
    Let $H$ be a Heyting algebra. By \DEF{Booleanization} of $H$ we mean the set
    $$
    \DEFSYM{Booleanization}{\Bo H} = \Set{ a^{**} | a \in H }.
    $$
\end{definition}

\begin{proposition}
    If $H$ be a Heyting algebra then $\Bo H$, with joins and meets defined
    $$ a \sqcup b = (a^* \wedge b^*)^* \quad\text{and}\quad a\sqcap b = a \wedge b,$$

    \noindent is a Boolean algebra.
\end{proposition}
\begin{proof}
    First, we will show that the operations $\sqcap$ and $\sqcup$ are really the meet and the join on $\Bo H$. Let $a,b,c \in \Bo H$:

    \begin{itemize}
        \item From~\ref{p:pseudcomplProperties}, $a\sqcap b = a^{**}\wedge b^{**} = (a\wedge b)^{**} \in \Bo H$.
        \item Whenever $a,b \leq c$ then $a^*, b^*\geq c^*$ and so $a^*\wedge b^*\geq c^*$, hence $a\sqcup b = (a^*\wedge b^*)^* \leq c^{**} = c$. Trivially $a\sqcup b \in \Bo H$.
    \end{itemize}

    For $a \in \Bo H$: $a^* = a^{***} \in \Bo H$ too, $a\sqcup a^* = (a^*\wedge a^{**})^* = 0^* = 1$ and also $a\sqcap a^* = a\wedge a^* = 0$. Hence, each element is complemented and $\Bo H$ is a Boolean algebra.
\end{proof}

\num\label{p:propertiesBooleanization}Let $L$ be a frame. The mapping defined as $a \mapsto a^{**}$ is a nucleus:
    \begin{enumerate}[label=(N\arabic*)]
        \item $a \leq a^{**}$: it is a standard inequality (from \ref{p:pseudcomplProperties});
        \item $a \leq b$ implies $a^{**} \leq b^{**}$: since taking pseudocomplements is antitone;
        \item $a^{**\;**} = a^{**}$: follows from $a^* = a^{***}$; and
        \item $(a \wedge b)^{**} = a^{**} \wedge b^{**}$: it is again a standard equality (from \ref{p:pseudcomplProperties}).
    \end{enumerate}

    Therefore, by~\ref{p:nuclProp}, $\Bo L$ is a sublocale of $L$ and consequently a complete Boolean algebra. As a direct consequence of~\ref{p:nuclProp}, the mapping
    $$\text{\DEFSYM{BetaL}{$\beta_L$}}\colon L \to \Bo L,\quad a \mapsto a^{**},$$
    is a frame homomorphism.

\begin{block}{Note}
    $\Bo L$ is the smallest dense sublocale of $L$; joins are given by the formula $(a \vee b)^{**}$ (since $a \mapsto a^{**}$ is a nucleus).
\end{block}


\begin{definition}\label{d:BooleanizationMorph}
    For a frame homomorphism $f\colon L \to M$ set \DEFSYM{BooleanizationMorph}{$\Bo f$}$\colon \Bo L \to \Bo M$ to be the mapping
    $$\Bo f\colon a \mapsto f(a)^{**}.$$
\end{definition}

\num\label{p:boolenizationCond}The following Proposition is taken from~\cite{banaschewski1996booleanization}.
\begin{proposition*}
    Let $f\colon L \to M$ be a frame homomorphism. Then $\Bo f$ is a frame homomorphism such that the following diagram commutes
    \begin{diagram}
        L \ar{r}{\beta_L} \ar{d}{f} & \Bo L \ar{d}{\Bo f}\\
        M \ar{r}{\beta_M}           & \Bo M
    \end{diagram}
    if and only if $f(a^{**}) \leq f(a)^{**}$ for all $a \in L$.
\end{proposition*}
\begin{proof}
    First observe that
    \begin{align}
        f(a^{**})^{**} = f(a)^{**} \iff f(a^{**}) \leq f(a)^{**},\quad\text{for all } a\in L.\label{e:2202iff20leq02}\tag{W.O.}
    \end{align}

    $\Leftarrow$ is straightforward and $\Rightarrow$ follows from $f(a^{**}) \leq f(a^{**})^{**} $. Commutativity of the diagram is precisely the equality $f(a^{**})^{**} = f(a)^{**}$.

    The last thing we need to show is that $f(a^{**}) \leq f(a)^{**}$ implies that $\Bo f$ is a frame homomorphism. It is straightforward to see that $\Bo f$ preserves $0$, $1$ and $\sqcap$. For $\sqcup$--preserving take $a, b \in \Bo L$ and compute
    $$(\Bo f)(a)\sqcup (\Bo f)(b) = (f(a)^{**}\vee f(b)^{**})^{**} \geq f(a^{**}\vee b^{**})^{**} = f(a\vee b)^{**} = (\Bo f)(a\sqcup b),$$

    \noindent where the middle and the last equality hold by~(\oldref{e:2202iff20leq02}). The opposite inequality follows from $f(a)^{**}\vee f(b)^{**}\leq f(a\vee b)^{**}$. Thus $\Bo f$ is a Boolean homomorphism.

    We will show that $\Bo f$ preserves arbitrary joins. Again using~(\oldref{e:2202iff20leq02}), we get
    $$ (\Bo f)(\bigsqcup A) = f((\bigvee A)^{**})^{**} = f(\bigvee A)^{**} = (\bigvee f[A])^{**}. $$

    Now take any $a\in A$, from $f(a) \leq \bigvee f[A]$ we have $f(a)^{**} \leq (\bigvee f[A])^{**}$. Since $a$ was chosen arbitrary, we also have $\bigvee_{a\in A} f(a)^{**} \leq (\bigvee f[A])^{**}$ and therefore $(\bigvee_{a\in A} f(a)^{**})^{**} \leq (\bigvee f[A])^{**}$. The opposite inequality is trivial, hence $\bigsqcup (\Bo f)[A] = (\bigvee f[A])^{**}$.

    Thus we obtain
    $$ \bigsqcup (\Bo f)[A] = (\bigvee f[A])^{**} = (\Bo f)(\bigsqcup A),$$
    \noindent which is what we wanted.
\end{proof}

\begin{proposition}\label{p:BooleanizationFunctor}
    Let $\p C$ be a subcategory of the category of frames and frame homomorphisms satisfying (\oldref{e:2202iff20leq02}). Then

    $$\Bo\colon \p C \to \ComplBool,$$

    \noindent defined on objects as in~\ref{d:Booleanization} and on morphisms as in~\ref{d:BooleanizationMorph}, is a functor.
    % TODO do not reference inside a proof -- ref{e:2202iff20leq02}
\end{proposition}
\begin{proof}
    From~\ref{p:propertiesBooleanization} we know that $\Bo L$ is a complete Boolean algebra. As a direct implication of Proposition~\ref{p:boolenizationCond} we get that for any morphism $f\colon L \to M$ in $\p C$ the following diagram commutes.

    \begin{diagram}
        L \ar{r}{\beta_L} \ar{d}{f} & \Bo L \ar{d}{\Bo f}\\
        M \ar{r}{\beta_M}           & \Bo M
    \end{diagram}

    \noindent $\Bo f$ is a frame homomorphisms, but it is also a complete Boolean homomorphisms, as we know from~\ref{p:kappaCompleteBAfromMeets}. Since the following diagram also commutes for any morphisms $f, g$ we see that $\Bo$ respects morphisms composition.

    \begin{diagram}
        L \ar{r}{\beta_L}
          \ar{d}{f}
          \ar[bend right]{dd}[swap]{gf} &
        \Bo L \ar{d}[swap]{\Bo f}
              \ar[bend left]{dd}{\Bo (gf)}\\

        M \ar{r}{\beta_M} \ar{d}{g} & \Bo M \ar{d}[swap]{\Bo g}\\
        N \ar{r}{\beta_N}           & \Bo N
    \end{diagram}

    Finally, for any identity frame homomorphism $i_L$, $\Bo(i_L)$ is the identity on $\Bo L$. Consequently, $\Bo$ is a functor.
\end{proof}

\begin{lemma}\label{p:basicalMorphs}
    Let $f\colon L \to M$ be a frame homomorphism and $a \in L$ such that $a^{**}\vee a^* = 1$. Then
    $$ f(a^{**})^* = f(a^*).$$

    In particular, for $a = a^{**}$ we have
    $$ f(a)^* = f(a^*).$$
\end{lemma}
\begin{proof}
    From the assumptions we see that the following holds
    \begin{align*}
        f(a^{**}) \vee f(a^*) &= f(a^{**} \vee a^*) = 1, \\
        f(a^{**}) \wedge f(a^*) &= f(a^{**} \wedge a^*)  =  0.
    \end{align*}

    Hence, $f(a^{**})$ is complemented and $f(a^*)$ is its complement. From distributivity of $M$ we know that complements are unique; thus $f(a^*) = f(a^{**})^*$.
\end{proof}

\num\label{p:BooleanizationFunctorOnExtrDSF}For any frame homomorphism between two extremally disconnected Stone frames $f\colon L\to M$. By Lemma~\ref{p:basicalMorphs}, $f(a^{**})^* = f(a^*)$ for any $a\in L$. Note that we have
    \begin{align*}
        f(a^{**})^* = f(a^*)\text{ and }f(a^{**}) \leq f(a)^{**} \iff f(a^*) = f(a)^*,\quad\text{for all } a\in L.\label{e:1001}\tag{N.O.}
    \end{align*}

    \noindent (Indeed, the $f(a^*) \leq f(a)^*$ is always true and $f(a^*) = f(a^{**})^* \geq f(a)^{***} = f(a)^*$, the opposite direction is straightforward).

    Hence, for a category of extremally disconnected frames, in order to satisfy the conditions of Proposition~\ref{p:BooleanizationFunctor}, we need to restrict morphisms only to those frame homomorphisms satisfying~(\oldref{e:1001}).

    Observe that the category \ExtrStoneFrm{} satisfies the conditions of Proposition~\ref{p:BooleanizationFunctor}. For a frame homomorphism, to be basically complete is precisely the same condition as to satisfy~(\oldref{e:1001}).

\begin{conclusion*}
    $\Bo\colon \ExtrStoneFrm \to \ComplBool$ is a functor.
\end{conclusion*}

In literature the frame homomorphisms satisfying (\oldref{e:2202iff20leq02}) are called \emph{weakly open} and frame homomorphisms satisfying (\oldref{e:1001}) are called \DEF{nearly open}[ homomorphism]~\cite{banaschewski1994variants}.
% TODO topological interpretation

\begin{observation}\label{p:boEQbc}
    $\Bo L = \BcI L$ for any extremally disconnected Stone frame $L$.
\end{observation}
\begin{proof}
    $\Bo L \supseteq \BcI L$ holds always. Let $x \in \Bo L$. Then $x = x^{**}$ and from extremal disconnectedness also $x^{**}\vee x^*=1$, hence $x \in \BcI L$.
\end{proof}


\num Until the end of this section, we will use the following Lemma frequently and without further reference.
\begin{lemma*} Let $L$ be a frame. Then:
    \begin{enumerate}
        \item For all $J \in \J L$: $J^* =\;\downset (\bigvee J)^*$.
        \item For all $a \in L$: $(\downset a)^* = \downset a^*$ in $\J L$.
        \item If $L$ is Boolean, then for all $J \in \J L$: $J^{**} =\;\downset \bigvee J$.
    \end{enumerate}
\end{lemma*}
\begin{proof}
    We will prove just the first statement, the others follow directly from it. Let $J$ be any ideal on $L$. Observe that $a \wedge \bigvee J = 0 $ iff $\downset a \wedge J = 0_{\Bo L}$. Since
        \begin{align*}
            J^* & = \bigvee \set{ L | L \wedge J = 0_{\Bo L} } = \bigcup \Set{ \downset a | \downset a \wedge J = 0_{\Bo L} } \text{ and}\\
            (\bigvee J)^* & = \bigvee \set{ a | a \wedge \bigvee J = 0 },
        \end{align*}

    \noindent we see that $\downset a \subseteq J^*$ iff $a \in J^*$ iff $a \leq (\bigvee J)^*$; hence $J^* = \downset (\bigvee J)^*$.
\end{proof}


\begin{proposition}
    The functor $\Bo\circ\JI$ is naturally equivalent to the identity functor on \ComplBool{}.
\end{proposition}
\begin{proof}
    Let $B$ be a Boolean frame. We see that $J \in \Bo\JI(B)$ iff $J = J^{**} = \downset \bigvee J$.

    Denote by $\tilde i_B\colon B \to \Bo\JI(B)$ the mapping $a \mapsto \downset a$. From the previous Observation we know that the definitions of $\tilde i_B$ and $i_B$ coincide, therefore by~\ref{p:BoolEquivalence}, $\tilde i_B$ is a Boolean isomorphism. From~\ref{p:completePartUnits}, we see that $\tilde i_B$ is a morphisms of \ComplBool{}.

    Now, for any complete Boolean algebras $A$ and $B$ and for any complete Boolean homomorphism $f\colon A \to B$, the following diagram commutes
    \begin{diagram}
        A \ar{r}{\tilde i_A} \ar{d}[swap]{f} & \Bo\JI(A) \ar{d}{\Bo\J(f)}\\
        B \ar{r}{\tilde i_B}                 & \Bo\JI(B)
    \end{diagram}

    \noindent Indeed, we have
    $$
    \Bo\JI(f)(\downset a) = (\downset f[\downset a])^{**} = \downset f(a)^{**} = \downset f(a).
    $$

    Hence, the collection $\tilde i = (\tilde i_B)_B$ of complete Boolean homomorphisms forms a natural equivalence between $\Bo\JI$ and the identity functor on \ComplBool.
\end{proof}

\begin{proposition}\label{p:JOBoIdentity}
    The functor $\JI\circ\Bo$ is naturally equivalent to the identity functor on \ExtrStoneFrm{}.
\end{proposition}
\begin{proof}
    Let $L$ be an extremally disconnected Stone frame. Similarly to the general case, define $\tilde v_L\colon \JI\Bo(L) \to L$ as
    $$ \tilde v_L\colon I \mapsto \bigvee I.$$

    \noindent From Observation~\ref{p:boEQbc}, we know the definitions of $\tilde v_L$ and $v_L$ are the same, and $\tilde v_L$ is a frame isomorphism by~\ref{p:StoneFrmEquivalence}. From~\ref{p:completePartUnits}, $\tilde v_L$ is a morphism of \ExtrStoneFrm.

    Further, let $L$ and $M$ be extremally disconnected Stone frames and let $f\colon L \to M$ be a basically complete frame homomorphism. The following diagram commutes
    \begin{diagram}
        \JI\Bo(L) \ar{d}[swap]{\JI\Bo(f)} \ar{r}{\tilde v_L} & L \ar{d}{f} \\
        \JI\Bo(M) \ar{r}{\tilde v_M}    & M
    \end{diagram}

    To show that, let $I \in \JI\Bo(L)$. First observe, for $x\in I$, $x = x^{**}$ and since $L$ is extremally disconnected, $x\vee x^* = 1$. Therefore, $f(x^*) = f(x)^*$ by Lemma~\ref{p:basicalMorphs}. Now, we can compute
    $$
    (\JI\Bo)(f)(I) = \downset \Set{ f(x)^{**} | x \in I } = \downset \Set{ f(x) | x \in I} = \downset f[I].
    $$

    \noindent And from that, we see
    $$
    \tilde v_M((\JI\Bo)(f)(I)) = \bigvee \downset f[I] = \bigvee f[I] = f(\bigvee I) = f(\tilde v_L(I)).
    $$

    The conclusion follows from commutativity of the diagram above.
    The collection $\tilde v = (\tilde v_L)_L$ of basically complete frame homomorphisms is a natural equivalence between $\JI\Bo$ and the identity functor on \ExtrStoneFrm.
\end{proof}


\num From \ref{p:BooleanizationFunctorOnExtrDSF}---\ref{p:JOBoIdentity}, we obtain the main result of this section.

\begin{theorem*}\label{p:completePart2}
    The functors \Bo{} and \JI{} provide an equivalence between the categories \\\ExtrStoneFrm{} and \ComplBool{}.
\end{theorem*}
