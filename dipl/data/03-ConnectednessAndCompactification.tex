\chapter{Connectedness and Compactification}

Compactness, connectedness and variants of disconnectedness are commonly studied properties of topological spaces. In this chapter, we will prove some results which are well known from Set theoretical topology and we will prove them in the context of point--free topology.

\section{Connectedness and variants of disconnectedness}
\begin{definition}
    We say a frame $L$ is \DEF{disconnected} if $L$ contains non-trivial elements $a$ and $b$ (that is, different from the top and the bottom of $L$) such that $a\vee b = 1$ and $a\wedge b = 0$. If a frame is not disconnected, we say it is \DEF{connected}.
\end{definition}

\begin{observation}\label{p:disconnectednessEquivalently}
    For a frame $L$, the following conditions are equivalent:

    \begin{enumerate}
        \item $L$ is disconnected.
        \item There exists a non-trivial both open and closed sublocale of $L$ (that is, different from $\emptyFrm$ and $L$).
        \item There exists an one-one frame homomorphism $f\colon $\DEFSYM{B2}{$B_2$}$ \to L$, where $B_2$ is the Boolean algebra on four elements.
    \end{enumerate}
\end{observation}

\begin{lemma}
    Let $L$ be a frame. The following are equivalent:

    \begin{enumerate}
        \item Closure of each open sublocale of $L$ is open.
        \item For all $a \in L$: $\closure{\mathfrak{o}(a)} = \mathfrak{o}(a^{**})$.
        \item For all $a \in L$: $a^{**} \vee a^* = 1$.
    \end{enumerate}
\end{lemma}
\begin{proof}
    The implication from 2.\ to 1.\ is trivial. For the implication from 3.\ to 2., first observe that
    \begin{align*}
        \open(a^{**})\vee \open(a^*) &= \open(a^{**}\vee a^*) = \open(1) = L, \text{ and} \\
        \open(a^{**})\wedge \open(a^*) &= \open(a^{**}\wedge a^*) = \open(0) = \emptyFrm.
    \end{align*}

    \noindent However, $\open(a^*)$ is complemented with $\clos(a^*)$. In other words
    \begin{align*}
        \clos(a^*)\vee \open(a^*) &= L, \text{ and} \\
        \clos(a^*)\wedge \open(a^*) &= \emptyFrm.
    \end{align*}

    \noindent By distributivity of the frame of all all sublocales of $L$, $\open(a^{**}) = \clos(a^*) = \closure{\open(a)}$.

    Finally, for the implication from 1.\ to 3.: $\closure{\open(a)} = \clos(a^*) = \open(b)$ for some $b \in L$, hence
    \begin{align*}
        \open(a^*\vee b) &= \open(a^*)\vee   \open(b) = L, \text{ and} \\
        \open(a^*\wedge b) &= \open(a^*)\wedge \open(b) = \emptyFrm.
    \end{align*}

    \noindent Thus $a^*$ is complemented with $b$ and $b = a^{**}$ from the uniqueness of complements.
\end{proof}

\begin{definition}
    A frame satisfying conditions 1., 2.\ and 3.\ from the previous Lemma is called \DEF{extremally disconnected} or \DEF{De Morgan}.
\end{definition}

\begin{definition}
    A frame $L$ is called \DEF{zero--dimensional} if it has basis consisting only of complemented elements.
\end{definition}

\begin{observation}
    \begin{enumerate}
        \item Any zero--dimensional frame is completely regular.
        \item Any regular and extremally disconnected frame is zero--dimensional.
    \end{enumerate}
\end{observation}
\begin{proof}
    \begin{enumerate}
        \item For every complemented $a$, we have $a \crbelow a$. Therefore, by zero--dimensionality
            $$e = \bigvee \Set{ c | c \leq e, c \text{ is complemented}} \leq \bigvee \Set{ x | x \crbelow e} \leq e$$

            \noindent for each $e$.

        \item Note, $x \rbelow y$ implies $x^{**} \rbelow y$. Since $x \leq x^{**}$ and every element of the form $x = x^{**}$ is complemented in extremally disconnected frame, we obtain a frame is zero--dimensional.
    \end{enumerate}
\end{proof}

\num Similarly for topological spaces, we say a topological space is \emph{connected}, \emph{disconnected}, \emph{zero--dimensional} or \emph{extremally disconnected} if the frame of its open sets is connected, disconnected, zero--dimensional or extremally disconnected.

\section{Compactness and compactification}

\begin{definition}
    A frame $L$ is \DEF{compact} if for every subset $C \subseteq L$ such that $\bigvee C = 1$ there exists a finite $F \subseteq C$ such that $\bigvee F = 1$.
\end{definition}

\begin{proposition}\label{p:compRegIsCR}
    If $L$ is a compact regular frame, then $L$ is completely regular.
\end{proposition}
\begin{proof}
    It suffices to check that $L$ is normal. Then by~\ref{p:normalRegular}, $L$ is also completely regular. Let $a, b \in L$ and satisfy $a\vee b = 1$, then from regularity, we know that
    $$ a = \bigvee \Set{ x | x \rbelow a}\quad\text{and}\quad b = \bigvee \Set{ y | y \rbelow b}.$$

    Set $A = \Set{ x | x \rbelow a}$ and $B = \Set{ y | y \rbelow b}$. Since $\bigvee A \vee \bigvee B = 1$, by compactness, there exists a finite $F \subseteq A\cup B$, such that $\bigvee F = 1$. From~\ref{p:rbellowProperties}, $f_a = \bigvee (F\cap A) \rbelow a$ and $f_b = \bigvee (F\cap B) \rbelow b$. Further,

    $$
    (f_a\wedge f_b^*)\vee b = (f_a\vee f_b)\wedge (f_b^*\vee b) = (\bigvee F)\wedge 1 = 1.
    $$

    \noindent By the same argument, $(f_b\wedge f_a^*)\vee a = 1$. And $(f_b\wedge f_a^*)\wedge (f_a\wedge f_b^*) = 0$, therefore $L$ is normal.
\end{proof}

\num\label{p:frameOfIdeals} {\bf The frame of ideals.} Let $L$ be a join--semilattice with 0. Denote by $\J L$ the set of all ideals of $L$. We will show that $\J L$ ordered by inclusion is a frame. Set intersection of two ideals is again an ideal, hence finite meets exist.

    Let $I_i \in \J L$, for $i \in J$, and set
    \begin{align*}
        \bigvee_{i\in J} I_i = \set{ \bigvee F | F\text{ is a finite subset of } \bigcup_{i\in J} I_i}.\label{e:idealJoin}\tag{Idl-$\bigvee$}
    \end{align*}

    \noindent The set defined this way is again an ideal and it is the supremum of $\Set{ I_i | i\in J}$. Also for any ideals $J$ and $I_i$ of $L$, for $i \in J$, the following equality holds
    $$(\bigvee_i I_i) \cap J = \bigvee_i (I_i \cap J).$$

    \noindent The $\supseteq$ inclusion is trivial, for the other inclusion take $x\in J$ such that $x = \bigvee F$ for some finite $F \subseteq \bigcup_i I_i$. Then $F \subseteq J$ (as $J$ is an ideal) and also $F \subseteq \bigcup_i (I_i \cap J)$. We get, $x \in \bigvee_i (I_i \cap J)$. Thus, $\J{L}$ is a frame.

    Moreover, $\J L$ is a compact frame: Let $I_i \in \J L$, for $i \in J$, such that $\bigvee_i I_i = L = 1_{\J L}$, then there exists a finite $F \subseteq \bigcup_i I_i$ such that $\bigvee F = 1$. Set $i(f) \in J$ such that $f \in I_{i(f)}$ for all $f \in F$. Then also $\bigvee_{f \in F} I_{i(f)} = 1_{\J L}$. Therefore $\J L$ is a compact frame.

\begin{blockProp*}{Conclusion}
    The set $\J L$ with joins as defined in (\oldref{e:idealJoin}) and meets defined as set intersections is a compact frame.
\end{blockProp*}

\begin{definition}
    We say a frame homomorphism $f\colon L \to M$ is \DEF{dense}[ frame homomorphism] if $f(a) = 0$ implies $a = 0$.

    We say a compact frame $K$ together with a frame homomorphism $c\colon K \to L$ is the \DEF[Cech--Stone compactification@]{(Čech--Stone) compactification} of a frame $L$ if $c$ is dense and for every dense frame homomorphism $d\colon K' \to L$, with $K'$ compact, there exists an unique frame homomorphism $\tilde d\colon K \to K$ such that the following diagram commutes
    \begin{diagram}
        K\ar{r}{c}& L \\
        K'\ar[dashed]{u}{\tilde d} \ar{ur}{d} &
    \end{diagram}
\end{definition}

\begin{observation}
    If the Čech--Stone compactification exists, it is determined uniquely, up to isomorphism.
\end{observation}

The following construction is due to Banaschewski and Mulvey~\cite{banaschewski1984stone}.

\num {\bf Regular ideals and a frame of regular ideals.} Let $L$ be a completely regular frame. We say an ideal $I$ is \DEF{regular}[ ideal] if for any $a \in I$ there exists a $b \in I$ such that $a\crbelow b$. Denote by \DEFSYM{RegIdeals}{$\R L$} the set of all regular ideals of $L$.

    For two regular ideals $I_1, I_2\in \R L$, their set intersection is again a regular ideal. Indeed, take any $a \in I_1\cap I_2$, then $a\crbelow b_i$ for some $b_i \in I_i$ and $a\crbelow b_1 \wedge b_2\in I_1\cap I_2$ from~\ref{p:crbellowProperties}.

    For any regular ideals $J$ and $I_i$ of $L$, for $i \in J$, from previous we know that $\bigvee_{i\in J} I_i$ is an ideal. We will show that it is a regular ideal. For $a \in \bigvee_{i\in J} I_i$, by~(\oldref{e:idealJoin}), there exists a finite $F \subseteq \bigcup_{i\in J} I_i$ such that $a = \bigvee F$. For every $f \in F$ there exists an $e_f \in I_j$, for some $j\in J$, such that $f \crbelow e_f$. But $\bigvee_{f\in F} e_f \in \bigvee_{i\in J} I_i$ and from~\ref{p:crbellowProperties}$, \bigvee F \crbelow \bigvee_{f\in F} e_f$.

    Hence, the set $\R L$ is a subframe of $\J L$. Consequently, $\R L$ is compact.

\num For an $a \in L$, set
    $$\sigma_L(a) = \Set{ x | x \crbelow a}.$$

\noindent This set is a regular ideal. We will omit the subscript if the frame $L$ can be determined from the context. % TODO better wording?

    The following property of $\sigma$ will be handy: $\sigma(a) \rbelow \sigma(b)$ for any $a\crbelow b$.
\begin{center}
\parbox{0.85\linewidth}{
    (\emph{Proof.} First, interpolate $a \crbelow x\crbelow y\crbelow b$. By~\ref{p:crbellowProperties}, $x^*\crbelow a^*$ and $x^*\in \sigma(a^*)\subseteq \sigma(a)^*$. Since $y\in \sigma(b)$ and $x^*\vee y = 1$, we have $\sigma(a)^*\vee \sigma(b) = 1_{\R L}$.\qed)
}
\end{center}

\begin{proposition}
    If $L$ is a completely regular frame, then $\R L$ is also completely regular.
\end{proposition}
\begin{proof}
    % $$
    % I = \bigcup \Set{ \sigma(a) | a\in I} = \bigvee \Set{ \sigma(a) | a\in I} = \bigvee \Set{ \sigma(a) | a\crbelow b\in I}.
    % $$

    to be done
\end{proof}


\num\label{p:compactificationFunctor} {\bf The functor \C{}.} Now, we are ready to define a functor from the category of completely regular frames to the category of compact regular frames. Denote by
    $$\DEFSYM{compactificationFunctor}{\C}\colon \CRegFrm \to \RegKFrm$$

    \noindent the following two mappings. On objects:
    $$\C(A) = \R A,$$

    \noindent for every $A \in \CRegFrm$, and on morphisms:
    $$ (\C f)(I) = \downset f[I],$$
    \noindent for every morphism $f\colon L \to M \in \CRegFrm$ and $I \in \R L$.

    The set $f[I]$ is obviously closed under finite meets, hence $\downset f[I]$ is an ideal. From the fact that, $a\crbelow b$ implies $f(a)\crbelow f(b)$, we see that $\downset f[I]$ is a regular ideal. Now, we observe that

    \begin{itemize}
        \item $\C f$ preserves joins:
        \begin{align*}
            (\C f)(\bigvee_i I_i) &= \downset f[\set{ \bigvee F | F\text{ a finite subset of } \bigcup_i I_i}] \\
                &= \downset \set{ \bigvee F | F\text{ a finite subset of } \bigcup_i f[I_i]} \\
                &= \set{ \bigvee F | F\text{ a finite subset of } \bigcup_i \downset f[I_i]} \\
                &= \bigvee_i (\C f)(I_i); \text{ and that}
        \end{align*}

        \item $\C f$ preserves finite meets:
        $$ (\C f)(I_1\cap I_2) = \downset f[I_1\cap I_2] \subseteq \downset f[I_1]\cap \downset f[I_2] = (\C f)(I_1)\cap (\C f)(I_2). $$

        \noindent For $x \in \downset f[I_1]\cap \downset f[I_2]$, there exists $y_i\in I_i$, for $i = 1,2$, such that $f(y_1) = f(y_2) = x$. Since $f$ is a frame homomorphism, $f$ preserves finite meets, hence $f(y_1\wedge y_2) = f(y_1)\wedge f(y_2) = x$. We know that $y_1\wedge y_2 \in I_1\cap I_2$, therefore $x\in \downset f[I_1\cap I_2]$.

    \end{itemize}

    We obtain that $\C f$ is a frame homomorphism. As a result, we have the following

    \begin{proposition*}
        $\C$ is a functor.
    \end{proposition*}

\num For a completely regular frame $L$, define \DEFSYM{compactification}{$\comp_L$}$\colon \C L\to L$ as follows:
    $$\comp_L\colon I \mapsto \bigvee I.$$

    And we see that

    \begin{align}
        \comp_L\sigma(a) = a\quad\text{and}\quad \sigma\comp_L(I) \subseteq I.\label{e:compactificationInequality}
    \end{align}

    \noindent Since both $\comp_L$ and $\sigma$ are monotone maps, we know that they form a Galois ajdunction, with $\comp_L$ to the left and $\sigma$ to the right. Hence, $\comp_L$ preserves joins, it also preserves finite meets: $\bigvee I_1 \wedge \bigvee I_2 = \bigvee \Set{a_1 \wedge a_2 | a_i \in I_i} \leq \bigvee \Set{ a | a \in I_1 \cap I_2 } = \bigvee (I_1 \cap I_2) \leq \bigvee I_1 \wedge \bigvee I_2$.

    Consequently, $\comp_L$ is a frame homomorphism and $\sigma$ is a localic map. Observe that $\comp_L$ is also dense, $\bigvee I = 0$ implies $I = \{0\}$.


\num We see that $\comp_*\colon \C{} \natto \Id$ is a natural transformation because the following diagram commutes

\begin{diagram}
    \C L\ar{r}{\comp_L}\ar{d}{\C f} & L\ar{d}{f} \\
    \C M\ar{r}{\comp_M}             & M
\end{diagram}

\noindent for any frame homomorphism $f\colon L\to M$ between completely regular frames (since $\comp_M(\C f)(I) = \bigvee (\downset f[I]) = \bigvee f[I] = f[\bigvee I] = f \comp_M(I)$).

\begin{lemma}
    If $L$ is a compact regular frame, then $\comp_L$ is an isomorphism and $L \cong \R L$.
\end{lemma}
\begin{proof}
    Let $I$ be a regular ideal of $L$. We already know that $\sigma\comp_L(I) \supseteq I$. For $x \in \sigma\comp_L(I)$, we have $x \crbelow \bigvee I$, hence $x^*\vee \bigvee I= 1$. By compactness of $L$, there exists a finite $F \subseteq I$ such that $x^*\vee \bigvee F = 1$. But $\bigvee F\in I$ and $x\crbelow \bigvee F$, hence $x$ belongs to $I$ and $\sigma\comp_L(I) \subseteq I$.

    Regular ideals of $L$ are precisely the ideals of the form $\sigma(a)$. Thus, $\comp_L$ is an one-one frame homomorphism and consequently $\comp_L$ is an isomorphism.
\end{proof}

\num From previous we obtain

\begin{theorem*}\label{p:universalCompactification}
    The functor \C{} and the natural transformation $\comp_*$ provides a coreflection of the category of completely regular frames onto the category of compact regular frames.

    Or in other words, for a completely regular frame $L$, the mapping $\comp_L\colon \C L \to L$ is the compactification of $L$.
\end{theorem*}

\begin{definition}
    A topological space is \emph{compact} if the frame of its open sets is compact.
\end{definition}


\section{Properties of compactification with respect to disconnectedness}

\begin{proposition}
    A completely regular frame is disconnected iff its Čech--Stone compactification is disconnected.
\end{proposition}
\begin{proof}
    For a completely regular frame $L$, if $L$ is disconnected, then by~\ref{p:disconnectednessEquivalently} there exists an one-one frame homomorphism $f\colon B_2\to L$. From~\ref{p:universalCompactification} we know that there exists an extension, a frame homomorphism $\tilde f\colon B_2\to \C L$, such that $\comp_L \tilde f = f$. Since $f$ is one-one, $\tilde f$ is also one-one.

    For converse, suppose $\C L$ is disconnected. There exists a non-trivial clopen sublocale $S$ of $\C L$; and $S\cap L$ is a non-trivial clopen sublocale of $L$.
\end{proof}

\begin{lemma}\label{p:closureIntersectedByDense}
    Let $S$ be a closed sublocale of $L$ and let $U\subseteq L$ be an open sublocale. Then $\closure{U\cap S}^L = \closure{U}^L$.
\end{lemma}
\begin{proof}
    Let $U = \open(a)$, for some $a$, then $\bigwedge (\open(a)\cap S) = \bigwedge \Set{ a\to s | s \in S } = a\to \bigwedge S$. And $\bigwedge S = 0$, since $S$ is dense in $L$ (and includes $0_L$). Hence $\closure{U\cap S}^L = \upset (\bigwedge (U\cap S)) = \upset(a\to 0) = \clos(a^*) = \closure{U}^L$.
\end{proof}

\begin{lemma}\label{p:closureOfClopen}
    Let $L$ be a completely regular frame and let $M$ be a clopen sublocale of $L$, then the closure of $M$ in $\C L$, the $\closure{M}^{\C L}$, is also clopen.
\end{lemma}
\begin{proof}
to be done
\end{proof}

\begin{proposition}\label{p:extrDiscPreserv}
    Let $L$ be an extremally disconnected frame, then $\C L$ is also extremally disconnected.
\end{proposition}
\begin{proof}
    To simplify the notation we will omit the superscript for the closure of a sublocale in $\C L$. Let $U$ be an open sublocale of $\C L$. Take $V = U\cap L$ open sublocale of $L$. We know that $\closure V^L = \closure{U\cap L}^L = \closure{U\cap L}\cap L = \closure U\cap L$ (the last equality follows from Lemma~\ref{p:closureIntersectedByDense}). From extremal disconnected, $\closure V^L$ is clopen in $L$ and, from Lemma~\ref{p:closureOfClopen}, $\closure{\closure V^L}$ is clopen in $\C L$. Then

    $$ \closure U = \closure{U\cap L} \subseteq \closure{\closure{V}^L} = \closure{\closure U\cap L} \subseteq \closure{\closure U} = \closure U.$$
\end{proof}

\begin{block}{Remark}
    Analogously to set topology, it is not always the case that Čech--Stone compactification of zero--dimensional frames is again zero--dimensional. Frames in which this is true are called strongly zero--dimensional~\cite{kou2002strongly}.
\end{block}

% TODO?
% Walker: The Stone-Cech Compactification
% p 223, Lemma 9.8. An open subset U of \beta(X) is connected if and only if X\cap U is connected.
% p 10. Theorem (Cech) 1.14. \beta(X) is that compactification of a space X in which completely separated subsets of X have disjoint closures.

