\chapter{Preliminaries}
We assume basic knowledge of Set Theory, mainly basic knowledge about cardinals and cardinalities, ``axiom of choice''.

We refer to the monograph \emph{Frames and Locales} written by Pultr and Picado~\cite{picado2011frames} as a basic introduction to point--free topology.

\section{Partially ordered set}

\num Let $A$ be a set and let $\leq$ be a binary relation on $A$ ($(\leq) \subseteq A\times A$). We say $(A, \leq)$ is a \DEF{partially ordered set} if the relation $\leq$ is \emph{partial order}, i.e. satisfies:
\begin{enumerate}[(P1)]
    \item \emph{reflexivity:} $a \leq a$ for all $a \in A$,
    \item \emph{antisymmetry:} $a \leq b$ and $b\leq a$ implies $a = b$ for all $a,b \in A$,
    \item \emph{transitivity:} $a \leq b$ and $b\leq c$ implies $a \leq c$ for all $a,b,c \in A$.
\end{enumerate}

Denote by $(A,\leq)^{\op}$ the partially ordered set $(A,\leq')$ such that $(\leq') = \Set{ (b,a) \in A\times A | a\leq b}$.

% TODO better wording
Let $S$ be a subset of $A$.
We say $u$ is an upper bound for $S$ if $u \geq x$ for all $x$ of $S$. We say $s$ is a supremum for $S$ if $s$ is the least upper bound for $S$. We write $s = \bigvee S$.
We say $l$ is an lover bound for $S$ if $l \leq x$ for all $x$ of $S$. We say $i$ is a infimum for $S$ if $i$ is the supremum for $S$ in $A^{\op}$, i.e. if $i$ is the greatest lower bound for $S$. We write $i = \bigwedge S$.

A partially ordered set $(A,\leq)$ is \emph{join semilattice} if each two element subset has a supremum. Similarly, \emph{meet semilattice} has an infimum for each two element subset.
A \DEF{lattice} is partially ordered set which is both join-- and meet-- semilattice. We write $a\vee b$ (\emph{meet}) for the supreum of $\{a,b\}$ and $a\wedge b$ (\emph{join}) for the infimum.

A mapping $f\colon A\to B$ between two partially ordered sets is called \emph{monotone} if $f(x)\leq f(y)$ wherever $x\leq y$. If, moreover, $A$ and $B$ are lattices and $f(x\vee y) = f(x)\vee f(y)$ and $f(x\wedge y) = f(x)\wedge f(y)$ for all $x,y\in A$, we say $f$ is a \emph{lattice homomorphism}.

\num A lattice is called a \DEF{complete lattice} if every subset has an infimum and a supremum.

\begin{theorem*}
    For a lattice $L$ are the following conditions equivalent:
    \begin{itemize}
        \item $L$ is a complete lattice;
        \item every subset of $L$ has a supremum;
        \item every subset of $L$ has an infimum.
    \end{itemize}
\end{theorem*}

For a complete lattice we therefore have two infinitary operations the meet $\bigvee$ and the join $\bigwedge$.
Note that every complete lattice contains two distinguished elements: 0 and 1, where $0 = \bigvee \emptyset$ and $1 = \bigwedge \emptyset$.

A lattice homomorphism $f\colon A\to B$ between two complete lattices is \DEF{complete lattice homomorphism} if $f(\bigvee X) = \bigvee f[X]$ and $f(\bigwedge X) = \bigwedge f[X]$ for each subset $X \subseteq A$.

\num A lattice $D$ is \DEF{distributive} if the following equation holds for all $a,b,c\in D$:
\begin{align}
    a\vee (b\wedge c) = (a\vee b)\wedge (a\vee c).\tag{Dist}
\end{align}

\begin{theorem*}
    For a lattice $D$, the following conditions are equivalent:
    \begin{itemize}
        \item $D$ is a distributive lattice,
        \item $a\vee (b\wedge c) = (a\vee b)\wedge (a\vee c)$ for all $a,b,c \in D$,
        \item For each $a,b,c\in D$ there exists at most one $x$ such that
            \begin{align*}
                a\vee x &= b\\
                a\wedge x &= c.
            \end{align*}
    \end{itemize}
\end{theorem*}

\num Let $A$ and $B$ be two partially ordered sets. Two monotone mappings $f\colon A \to B$ and $g\colon B\to A$ are in \DEF{Galois connection} or are \DEF{Galois adjoing} if
$$ f(x) \leq y\quad\text{iff}\quad x\leq g(y)\quad\text{for all } x\in A, y\in B.$$

\noindent Or equivalently
$$ fg(y) \leq y\quad\text{and}\quad x\leq gf(y)\quad\text{for all } x\in A, y\in B.$$

Then we say $f$ is a left Galois adjoing and a $g$ is a right Galois adjoing.
For a monotone map, its left or right Galois adjoint do not need to exists but if they do, they are uniquely determined.

Consequently, for two monotone maps in Galois connection $f,g$ we have that
$$ f g f = f\quad\text{and}\quad g f g = g. $$

\begin{theorem*}
    For a monotone map $f\colon A\to B$, if $f$ is a left (resp. right) Galois adjoint
    then $f$ preserves all existing suprema (resp. infima).

    Moreover, the converse implication also holds if $A$ and $B$ are complete lattices.
\end{theorem*}

\num Let $L$ be a lattice with 0. A pseudocomplement $a^*$ for an element $a\in L$ is the greatest element $x$ such that
$$ x\wedge a = 0. $$

\noindent Equivalently, $a^*$ is the pseudocomplement of $a$ if
$$ x\wedge a = 0 \quad\text{iff}\quad x\leq a^*\quad\text{for all } x\in L. $$

A lattice is called \DEF{pseudocomplemented}[ lattice] if every element has a pseudocomplement.

\begin{proposition*}\label{p:pseudcomplProperties} Properties of pseudocomplement.
\begin{enumerate}
    \item $a \leq a^{**}$, $a^* = a^{***}$,
    \item $a \leq b \implies a^* \geq b^*$,
    \item $(a \wedge b)^{**} = a^{**} \wedge b^{**}$.
\end{enumerate}
\end{proposition*}

\num Heyting algebras, $\_^* = \_ \to 0$
\num Boolean algebras and Boolean homomorphisms, \DEF{\Bool}
\num filters, ideals, principal filter/ideal, prime/maximal/ultrafilter on BA
\num Axiom of Choice/Zorn's Lemma, Boolean Ultrafilter Theorem

\section{Category Theory}
\begin{itemize}
    \item category (in basic definition locally small), demonstrative  examples using categories from previous section (most importantly \Bool), dual category??
    \item functor (important example: identity functor and constant functor), covariant, contravariant
    \item natural transformation, natural equivalence, functor category
    \item limits/colimits via "universal functors", example products, sums
    \item ajdunction only via units and counits of adjunction

    explain importance of adjunction by mentioning the limit preserving
\end{itemize}

\section{Topology and point--free Topology}

\num Basic definitions: topology, frame, their categories, functor $\Omega$, \Top{} and \Frm. frame homomorphisms, localic maps, continuous map, homeomorphism

\num Separation axioms (regularity, complete regularity, normality, ...?), rather below (and useful facts about them)

\num\label{p:rbellowProperties} $a_1\wedge a_2 \rbelow b_1 \wedge b_2$ and $a_1\vee a_2 \rbelow b_1 \vee b_2$ for $a_i\rbelow b_i$
\num\label{p:crbellowProperties} $a_1\wedge a_2 \crbelow b_1 \wedge b_2$ and $a_1\vee a_2 \crbelow b_1 \vee b_2$ for $a_i\crbelow b_i$
                                   $a \crbelow x\crbelow y\crbelow b$ implies $b^* \crbelow x\crbelow y\crbelow a^*$

    Again, as for topological spaces, we have:

    \num\label{p:normalRegular} Any normal and regular frame is completely regular: For $x\rbelow y$, the $x^*\vee y = 1$ holds, from normality there exists a $q$, such that $x^*\vee q = 1$ and $q^*\vee y = 1$, thus $x\rbelow q\rbelow y$. So $\Set{x | x\rbelow b} = \Set{x | x\crbelow b}$

    $T_4 + T_1 \implies T_{3.5} + T_0 \implies T_3 + T_0 \implies T_2 \implies T_1 \implies T_0$

\num completeness and cocompleteness of \Top{} and \Frm{}
\num Spatiality, some spatiality theorems (II.5.3, Hofmann--Lawson's Duality)
\num Subspace and sublocale/subframe, dense sublocale, empty frame (\emptyFrm), co-frame of sublocales (we will need its distributivity), $\iota_S\colon L \to S$ as subframe inclusion and as onto frame homomorphisms? (the next part needs it)

\begin{block*}{Note}
    We will call sublocales which are both open and closed as clopen. % TODO check grammar
\end{block*}

TODO place somewhere StoneSp and StoneFrm and \J.

Notation.
$\downset a^*$ means $\downset (a^*)$ (rightmost unary operator applies first)

For the definition of frames, we see that $\wedge$ is the left/right Galois adjoint, therefore every frame is a complete Heyting algebra, moreover also a pseudocomplemented lattice. The pseudocomplement operation for open set in spacial frames corresponds precisely to the complement of the closure of that set, or the interior of the complement.

\begin{definition}
    Let $L$ be a locale, \DEF{nucleus} $\nu$ on $L$ is a monotone map $\nu\colon L \to L$ having the following four properties:
    \begin{enumerate}[(N1)]
        \item $a \leq \nu(a)$,
        \item $a \leq b \implies \nu(a) \leq \nu(b)$,
        \item $\nu\nu(a) = \nu(a)$, and
        \item $\nu(a \wedge b) = \nu(a) \wedge \nu(b)$.
    \end{enumerate}
\end{definition}

\num\label{p:nuclProp} \textbf{Properties of nuclei.} For a locale $L$ and a nucleus $\nu\colon L \to L$, set $S = \nu(L)$. The set $S$ together with suprema and infima defined
    $$ \bigsqcup a_i = \nu(\bigvee a_i)\quad\text{ and }\quad a \sqcap b = a \wedge b$$

\noindent is a sublocale of $L$. Indeed, we have
$$ (\bigsqcup a_i)\sqcap b = \nu(\bigvee a_i) \wedge \nu(b) = \nu(\bigvee(a_i \wedge b)) = \bigsqcup (a_i \sqcap b).$$

\noindent From (N4) we know, $\sqcap$ is the infimum and from (N2) we know $\bigsqcup$ is the supremum. Hence $\nu(L)$ is a locale, to show that $\nu(L)$ is a sublocale of $L$ for $\nu^*\colon L \to \nu(L)$, defined $a \mapsto \nu(a)$, we will show it is an onto frame homomorphism. It is enough to show
    $$\nu(\bigvee a_i) = \nu(\bigvee \nu(a_i)) \quad( = \bigsqcup \nu(a_i)).$$

    \noindent We have $\bigvee a_i \leq \bigvee \nu(a_i)$ by (N1) and $\nu(\bigvee a_i) \leq \nu(\bigvee \nu(a_i))$ by (N2). On the other hand $\nu(a_i) \leq \nu(\bigvee a_i)$ by (N2). Hence $\bigvee \nu(a_i) \leq \nu(\bigvee a_i)$ and $\nu(\bigvee \nu(a_i)) \leq \nu\nu(\bigvee a_i) = \nu(\bigvee a_i)$ by (N2) and (N3).
