\chapter{Preliminaries}
We assume basic knowledge of Set Theory, mainly basic facts about cardinals, cardinalities, classes and Axiom of Choice.

We refer to the monograph \emph{Frames and Locales} written by Pultr and Picado~\cite{picado2011frames} as a basic introduction to point--free topology.

\begin{block*}{Basic notation}
For sets $X,Y$ and a mapping $f\colon X\to Y$, denote $f[S] = \Set{ f(s) | s\in S}$ and $f^{-1}[M] = \Set{ x\in X | f(x)\in M}$ for $S\subseteq X$ and $M\subseteq Y$.
\end{block*}

\section{Partially ordered set}

\num Let $A$ be a set and let $\leq$ be a binary relation on $A$ ($(\leq) \subseteq A\times A$). We say $(A, \leq)$ is a \DEF{partially ordered set} if the relation $\leq$ is \emph{partial order}, i.e. satisfies:
\begin{enumerate}[(P1)]
    \item \emph{reflexivity:} $a \leq a$ for all $a \in A$,
    \item \emph{antisymmetry:} $a \leq b$ and $b\leq a$ implies $a = b$ for all $a,b \in A$,
    \item \emph{transitivity:} $a \leq b$ and $b\leq c$ implies $a \leq c$ for all $a,b,c \in A$.
\end{enumerate}

Denote by $(A,\leq)\op$ the partially ordered set $(A,\leq')$ such that $(\leq') = \Set{ (b,a) \in A\times A | a\leq b}$.

% TODO better wording
Let $S$ be a subset of $A$.
We say $u$ is an upper bound for $S$ if $u \geq x$ for all $x$ of $S$. We say $s$ is a supremum for $S$ if $s$ is the least upper bound for $S$. We write $s = \bigvee S$.
We say $l$ is an lover bound for $S$ if $l \leq x$ for all $x$ of $S$. We say $i$ is a infimum for $S$ if $i$ is the supremum for $S$ in $A\op$, i.e. if $i$ is the greatest lower bound for $S$. We write $i = \bigwedge S$.

A partially ordered set $(A,\leq)$ is \emph{join semilattice} if each two element subset has a supremum. Similarly, \emph{meet semilattice} has an infimum for each two element subset.
A \DEF{lattice} is partially ordered set which is both join-- and meet-- semilattice. We write $a\vee b$ (\emph{meet}) for the supreum of $\{a,b\}$ and $a\wedge b$ (\emph{join}) for the infimum.

A mapping $f\colon A\to B$ between two partially ordered sets is called \emph{monotone} if $f(x)\leq f(y)$ wherever $x\leq y$. If, moreover, $A$ and $B$ are lattices and $f(x\vee y) = f(x)\vee f(y)$ and $f(x\wedge y) = f(x)\wedge f(y)$ for all $x,y\in A$, we say $f$ is a \emph{lattice homomorphism}.

\num A lattice is called a \DEF{complete lattice} if every subset has an infimum and a supremum.

\begin{proposition*}
    For a lattice $L$ are the following conditions equivalent:
    \begin{itemize}
        \item $L$ is a complete lattice;
        \item every subset of $L$ has a supremum;
        \item every subset of $L$ has an infimum.
    \end{itemize}
\end{proposition*}

For a complete lattice we therefore have two infinitary operations the meet $\bigvee$ and the join $\bigwedge$.
Note that every complete lattice contains two distinguished elements: 0 and 1, where $0 = \bigvee \emptyset$ and $1 = \bigwedge \emptyset$.

A lattice homomorphism $f\colon A\to B$ between two complete lattices is \DEF{complete lattice homomorphism} if $f(\bigvee X) = \bigvee f[X]$ and $f(\bigwedge X) = \bigwedge f[X]$ for each subset $X \subseteq A$.

\num A lattice $D$ is \DEF{distributive} if the following equation holds for all $a,b,c\in D$:
\begin{align*}
    a\vee (b\wedge c) = (a\vee b)\wedge (a\vee c).
\end{align*}

\begin{proposition*}
    For a lattice $D$, the following conditions are equivalent:
    \begin{itemize}
        \item $D$ is a distributive lattice,
        \item $a\vee (b\wedge c) = (a\vee b)\wedge (a\vee c)$ for all $a,b,c \in D$,
        \item For each $a,b,c\in D$ there exists at most one $x$ such that
            \begin{align*}
                a\vee x &= b\\
                a\wedge x &= c.
            \end{align*}
    \end{itemize}
\end{proposition*}

\num Let $A$ and $B$ be two partially ordered sets. Two monotone mappings $f\colon A \to B$ and $g\colon B\to A$ are in \DEF{Galois connection} or are \DEF{Galois adjoing} if
$$ f(x) \leq y\quad\text{iff}\quad x\leq g(y)\quad\text{for all } x\in A, y\in B.$$

\noindent Or equivalently
$$ fg(y) \leq y\quad\text{and}\quad x\leq gf(y)\quad\text{for all } x\in A, y\in B.$$

Then we say $f$ is a left Galois adjoing and a $g$ is a right Galois adjoing.
For a monotone map, its left or right Galois adjoint do not need to exists but if it does, it is uniquely determined.

For two monotone maps in Galois connection $f,g$ we have that
$$ f g f = f\quad\text{and}\quad g f g = g. $$

\begin{proposition*}
    For a monotone map $f\colon A\to B$, if $f$ is a left (resp. right) Galois adjoint
    then $f$ preserves all existing suprema (resp. infima).

    Moreover, the converse implication also holds if $A$ and $B$ are complete lattices.
\end{proposition*}

\num Let $L$ be a lattice with 0. A \emph{pseudocomplement} $a^*$ for an element $a\in L$ is the greatest element $x$ such that
$$ x\wedge a = 0. $$

\noindent Equivalently, $a^*$ is the pseudocomplement of $a$ if
$$ x\wedge a = 0 \quad\text{iff}\quad x\leq a^*\quad\text{for all } x\in L. $$

A lattice is called \DEF{pseudocomplemented}[ lattice] if every element has a pseudocomplement.

\begin{proposition*}\label{p:pseudcomplProperties} Properties of pseudocomplement.
\begin{enumerate}
    \item $a \leq a^{**}$, $a^* = a^{***}$,
    \item $a \leq b \implies a^* \geq b^*$,
    \item $(a \wedge b)^{**} = a^{**} \wedge b^{**}$.
\end{enumerate}
\end{proposition*}

\num Let $L$ be a meet semilattice with 0 and with a binary operation $\to$, we say $L$ is a \DEF{Heyting algebra} if the following holds
$$ a\wedge x\leq b\quad\text{iff}\quad a\leq x\to b\quad\text{for all } a,b,x\in L.$$

Note that each Heyting algebra is pseudocomplemented semilattice with pseudocomplements defined
$$ a^* = a\to 0.$$

As we see from the definition, the operation $x\to (-)$ is the right Galois adjoint to $(-) \wedge x$. Therefore, for each meet semilattice, a Heyting operation, if it is defined, it is uniquely determined.

\begin{proposition*}
    Let $H$ be a Heyting algebra, then
    $$ (\bigvee_i a_i)^* = \bigvee_i a_i^*$$
    holds whenever $\bigvee_i a_i$ exists.
\end{proposition*}

\begin{proposition*}
    A complete lattice admits a Heyting operation iff the following equation holds
    $$ (\bigvee_i a_i)\wedge b = \bigvee_i (a_i\wedge b)\quad\text{for all } a_i, b. $$
\end{proposition*}

\num Let $B$ be a bounded distributive lattice, we say $B$ is a \DEF{Boolean algebra} if for every element $a\in B$ there exists a (\emph{complement}) $a^c\in B$ such that
$$ a\vee a^c = 1\quad\text{and}\quad a\wedge a^c=0. $$

Since $B$ is a distributive lattice, such $a^c$ is uniquely determined. Each Boolean algebra is a Heyting algebra with Heyting operation defined as follows
$$ a\to b = a^c\vee b.$$
Hence, it is also a pseudocomplemented lattice with pseudocomplements equal to complements (from unicity).

Let $f$ be a lattice homomorphism between two Boolean algebras, $f$ is a \DEF{Boolean homomorphisms} if $f$ preserves 0s and 1s.

\begin{proposition*}
    For a Boolean algebra, we have
    $$ (\bigvee_i a_i)^* = \bigwedge_i a_i^*\quad\text{resp.}\quad (\bigwedge_i a_i)^* = \bigvee_i a_i^*,$$
    whenever $\bigvee_i a_i$ resp. $\bigwedge_i a_i$ exists.
\end{proposition*}

\num Let $L$ be a bounded distributive lattice, then a subset $F\subseteq L$ is a \DEF{filter} if
\begin{enumerate}[(F1)]
    \item $0\not\in F$ and $1\in F$,
    \item $a\in F$ and $a\leq b\in L$ implies $b\in F$, and
    \item $a,b\in F$ implies $a\wedge b\in F$.
\end{enumerate}

Similarly, an ideal is a subset $I\subset L$ such that
\begin{enumerate}[(I1)]
    \item $1\not\in I$ and $0\in I$,
    \item $a\in I$ and $a\geq b\in L$ implies $b\in I$, and
    \item $a,b\in I$ implies $a\vee b\in I$.
\end{enumerate}

Let $L$ be a bounded distributive lattice.
A filter $F\subseteq L$ is called \DEF{prime filter} if $a\vee b\in F$ implies $a\in F$ or $b\in F$.
$F$ is called \DEF{completely prime filter} if $\bigvee A\in F$ implies $a\in F$ for some $a\in A$.
A \DEF{principal filter} is any filter of $L$ of the form $\Set{ x | x\geq a}$ for some $a$.

Similarly, an ideal $I$ is called \DEF{prime ideal}, \DEF{completely prime ideal} or \DEF{principal ideal} if $I$ is an prime filter, completely prime filter or principal filter of $L\op$.

\begin{proposition*}
    Let $B$ be a Boolean algebra and a filter $F\subseteq B$. Then the following are equivalent
    \begin{itemize}
        \item $F$ is a maximal filter,
        \item $F$ is a prime filter.
        \item For every $b\in B$, either $b\in F$ or $b^c\in F$.
    \end{itemize}
\end{proposition*}

From the Proposition we see that the definition of a maximal filter and a prime filter coincide for Boolean algebras, hence we call such filter simply an \DEF{ultrafilter}.

% TODO say something
\num When we refer to Axiom of Choice, we usually refer to its equivalent form -- Zorn's Lemma:
\begin{center}\em
    \emph{(AC)}\quad Let $X$ be a non-empty partially ordered set such that every non-empty chain has an upper bound, then $X$ has at least one maximal element.
\end{center}

\noindent By Boolean Ultrafilter Theorem we mean the following choice principle
\begin{center}\em
    \emph{(BUT)}\quad Let $B$ be a Boolean algebra and let $F$ be a filter of $B$, then there exists a prime filter $G$ extending $F$.
\end{center}

\noindent Which is equivalent to
\begin{center}\em
    \emph{(BUT')}\quad Let $B$ be a Boolean algebra and let $F$ and $I$ be a filter and ideal of $B$ such that $F\cap I = \emptyset$, then there exists a prime filter $G$ extending $F$ such that $G\cap I=\emptyset$.
\end{center}

\noindent Finally, the last choice principle we will need to know is Axiom of Countable Dependent Choice
\begin{center}\em
    \emph{(ACC)}\quad Let $R$ be a binary relation on a set $X$ such that for every $a\in X$ there exists $b\in X$ satisfying $aRb$, then there exist a countable sequence $(a_i)_{i=1}^\infty$ such that $a_iRa_{i+1}$ for all $i=1,2,\dots$.
\end{center}

Axiom of Choice is ``stronger'' than Boolean Ultrafilter Theorem or Axiom of Countable Dependent Choice, Axiom of Choice logically implies both of them.

\section{Category Theory}
\num A category $\p C$ is a class of objects ($\obj \p C$), a class of morphisms ($\morph \p C$) and two mappings $\dom,\codom\colon \morph\p C\to \obj\p C$ (\emph{domain} and \emph{codomain}) satisfying conditions (C1)--(C3) bellow.

Notation: For a morphism $f\in \morph \p C$ to point out that $A = \dom(f)$ and $B = \codom(f)$, we write $f\colon A\to B$ or $A \stackrel{f}{\to} B$.

\begin{enumerate}[(C1)]
    \item Let $f\colon A\to B, g\colon B\to C$ be two morphisms of $\p C$, then there exists their \emph{composition}, the morphism $g\cdot f\colon A\to C$.
    \item Morphism composition satisfies the \emph{associativity law}: $(f\cdot g)\cdot h = f\cdot (g\cdot h)$ whenever are the composition defined.
    \item For each object $A\in \obj \p C$ there exists an \emph{identity morphism} $1_A$ satisfying $1_A\cdot f = f$ and $g\cdot 1_A = g$ whenever are the compositions defined.
\end{enumerate}

We say a category is \emph{small} if $\morph \p C$ is a set.
A \emph{dual category} $\p C\op$ of a category $\p C$ is the category with the same class of objects as $\p C$ has and with all morphisms of $\p C$ reversed, i.e. for each morphism $f\colon A\to B\in \morph \p C$, the morphism $\hat f\colon B\to A$ belongs to $\morph \p C\op$. The composition $\diamond$ is defined as
$$ \hat f\diamond \hat g = \widehat{g\cdot f}. $$

To simplify the notation even more, we will write $f g$ instead of $f\cdot g$, $A\in \p C$ instead of $A\in \obj \p C$ and similarly for $\morph$, when there is no danger in confusion.

\begin{block*}{Examples}
    The most natural example of a category is the category $\categoryStyle{Set}$ of all sets, all maps between them and composition defined as map composition.

    Another examples: the category \Bool{} of all Boolean algeras and all Boolean homomorphisms; an arbitrary partially ordered set $(X,\leq)$ is a category with the set $X$ as objects and one morphism between $x,y\in X$ whenever $x\leq y$ (composition holds thanks to transitivity of $\leq$).

    Observe that the dual category of $(X,\leq)$ is precisely the category of the partially ordered set with the opposite order to $(X,\leq)$.
\end{block*}

\num A morphism $f$ is a \emph{monomorphism} whenever $fg = fh$ implies $g = h$. Analogously, $f$ is an \emph{epimorphism} if $gf = hf$ implies $g = h$. Finally, $f$ is an \emph{isomorphism} if there exists a $f^{-1}$ such that
$$ f\cdot f^{-1} = 1\quad\text{and}\quad f^{-1}\cdot f = 1.$$

\num For categories $\p C, \p D$ the mappings $F\colon \obj \p C\to \obj \p D$ and $F\colon \morph \p C\to \morph\p D$ constitute a \emph{(covariant) functor} if
$$ F(f)\colon F(A)\to F(B)\quad\text{for any }f\colon A\to B\in \morph\p C, $$
$$ F(1_A) = 1_{F(A)}\quad\text{for any }A \in \obj \p C,\quad\text{and}\quad F(gh)=F(g)F(h).$$

\begin{block*}{Examples}
    Let $\p C$ be a category, the \DEF{identity functor} on $\p C$ is defined by the mappings
    $$\Id_{\p C}(A) = A\colon \obj\p C\to \obj\p C,\quad\text{and}\quad \Id_{\p C}(f) = f\colon \morph\p C\to \morph\p C.$$

    Let $\p D$ be a non-empty category and let $T$ be an object of $\p D$, then the \emph{constant functor} is defined as follows
    $$K(A)=T\quad\text{and}\quad K(f)=1_T\quad\text{for all } A,f\in \p C.$$
\end{block*}

We say a functor is \DEF{contravariant}[ functor] if it is of the form $F\colon \p C\to \p D\op$. The composition of functors $F\colon \p C\to \p D$ and $G\colon \p D\to \p E$ is denoted by $G\circ F\colon \p C\to \p E$.

\num Let $F,G\colon \p C\to \p D$ be two functors, a collection of morphisms $m_* = (m_A)_{A\in \obj\p C}$ is a \emph{natural transformation} between $F$ and $G$, $m_*\colon F\natto G$, if
$$ m_A\colon F(A)\to G(A),\quad\text{for all } A\in \obj\p C, $$

% TODO first use of the word ``diagram'', it should be explained
\noindent and the following diagram commutes
\begin{diagram}
    F(A)\ar{r}{m_A}\ar{d}{F(f)} & G(A)\ar{d}{G(f)}\\
    F(B)\ar{r}{m_B} & G(B)
\end{diagram}
for all $f\colon A\to B$, morphisms of $\p C$.

If a natural transformation $m_*$ is a collection of isomorphism, we say $m_*$ is a \emph{natural equivalence}. For two functors $F$ and $G$, if there exists a natural equivalence $m_*\colon F\natto G$, we say $F$ and $G$ are \emph{naturally equivalent} and write $F\cong G$.

\num A \emph{limit} of a \emph{diagram} (that is, of a functor $D\colon \p D\to \p C$ for a small category $\p D$) is a constant functor $L\colon \p D\to \p C$ and a natural transformation $l_*\colon L\natto D$ such that $l_*$ is universal among natural transformations from constant functors.

Or in other words, for any constant functor $K\colon \p D\to \p C$ and a natural transformation $k_*\colon K\natto D$ there exists an unique natural transformation $m_*\colon K\natto L$ such that the following diagram commutes
\begin{diagram}
    L\ar{r}{l_*} & D\\
    K\ar[dashed]{u}{m_*}\ar{ur}{k_*} &
\end{diagram}

\noindent (in the category $[\p D,\p C]$ of all functors from $\p D$ to $\p C$ and natural transformations as morphisms).

A \emph{colimit} is defined dually to limit, that is, with all natural transformations from the definition of limit reversed. % TODO? use [\p D,\p C]\op ?
We say a category is \emph{(co)complete} if there is a (co)limit for every diagram.

\begin{block*}{Examples}
    A well--known example of a limit in \categoryStyle{Set} is the (cartesian) product of sets $\prod_{i\in I} X_i$ together with projections $(\prod_{i\in I} X_i\to X_j)_{j\in I}$.

    Similarly, a natural example of colimit in \categoryStyle{Set} is the coproduct, the disjoint union, of sets $\coprod_{i\in I} X_i$ together with injections $(X_j\to \coprod_{i\in I} X_i)_{j\in I}$.
\end{block*}

\num Functors $F\colon \p C\to \p D$ and $G\colon \p D\to \p C$ are \emph{adjoint} (with $F$ on the left and $G$ on the right) if there exist a natural transformations (\emph{units of adjuction})
$$ \lambda_*\colon FG\natto \Id_{\p D}\quad\text{and}\quad \rho_*\colon \Id_{\p C}\natto GF,$$

\noindent such that the following compositions of natural transformations
$$ F\xrightarrow{~F(\rho_*)~} FGF\xrightarrow{~\lambda_{F(*)}~} F\quad\text{and}\quad G\xrightarrow{~\rho_{G(*)}~} GFG\xrightarrow{~G(\lambda_*)~} G$$

\noindent are equal to identity natural transformations on $F$ and on $G$, or more precisely: $F(\rho_A)\cdot \lambda_{F(A)} = 1_{F(A)}$ and $\rho_{G(B)}\cdot G(\lambda_B) = 1_{F(B)}$ for all $A\in\p C$ and $B\in\p D$.

\begin{proposition*}
    Right adjoinits preserve limits and left adjoinits preserve colimits.
\end{proposition*}

Two categories $\p C,\p D$ are said to be \emph{isomorphic}, $\p C\cong \p D$, if there exists adjoint functors $F\colon \p C\to \p D$ and $G\colon \p D\to \p C$ with units of adjunction consisting of natural equivalences.

\num A category $\p C$ is said to be \emph{subcategory} of a category $\p D$ if $\obj\p C\subseteq \obj\p D$, $\morph\p C\subseteq \morph\p D$ and the composition of morphisms in $\p C$ coincide with that in $\p D$. Moreover, if $\Set{f | f\colon A\to B\in \p C} = \Set{f | f\colon A\to B\in \p D}$ for every $A, B\in \obj\p C$, we say $\p C$ is a \emph{full subcategory} of $\p D$.

A full subcategory $\p C$ of a category $\p D$ is \emph{reflexive} (or \emph{coreflexive}) if the embedding functor $J\colon \p C\xrightarrow{~\subseteq~} \p D$ is a right (or left) adjoint.

\begin{block*}{Examples}
    TODO
\end{block*}

\begin{proposition*}
    Each reflexive subcategory of a (co)complete category is (co)complete. Similarly for coreflexive categories.
\end{proposition*}

\section{Topology and point--free Topology}
homeomorphism, closure for spaces

\num Let $X$ be a set and $\tau$ an arbitrary set of subsets of $X$, then the pair $(X,\tau)$ is a \emph{topological space} if
\begin{enumerate}[(T1)]
    \item $\emptyset, X \in \tau$,
    \item $\p M \subseteq \tau$ implies $\bigcup \p M \in \tau$, and
    \item for $U, V \in \tau$, also $U\cap V\in \tau$.
\end{enumerate}

Subsets of $X$ which are elements of $\tau$ are called open and their complements are called closed.

Let $(X,\tau), (Y,\rho)$ be topological spaces and a mapping $f\colon X\to Y$ is said to be \emph{continuous with respect to $(X,\tau)$ and $(Y,\rho)$} if
$$ f^{-1}[O] \in \tau,\quad\text{for all } O\in \rho.$$

\noindent Denote by \Top{} the category of all topological spaces and continuous mappings.

As we see from the conditions (T1)--(T3), the set $\tau$ ordered by inclusion is a complete lattice. Moreover, we have $(\bigcup_i U_i)\cap V = \bigcup_i (U_i\cap V)$ for $U_i, V\in \tau$.
Therefore, we can generalize the notion of topological space:

Let $L$ be a complete lattice, we say $L$ is a \DEF{frame} if
$$ (\bigvee A)\wedge b = \bigvee_{a\in A} ( a\wedge b ) $$
for any $A\subseteq L$ and $b\in L$. We see that frames are just complete Heyting algebras and, consequently, pseudocomplemented lattices.
The pseudocomplements are given by the following formula
$$ a^* = \bigvee \Set{ x | x\wedge a = 0}. $$

\noindent For an open set of some space, taking pseudocomplements in the corresponding frame of open sets is the same as taking the interior of the complement.

Let $L$ and $M$ be frames and $f\colon L\to M$ a monotone map, $f$ is said to be a \DEF{frame homomorphism} if $f$ preserves all joins, finite meets, the top and the bottom (1 and 0).
Denote by \Frm{} the resulting category of all frames and frame homomorphisms.

We have the obvious (contravariant) functor $\Omega\colon \Top\to \Frm$
$$ \Omega(X,\tau) = \tau\quad\text{and}\quad \Omega(f)\colon U\mapsto f^{-1}[U],$$
for $(X,\tau), f \in \Top$.

A frame homomorphism $f$ is monotone mapping between complete lattices and preserves all suprema. Hence, it has the right adjoint $f^*$, a \emph{localic map}.
Denote the category of frames and localic maps as morphisms by $\Loc$. We will refer to objects of $\Loc$ as \emph{locales}. Trivially $\Loc\cong\Frm\op$.

Again, we have a functor $\Lc\colon \Top\to \Loc$ (covariant this time)
$$ \Lc(X,\tau) = \tau\quad\text{and}\quad \Lc(f) = \Omega(f)^*.$$

\num adjunction $\Sigma$ and $\Omega$, Spatiality, some spatiality theorems (II.5.3, Hofmann--Lawson's Duality), points!

\num completeness and cocompleteness of \Top{} and \Frm{}

\num {\bf Separation axioms.} For a topological space $(X,\tau)$, we say it is
\begin{itemize}
    \item[$T_0$:] if for every two distinct points $x,y\in X$ there exist an open $U\in \tau$ such that $x\in U\not\ni y$ or $x\not\in U\ni y$.
    \item[$T_1$:] if for every two distinct points $x,y\in X$ there exist an open $U\in \tau$ such that $x\in U\not\ni y$.
    \item[$T_2$:] if for every two distinct points $x,y\in X$ there exist disjoint open sets $U,V\in \tau$ separating $x$ and $y$, i.e. $x\in U\not\ni y$ and $x\not\in V\ni y$.
    \item[$T_3$:] if for each point $x\in X$ and each closed $F\subseteq X$ such that $x\notin F$ there exist two disjoint open sets separating $x$ and $F$.
    \item[$T_{3.5}$:] if for each point $x\in X$ and each closed $F\subseteq X$ such that $x\notin F$ there exist a continuous function $f\colon X\to [0,1]$ separating $x$ and $F$, i.e. $f(x) = 0$ and $f[F] = 1$.
    \item[$T_{4}$:] if for every two disjoin closed subsets of $X$ there exist two disjoint open sets separating them, or equivalently, there exists a continuous function separating them.
\end{itemize}

And we see that
$$ T_4~\&~T_1 \implies T_{3.5}~\&~T_0 \implies T_3~\&~T_0 \implies T_2 \implies T_1 \implies T_0. $$

\noindent Topological spaces satisfying $T_2$. $T_3$ and $T_0$, $T_{3.5}$ and $T_0$, or (just) $T_4$ are called, in order, \emph{Hausdorff}, \emph{regular}, \emph{completely regular}, or \emph{normal}.
The condition $T_3$ is equivalent to condition
$$U = \bigcup \Set{ V\subseteq X | \closure V \subseteq U},\quad\text{for all open } U\in \tau.$$

Observe, $\closure V\subseteq U$ if and only if $V^*\cup U = X$ in the frame $\Omega(X)$. Define,
$$ a\rbelow b\quad\equiv\quad a^*\vee b = 1,$$

\noindent A frame $L$ is said to be \emph{regular} if
$$ a = \bigvee \Set{ x | x\rbelow a},$$
\noindent for all $a\in L$. Then, a space $X$ is regular if and only if the frame $\Omega(X)$ is regular.
Analogously for complete regularity, define
\begin{align*}
    a\crbelow b\quad\equiv\quad &\text{exists } a_i\in L,\text{ for all } i\in \mathbb Q\cap [0,1],\text{ such that}\\
        & a_0 = a,~ a_1 = b\text{ and } a_i\rbelow a_j\text{ whenever } i<j.
\end{align*}

\noindent Then, a frame $L$ is \emph{completely regular} if
$$ a = \bigvee \Set{ x | x\crbelow a},$$

\noindent for all $a\in L$. Again, a space $X$ is completely regular if and only if the frame $\Omega(X)$ is.

Finally, a frame is said to be \emph{normal} if
$$ a\vee b = 1\quad\implies\quad \exists c\text{ such that } a\vee c = 1\text{ and }c^*\vee b = 1. $$

\begin{observation*}\label{p:rbellowProperties}\label{p:crbellowProperties} Let $\lhd\in \{\rbelow, \crbelow\}$, then
\begin{enumerate}
    \item $a'\leq a\lhd b\leq b;$ implies $a'\lhd b'$,
    \item $a_1\lhd b_1$ and $a_2\lhd b_2$ implies $a_1\wedge a_2 \lhd b_1 \wedge b_2$ and $a_1\vee a_2 \lhd b_1 \vee b_2$,
    \item $a \lhd b$ implies $b^* \lhd a^*$.
\end{enumerate}
\end{observation*}

Similarly to topological spaces, we have:
\num\label{p:normalRegular} Any normal and regular frame is completely regular: For $x\rbelow y$, the $x^*\vee y = 1$ holds, from normality there exists a $q$, such that $x^*\vee q = 1$ and $q^*\vee y = 1$, thus $x\rbelow q\rbelow y$. Hence, under (CDC): $\Set{x | x\rbelow b} = \Set{x | x\crbelow b}$

\num A \emph{subspace} of a topological space $(X,\tau)$ (that is, a space $(Y\subseteq X,\restr{\tau}{Y})$ where $\restr{\tau}{Y} = \Set{ Y\cap O | O\in \tau}$) determines the one-one inclusion (and continuous) mappings $j\colon Y\subseteq X$. We have the associated one-one localic map $\Lc(j)\colon \Lc(Y)\subseteq \Lc(X)$ and its adjoint is the onto frame homomorphisms $\Omega(j)\colon \Omega(X)\to \Omega(Y)$.

% TODO better wording (starting with ``if the inclusion ...'')
Thus, for a locales $L$ and $S\subseteq L$, $S$ is said to be a \DEF{sublocale} of $L$ if the inclusion (one-one) mapping $j\colon S\subseteq L$ is a localic map and, for $s,t\in S$, $s\leq t$ whenever $j(s)\leq j(t)$.
Or equivalently, there exists an onto frame homomorphisms $j^*\colon L\to S$.

One has an another useful characterisation of sublocales, let $L$ be a frame, a subset $S\subseteq L$ is sublocale if and only if
\begin{enumerate}[(S1)]
    \item $S$ is closed under all meets, and
    \item $x\to s\in S$ for each $s\in S$ and $x\in L$.
\end{enumerate}

Denote by $\Sl(L)$ the set of all sublocales of $L$. Then, $\Sl(L)$ ordered by inclusion is a co-frame (in other words, $\Sl(L)\op$ is a frame). The empty sublocale $\emptyFrm = \{1\}$ is the least sublocale and $L$ is the greatest.

A sublocale is said to be \DEF{open}[ sublocale] resp. \DEF{closed}[ sublocale] if it is of the form
$$ \open(a) = \Set{ a\to x | x\in L}\quad\text{resp.}\quad \clos(c) = \upset a.$$

\noindent The inspiration for this definition comes naturally from topological spaces. For a topological space $(X,\tau)$ and an open subset $O\in \tau$, we have the inclusion mapping
$$ j\colon (O, \Set{U\in\tau | U\subseteq O})\to (X,\tau) $$
and an onto frame homomorphisms $\Omega(j)\colon U\mapsto U\cap O$. Hence, we have the formula 
$$ \Lc(j)\colon U\mapsto O\to U,$$
for the associated localic embedding, the right adjoint to $\Omega(j)$. ``As one could expect'', $\clos(a)$ and $\open(a)$ are mutually complemented in $\Sl(L)$.

For a sublocale $S\subseteq L$, its \DEF{closure}[ of a sublocale], the least closed sublocale containing $S$, is given by the following formula
$$ \closure S = \upset (\bigwedge S). $$
From that, we immediately see that a sublocale $S\subseteq L$ is \emph{dense} (that is, $\closure S = L$) iff $0_L\in S$.

\begin{proposition*}
    Preimage of closed (resp. open) sublocale under a localic map is again a closed (resp. open).
\end{proposition*}

\begin{block*}{Note}
    We will call sublocales which are both open and closed as clopen. % TODO check grammar
\end{block*}

\begin{definition}
    Let $L$ be a locale, \DEF{nucleus} $\nu$ on $L$ is a monotone map $\nu\colon L \to L$ with the following four properties:
    \begin{enumerate}[(N1)]
        \item $a \leq \nu(a)$,
        \item $a \leq b \implies \nu(a) \leq \nu(b)$,
        \item $\nu\nu(a) = \nu(a)$, and
        \item $\nu(a \wedge b) = \nu(a) \wedge \nu(b)$.
    \end{enumerate}
\end{definition}

\num\label{p:nuclProp} \textbf{Properties of nuclei.} For a locale $L$ and a nucleus $\nu\colon L \to L$, set $S = \nu(L)$. The set $S$ together with suprema and infima defined
    $$ \bigsqcup a_i = \nu(\bigvee a_i)\quad\text{ and }\quad a \sqcap b = a \wedge b$$

\noindent is a sublocale of $L$. Indeed, we have
$$ (\bigsqcup a_i)\sqcap b = \nu(\bigvee a_i) \wedge \nu(b) = \nu(\bigvee(a_i \wedge b)) = \bigsqcup (a_i \sqcap b).$$

\noindent From (N4) we know that $\sqcap$ is the infimum and from (N2) we know that $\bigsqcup$ is the supremum. Hence $\nu(L)$ is a locale, to show that $\nu(L)$ is a sublocale of $L$ for $\nu^*\colon L \to \nu(L)$, defined $a \mapsto \nu(a)$, we will show that it is an onto frame homomorphism. It is enough to show that
    $$\nu(\bigvee a_i) = \nu(\bigvee \nu(a_i)) \quad( = \bigsqcup \nu(a_i)).$$

    \noindent We have $\bigvee a_i \leq \bigvee \nu(a_i)$ by (N1) and $\nu(\bigvee a_i) \leq \nu(\bigvee \nu(a_i))$ by (N2). On the other hand $\nu(a_i) \leq \nu(\bigvee a_i)$ by (N2). Hence $\bigvee \nu(a_i) \leq \nu(\bigvee a_i)$ and $\nu(\bigvee \nu(a_i)) \leq \nu\nu(\bigvee a_i) = \nu(\bigvee a_i)$ by (N2) and (N3).

    Thus, a nuclei deteremine sublocales. One has more, there is an one-one correspondence between nuclei and sublocales.

\noindent\dotfill

TODO place somewhere StoneSp and StoneFrm and \J.

Notation.
$\downset a^*$ means $\downset (a^*)$ (rightmost unary operator applies first)

