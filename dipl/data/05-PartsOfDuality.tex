\chapter{Parts of duality}

In this chapter, we analyse some parts of the Stone correspondence for Stone frames and Boolean algebras.
Namely, we will show that the category of $\kappa$--complete Boolean algebras is in the correspondence with the category of $\kappa$--basically disconnected Stone frames.

We will also show how the restriction of the class of Boolean homomorphisms affects the morphisms on the side of Stone frames.

\section{$\kappa$--complete Boolean algebras}
For the rest of this section, let $\kappa$ be a fixed infinite regular cardinal.

\begin{definition}
    A lattice is \DEFSYM[completeKL]{}{$\kappa$--complete}[ lattice], if any subset of cardinality less than $\kappa$ has a supremum and an infimum. A homomorphism is said to be \DEFSYM[completeKH]{}{$\kappa$--complete}[ homomorphism] if it preserves all suprema and infima of subsets of cardinality less than $\kappa$.

    Let $L$ be a Stone frame and let $s$ be an element of $L$. We say that $s$ is \DEFSYM[generated]{}{$\kappa$--generated} if $s = \bigvee S$ for some $S \subseteq \BcO L$ of cardinality less than $\kappa$.
\end{definition}

\num The following Lemma will be often expedient for computations.
\begin{lemma*}\label{p:idealsFrame}
    Let $B$ be a $\kappa$--complete Boolean algebra and let $I$ be a $\kappa$--generated element of $\JO B$. Then $I^* =\;\downset (\bigvee I)^c$ in $\JO B$. In particular, for $a \in B$, $(\downset a)^* = \downset a^c$.
\end{lemma*}
\begin{proof}
    $I$ is of the form $\bigvee \Set{\downset s | s\in S}$ for some $S\subseteq B$ of cardinality less than $\kappa$. Since $b\wedge s = 0$ iff $b\leq s^c$, we have
    \begin{align*}
        I^* = (\bigvee \Set{\downset s | s\in S})^* &= \bigvee \Set{ \downset b | b\in B,~b\wedge s = 0, \text{ for all }s\in S} \\
                &= \bigvee \set{ \downset b | b\in B,~b\leq \bigwedge_{s\in S} s^c} \\
                &= \bigvee \set{ \downset b | b\in B,~b\leq (\bigvee S)^c} \\
                &= \downset (\bigvee S)^c = \downset (\bigvee I)^c.
    \end{align*}
    \paragraph{}
    \vspace{-2em}
\end{proof}

\begin{definition}

    Let $L$ be a Stone frame, we say that $L$ is \DEFSYM[basically disconnected]{}{$\kappa$--basically disconnected} if $m^{**} \vee m^* = 1$ for all $\kappa$--generated elements $m \in L$.
\end{definition}

\begin{lemma}\label{p:kappaCompleteBA}
    If $B$ is a $\kappa$--complete Boolean algebra then $\J B$ is a $\kappa$--basically disconnected Stone frame.
\end{lemma}
\begin{proof}
    By Lemma~\ref{p:complIdeal}, we know that complemented ideals are precisely the principal ideals of $B$. For a subset $M$ of $B$ of cardinality less than $\kappa$ set $I = \bigvee \Set{ \downset a | a \in M }$. We will show that $I^{**} \vee I^* = 1_{\J B}$.

    Set $m = \bigvee M$ and $J = \downset m$. Observe that $I^{**} = J$: Trivially from Lemma~\ref{p:idealsFrame} we have $I^{**} \subseteq J^{**} = J$. The other inclusion, the $J \subseteq I^{**}$, follows from
    $$ I^* = \bigcup \Set{ \downset a | \downset a \wedge I = 0_{\J B}} = \bigcup \Set{ \downset a | a \wedge m = 0 } = (\downset m)^* = J^*.$$

    Consequently, $I^{**} \vee I^* = J \vee J^* = \downset m \vee \downset m^c = 1_{\J B}$.
\end{proof}

\num\label{p:kappaCompleteBAfromMeets}The following Lemma will be often very useful. We will use it without further reference.
\begin{lemma*}
    Let $B$ be a Boolean algebra such that each of its subset of cardinality at most $\kappa$ has a supremum. Then $B$ is $\kappa$--complete.

    Consequently, any Boolean homomorphism preserving all $\kappa$--meets (or $\kappa$--joins) is $\kappa$--complete.
\end{lemma*}
\begin{proof}
    Let $S$ be an arbitrary subset of $B$ such that $|S| < \kappa$. Set $M = (\bigvee \Set{ b^c | b \in S})^c$, we will show that $M$ is the infimum of $S$.

    Take an $a \in S$. We have $a \wedge M = a^{cc}\wedge (\bigvee \Set{ b^c | b \in S})^c = (a^c\vee \bigvee \Set{ b^c | b \in S})^c = M$. Hence $a \geq M$.

    Now suppose $m \leq a$ for all $a \in S$. Then $m\wedge M = (m^c\vee \bigvee \Set{b^c | b \in S})^c = (m^c)^c = m$. Hence $m \leq M$.
\end{proof}

\begin{lemma}\label{p:kappaComplStoneFrm}
    If $L$ is a $\kappa$--basically disconnected Stone frame then $\BcO L$ is a $\kappa$--complete Boolean algebra. The joins in $\BcO L$ are defined by the following formula

    $$\bigsqcup M = (\bigvee M)^{**}.$$
\end{lemma}
\begin{proof}
    For $M$ a subset of $\BcO L$ of cardinality less than $\kappa$, set $m = \bigvee M$. Since $L$ is $\kappa$--basically disconnected, we have $m^{**} \vee m^* = 1$ and therefore $m^{**} \in \BcO L$. So $m^{**}$ is an upper bound for $M$ in $\BcO L$.

    Now, let $n$ be an arbitrary upper bound for $M$ in $\BcO L$. Then $n$ is also an upper bound in $L$, but $m \leq n$ since $m$ is the supremum of $M$ in $L$. This gives us the desired relation $m^{**} \leq n^{**} = n$, hence $m^{**}$ is the supremum of $M$ in $\BcO L$.
\end{proof}

\num From Lemma~\ref{p:kappaCompleteBAfromMeets} and Lemma~\ref{p:kappaComplStoneFrm}, we conclude that $\kappa$--complete Boolean algebras are in (Stone) correspondence with $\kappa$--basically disconnected Stone frames. The same holds for topological spaces, a topological space is $\kappa$--basically disconnectedness iff any union of cardinality less than $\kappa$ clopen sets has open closure. Hence, we have a duality between $\kappa$--complete Boolean algebras and $\kappa$--basically disconnected Stone spaces~\cite{monk1989handbook}.

Now, we will focus on morphisms.

\begin{observation}
    Let $f\colon A \to B$ be a $\kappa$--complete Boolean homomorphism and let $I$ be a $\kappa$--generated ideal of $A$, then
    $$(\JO f)(I^*) = (\JO f)(I)^*.$$
\end{observation}
\begin{proof}
    The result is obtained by simple computation,
    \begin{align*}
        (\JO f)(I^*) &= (\JO f)(\downset (\bigvee I)^c) & \text{(Lemma~\ref{p:idealsFrame})}\\
                &= \downset f[\downset (\bigvee I)^c] = \downset f((\bigvee I)^c) \\
                &= \downset f(\bigvee I)^c & \text{($f$ is a Boolean homomorphism)} \\
                &= \downset (\bigvee f[I])^c & \text{($\kappa$--completeness of $f$)} \\
                &= \downset (\bigvee \downset f[I])^c = \downset (\bigvee (\JO f)(I))^c \\
                &= (\JO f)(I)^*.
    \end{align*}

    The use of $\kappa$--completeness of $f$ in the fourth step is valid because $I$ is $\kappa$--generated. In particular, $I = \bigvee \Set{ \downset s | s \in S }$ for some $S$ a subset of $A$ of the cardinality less then $\kappa$.
\end{proof}

\begin{definition}
    Let $f\colon L \to M$ be a homomorphism between Stone frames. We say that $f$ is a \DEFSYM[basically complete]{}{$\kappa$--basically complete} if $f(a^*) = f(a)^*$ holds for all $\kappa$--generated elements $a \in L$.
\end{definition}

In other words, the last observation states that the functor \JO{} sends any $\kappa$--complete Boolean homomorphism to a $\kappa$--basically complete frame homomorphism. As we will see in the following Lemma, morphisms of the image of $\kappa$--complete part of Boolean algebras in Stone correspondence are characterised precisely this way.

\begin{lemma}
    Let $f\colon L \to M$ be a $\kappa$--basically complete frame homomorphism, then $\BcO f\colon \BcO L \to \BcO M$ is a $\kappa$--complete Boolean homomorphism.
\end{lemma}
\begin{proof}
    Let $A$ be an arbitrary subset of $\BcO L$ such that $|A| < \kappa$. We have
    \begin{align*}
        (\BcO f)(\bigsqcup A) &= f((\bigvee A)^{**}) \\
            &= f((\bigvee A)^*)^* & \text{($(\bigvee A)^*$ is complemented)}\\
            &= f(\bigvee A)^{**}  & \text{($\bigvee A$ is $\kappa$--generated)}\\
            &= (\bigvee f[A])^{**} & \text{($f$ is a frame homomorphism)}\\
            &= \bigsqcup f[A] = \bigsqcup (\BcO f)[A].
    \end{align*}
    Therefore $\BcO f$ is $\kappa$--complete.
\end{proof}

\num\label{p:kappaCompleteThm}From Lemmas~\ref{p:kappaCompleteBA} and~\ref{p:kappaComplStoneFrm}, we see that the restriction of Stone correspondence to subcategories of $\kappa$--complete Boolean algebras on one side and $\kappa$--basically disconnected Stone frames on the other side (without any restriction on morphisms) is still a duality of categories.

If we set \DEF{\kComplBool{}} to be the category of $\kappa$--complete Boolean algebras and $\kappa$--complete Boolean homomorphisms and set \DEF{\kBDStoneFrm} to be the category of $\kappa$--basically disconnected Stone frames and $\kappa$--basically complete frame homomorphisms, then it is sound (by the previous two Lemmas) to define two functors
\begin{align*}
    \DEFSYM{JK}{\JK}&\colon \kComplBool \to \kBDStoneFrm, \text{ and} \\
    \DEFSYM{BcK}{\BcK}&\colon \kBDStoneFrm \to \kComplBool,
\end{align*}

\noindent as the restriction of \JO{} and \BcO{} to the corresponding subcategories. And we get the following

\begin{theorem*}\label{p:kappaDuality}
    The functors \JK{} and \BcK{} constitute an isomorphism between categories \\ \kComplBool{} and \kBDStoneFrm.
\end{theorem*}
\begin{proof}
    The only thing we need to show is that the morphisms of natural equivalences for identity functors and functors $\BcO\JO$ and $\JO\BcO$ are morphisms of our categories.

    For the first part, we will show that $i_B\colon B \to \BcO\JO(B)$ is a $\kappa$--complete Boolean homomorphisms for any $\kappa$--complete Boolean algebra $B$. Let $B'$ be an arbitrary subset of $B$ of cardinality less than $\kappa$, then by straightforward computation we get

    $$
        \bigsqcup_{b\in B'} i_B(b) = \bigsqcup_{b\in B'} \downset b = (\bigvee_{b\in B'} \downset b)^{**} = \downset (\bigvee_{b\in B'} b) = i_B(\bigvee_{b\in B'} b),
    $$

    \noindent where the third equality follows from~\ref{p:idealsFrame}.

    For the second part, we need to show that $v_L\colon \JO\BcO(L) \to L$ is $\kappa$--basically complete for any $\kappa$--basically disconnected Stone frame $L$. We will show that $v_L$ is $\lambda$--basically complete for any regular cardinal $\lambda$. Take any $I \in \JO\BcO(L)$, we have

    \begin{align*}
        v_L(I^*) &= v_L(\bigvee \Set{ \downset s | \downset s \wedge I = 0 })
                  = \bigvee \Set{ v_L(\downset s) | \downset s \wedge I = 0 } \\
                 &= \bigvee \Set{ s | s \wedge v_L(I) = 0 } = v_L(I)^*.
    \end{align*}
\end{proof}

% TODO discuss corresponding morphisms in Top

\section{Complete Boolean algebras}

By Theorem~\ref{p:kappaCompleteThm} we know that there is an isomorphism between the category of $\kappa$--complete Boolean algebras and $\kappa$--complete Boolean homomorphisms and the category of $\kappa$--basically disconnected Stone frames and $\kappa$--basically complete frame homomorphisms. Since, there is no limitation or upper bound for the cardinal $\kappa$ in Theorem~\ref{p:kappaDuality}, let us have a look at the part of the correspondence where $\kappa$ is arbitrary large.

\num It is interesting to see, how the Stone frame part of the correspondence looks like. Take any object $L$ and an element $a\in L$. From zero--dimensionality, we know that $a$ is $\lambda$--generated for some regular cardinal $\lambda$, but $L$ is $\lambda$--basically disconnected so that

$$a^{**}\vee a^* = 1$$

\noindent holds. Since $a$ has been chosen arbitrarily we see that $L$ extremally disconnected. Similarly, any morphism of this part of the duality $f$ with domain $L$ is $\lambda$--basically complete, hence the following holds
\begin{align}
    f(a^*) = f(a)^*.\label{e:no0}
\end{align}

\noindent We will call morphisms satisfying (\oldref{e:no0}) for all elements \DEF{basically complete} frame homomorphisms.

Thus, the resulting category is the category of extremally disconnected Stone frames and basically complete frame homomorphisms. We will denote this category by \DEF{\ExtrStoneFrm}.

\num On the side of Boolean algebras, we have that Boolean algebras with arbitrary large joins and Boolean homomorphisms preserving such joins. Thus the observed category is the category of complete Boolean algebras and complete Boolean homomorphisms. We will denote it by \DEF{\ComplBool}.

\num\label{p:completePartUnits}Recall that, the definitions of $i_B\colon B \to \BcO\JO(B)$ and $v_L\colon \JO\BcO(L) \to L$ are

    $$i_B\colon b \mapsto \downset b\quad\text{and}\quad v_L\colon I \mapsto \bigvee I.$$

    For a complete Boolean algebra $B$, take any $\Set{ a_i | i\in I}$ subset of $B$. Then
    $$
    \bigsqcup_{i\in I} i_B(a_i) = \bigsqcup_{i\in I} \downset a_i = (\bigvee_{i\in I} \downset a_i)^{**} = \downset (\bigvee_{i\in I} a_i)^{cc} = \downset (\bigvee_{i\in I} a_i) = i_B(\bigvee_{i\in I} a_i).
    $$

    \noindent Consequently, $i_B$ is a complete Boolean homomorphism, that is a morphisms of \ComplBool{}.

    For any extremally disconnected Stone frame $L$, from the proof of Theorem~\ref{p:kappaCompleteThm}, we know that $v_L$ is $\lambda$--basically complete frame homomorphism for all regular cardinals $\lambda$, therefore $v_L$ is also basically complete and is a~morphisms of $\ExtrStoneFrm$.

     Therefore, \emph{$(v_L)_{L\in \ExtrStoneFrm}$ and $(i_B)_{B\in \ComplBool}$ are collection of morphisms of categories \ExtrStoneFrm{} and \ComplBool{}}.

\num As a conclusion, we can define functors
\begin{align*}
    \DEFSYM{JI}{\JI}&\colon \ComplBool \to \ExtrStoneFrm, \text{ and} \\
    \DEFSYM{BcI}{\BcI}&\colon \ExtrStoneFrm \to \ComplBool,
\end{align*}

\noindent as the restriction of \JO{} and \BcO{} to the corresponding categories and obtain the following

\begin{theorem*}
    The categories \ExtrStoneFrm{} and \ComplBool{} are isomorphic.
\end{theorem*}

The whole situation is depicted in Figure~\oldref{f:inclusionsOfDuality} where $\kappa$ is a regular cardinals greater or equal to $\omega_1$, $\sigma$--\ComplBool{} denotes the category $\omega_1$--\ComplBool{} and $\sigma$--\categoryStyle{BDStoneFrm} denotes the category $\omega_1$--\categoryStyle{BDStoneFrm}.

\begin{figure}[t]
\begin{diagram}[row sep=0.7cm]
    \StoneFrm
        \ar[yshift=0.2em]{r}{\scalebox{1.5}\BcO}
    & \Bool
        \ar[yshift=-0.2em]{l}{\scalebox{1.5}\JO} \\
    \\
    \sigma\text{--}\categoryStyle{BDStoneFrm}
        \ar[yshift=0.2em]{r}{\scalebox{1.5}{$\Bc_\sigma$}}
        \ar{uu}{\InclUp}
    & \sigma\text{--}\ComplBool
        \ar{uu}{\InclUp}
        \ar[yshift=-0.2em]{l}{\scalebox{1.5}{$\J_\sigma$}} \\
    \DotsUp
        \ar{u}{\InclUp}
    & \DotsUp
        \ar{u}{\InclUp} \\
    \kBDStoneFrm
        \ar[yshift=0.2em]{r}{\scalebox{1.5}\BcK}
        \ar{u}{\InclUp}
    & \kComplBool
        \ar{u}{\InclUp}
        \ar[yshift=-0.2em]{l}{\scalebox{1.5}\JK} \\
    \DotsUp
        \ar{u}{\InclUp}
    & \DotsUp
        \ar{u}{\InclUp} \\
    \ExtrStoneFrm
        \ar[yshift=0.2em]{r}{\scalebox{1.5}{\BcI}}
        \ar{u}{\InclUp}
    & \ComplBool
        \ar{u}{\InclUp}
        \ar[yshift=-0.2em]{l}{\scalebox{1.5}\JI}
\end{diagram}
    \caption{Diagram of categories in Stone correspondence.}\label{f:inclusionsOfDuality}

    \vspace{0.5em}
    \centering
    (Note that the subcategories indicated as $\subseteq$ are not full.)
\end{figure}

\num Here we present an alternative point of view on what we already know from the discussion above.

\begin{proposition*}\label{p:complBool2}
    Let $B$ be a complete Boolean algebra. Then the frame $\J B$ is an extremally disconnected Stone frame.
\end{proposition*}
\begin{proof}
    In a Boolean algebra the relations $\rbelow$ and $\leq$ coincide. Moreover, each complete Boolean algebra is a (completely regular extremally disconnected) Boolean frame. Thus $\J{B}$ equals $\C{B}$.

    From Lemma~\ref{p:extrDiscPreserv}, we know that compactification preserves extremal disconnectedness and therefore $\J{B}$ is also extremally disconnected. We also know that $\J{B}$ is a Stone frame by Lemma~\ref{p:JisFunctor}.
\end{proof}

The proof of the previous Proposition unveils an important fact about the complete part of Stone correspondence. It shows that this part of the correspondence is an isomorphism of two subcategories of the category of frames (on Boolean side, without any restriction to frame homomorphisms) and that the correspondence is provided by a purely topological construction by compactification.
