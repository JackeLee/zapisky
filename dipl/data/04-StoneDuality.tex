\chapter{Stone duality}

In 1936 Marshall Stone published a paper called ``The Theory of Representations of Boolean Algebras''~\cite{stone1936theory}, describing a duality of two categories, the category of Boolean algebras and Boolean homomorphisms and the category of compact Hausdorff zero--dimensional topological spaces and continuous maps. At that time he could not use the language of category theory, of course.

The word ``duality'' means that one category is equivalent to the dual of the other category.
The duality at that time was a big surprise because it has shown similarities between a objects with nice algebraic structure, the Boolean algebras, with, another mathematical objects witch was thought to have no algebraic structure at all, topological spaces.

In this chapter, we will first discuss the Stone duality in the point--free context. Since the adjunction between topological spaces and frames is contravariant, the construction will be covariant and we speak of the Stone correspondence. In the second part we will obtain the standard Stone duality for topological spaces.

\section{Stone correspondence for frames}

\begin{definition}
    We say a frame is \DEF{Stone frame} if it is compact and zero--dimensional.

    By \DEF{\StoneFrm{}} denote the category of Stone frames and frame homomorphisms.
\end{definition}
% TODO shifted text ("Definition")

\begin{definition}
    Let $B$ be a Boolean algebra. Define \DEFSYM{JO}{$\JO{}$}$B$ to be the set of all ideals of $B$.
    For a Boolean homomorphism $f\colon A \to B$, define $\JO f\colon \JO A \to \JO B$ as
    $$(\JO f)(I) = \downset f[I].$$
\end{definition}

\begin{block*}{Note}
    We do not assume $B$ complete, hence we depart from the notation used in \ref{p:frameOfIdeals}.
\end{block*}

\begin{lemma}\label{p:complIdeal}
    Let $B$ be a Boolean algebra and $I \in \JO B$. Then $I$ is complemented iff $I = \downset b$ for some $b \in B$.
\end{lemma}
\begin{proof}
    Let $I$ be a complemented ideal. Since $I \vee I^c = 1_{\JO B}$ there exists $a \in I$ and $b \in I^c$ such that $a \vee b = 1$. From $I \wedge I^c = 0_{\JO B}$ we have $a \wedge b = 0$ and $I \wedge \downset b = 0_{\JO B}$.
    From the uniques of complements we get $I^c = \downset b$, indeed $I \vee \downset b = 1_{\JO B}$ and $I \wedge \downset b = 0_{\JO B}$. Using the same argument we get $I = \downset a$.

    The converse implication is trivial, since $\downset a \vee \downset a^c = 1_{\JO B}$ and $\downset a \wedge \downset a^c = 0_{\JO B}$.
\end{proof}

\begin{proposition}\label{p:JisFunctor}
    $\JO\colon \Bool \to \StoneFrm$ is a functor.
\end{proposition}
\begin{proof}
    $\JO B$ is a compact frame by~\ref{p:frameOfIdeals}. Lemma~\ref{p:complIdeal} implies that $\downset a$ is complemented and $\downset a \rbelow \downset a$ for all $a \in B$. Thus for any ideal $I \in \JO B$, we obtain
    $$ I = \bigvee \Set{ \downset a | a \in I } = \bigvee \Set{ \downset a | \downset a \rbelow I } \subseteq \bigvee \Set{ J | J \rbelow I } \subseteq I.$$

\noindent Hence, $\JO B$ is zero--dimensional. If $f$ is a Boolean homomorphism, then $\JO f$ is a frame homomorphism (for the same reason as the $\C f$ is in \ref{p:compactificationFunctor}).
\end{proof}

\begin{definition}
    Let $L$ be a Stone frame. Define $\DEFSYM{BcO}{$\BcO{}$} L$ to be the set of all complemented elements of $L$ and for a frame homomorphism between Stone frames $f\colon L \to M$ define
    $$\BcO f = \restr{f}{\BcO L}\colon \BcO L \to \BcO M.$$
\end{definition}

From the fact that a homomorphic image of a complemented element is a complemented element and since join or meet of two complemented elements is again complemented (for complemented elements $a$ and $b$, $a^c \vee b^c$ is the complement of $a\wedge b$ and $a^c\wedge b^c$ is the complement of $a\vee b$), one can see that $\BcO L$ is a Boolean algebra and $\BcO f$ is well--defined Boolean homomorphism.

\begin{observation}
    $\BcO\colon \StoneFrm \to \Bool$ is a functor.
\end{observation}

\num\label{p:BoolEquivalence}For a Boolean algebra $B$, define \DEFSYM{istar}{$i_B$}$\colon B \to \BcO\JO(B)$ as follows
    $$i_B\colon b \mapsto \downset b.$$
    The definition is sound by Lemma \ref{p:complIdeal} and $i_B$ is a Boolean homomorphism: indeed $\downset a \vee \downset b = \downset (a \vee b)$, $\downset a \wedge \downset b = \downset (a \wedge b)$, and $\downset 1 = B$ respectively $\downset 0 = \{0\}$ is top respectively bottom of $\BcO\JO(B)$.

From Lemma \ref{p:complIdeal}, we also see that $i_B$ is an isomorphism and the following diagram commutes

\begin{diagram}
    A \ar{r}{i_A} \ar{d}[swap]{f} & \BcO\JO(A) \ar{d}{\BcO\JO(f)}\\
    B \ar{r}{i_B}                 & \BcO\JO(B)
\end{diagram}

\noindent for any Boolean homomorphism $f\colon A \to B$ (for any $a\in A$, $\BcO\JO(f)\,i_A (a) = \BcO\JO(f)(\downset a) = \downset f[\downset a] = \downset f(a) = i_B\,f (a)$). From previous we have

\begin{proposition*}
    The collection $i=(i_A)_A$ of Boolean homomorphisms forms a natural equivalence between $\BcO\JO$ and the identity functor on \Bool.
\end{proposition*}

\num\label{p:StoneFrmEquivalence}Similarly, for a Stone frame $L$, we have a mapping \DEFSYM{vstar}{$v_L$}$\colon \JO\BcO(L) \to L$ defined as
    $$v_L\colon I \mapsto \bigvee I,$$
    and a mapping in the opposite direction $\iota\colon L \to \JO\BcO(L)$:
    $$\iota\colon e \mapsto \downset e \cap \BcO L.$$

We can see that both $v_L$ and $\iota$ are monotone maps, $v_L \iota = \id_L$ (by zero--dimensionality of $L$) and $\id_{\JO\BcO(L)} \subseteq \iota v_L$. Therefore $v_L$ is the left Galois adjoint to $\iota$ and hence $v_L$ preserves all suprema.
Since $\bigvee I_1 \wedge \bigvee I_2 = \bigvee \Set{a_1 \wedge a_2 | a_i \in I_i} \leq \bigvee \Set{ a | a \in I_1 \cap I_2 } = \bigvee (I_1 \cap I_2) \leq \bigvee I_1 \wedge \bigvee I_2$, $v_L$ also preserves finite infima which makes $v_L$ a frame homomorphism.

Finally, $\id_{\JO\BcO(L)} = \iota v_L$: take any $x \in \iota v_L(I)$. From the definitions we immediately see that $x \leq \bigvee I$ and $x$ is complemented in $L$. By the fact that
    $$1 = x \vee x^c \leq \bigvee I \vee x^c$$
    and by compactness of $L$ there is a finite $F \subseteq I$ such that $\bigvee F \vee x^c = 1$. Since $x = 1 \wedge x = (\bigvee F \vee x^c) \wedge x = (\bigvee F \wedge x) \vee (x^c \wedge x) = \bigvee F \wedge x$ we get that $x \leq \bigvee F$ and therefore $x \in I$.

From previous observations, we know that $v_L$ is an isomorphism of $L$ and $\JO\BcO(L)$. Also by direct computation, for any homomorphism of Stone frames $f\colon L \to M$, we see that the following diagram commutes

\begin{diagram}
    \JO\BcO(L) \ar{d}[swap]{\JO\BcO(f)} \ar{r}{v_L} & L \ar{d}{f} \\
    \JO\BcO(M) \ar{r}{v_M}    & M
\end{diagram}

\noindent (for any $I\in \JO\BcO(L)$, $(v_M\cdot\JO\BcO(f)) (I) = \bigvee (\downset f[I]) = \bigvee f[I] = f(\bigvee I) = f\,v_L (I)$). Again, as a consequence of previous paragraphs we obtain

\begin{proposition*}
    The collection $v=(v_L)_L$ of frame homomorphisms is a natural equivalence between $\JO\BcO$ and the identity functor on \StoneFrm.
\end{proposition*}

\num Using previous facts we obtain the main result of this section.

\begin{theorem*}\label{p:stonefrmBoolIso}
    Functors \BcO{} and \JO{} constitute an equivalence of categories \StoneFrm{} and \Bool.
\end{theorem*}

\section{Stone duality for spaces}

The classical spatial version of Stone duality follows from the Stone correspondence between Stone frames and Boolean algebras from previous section. We have the duality between the category of compact zero--dimensional frames and the category compact zero--dimensional spaces (restricted Hofmann--Lawson's Duality, Theorem~\ref{p:HLDuality}).

However the Hofmann--Lawson's Duality depends on the Axiom of Choice. In this section we will prove the Stone duality by assuming only the Boolean Ultrafilter Theorem.

\num Similarly to frames:
\begin{definition*}
    We say a topological space is \DEF{Stone space} if it is compact, Hausdorff and zero--dimensional.

    By \DEF{\StoneSp{}} denote the category of Stone spaces and continuous mappings.
\end{definition*}

\num Let $B$ be a Boolean algebra. Define
    $$
    X_B  = \Set{F \subseteq B | F \text{ is an ultrafilter of } B}
    \quad\text{ and }\quad
    \tau_B = \Set{ W_I | I \in \JO B},
    $$
\noindent where $W_I$ is the set $\Set{ F \in X_B | F \cap I \neq \emptyset }$. The pair $(X_B, \tau_B)$ is a topological space:

\begin{enumerate}[label=(T\arabic*)]
    \item $\tau_B$ contains $\emptyset = W_{\downset 0}$ and $X_B = W_{\downset 1}$.

    \item $\bigcup_i W_{J_i} = W_{\bigvee_i J_i}$: $\subseteq$ holds trivially since $\bigcup J_i \subseteq \bigvee J_i$. On the other hand, for an ultrafilter $F$ such that $F\cap \bigvee_i J_i \neq \emptyset$, there is an $e = \bigvee E$ where $E$ is a finite subset of $\bigcup J_i$ and $e \in F$. Since $F$ is prime, there is an $e' \in E$ such that $e' \in F$ and therefore $J_j \ni e'$ and $F\cap J_j \neq \emptyset$ for some $j$.

    \item $W_I \cap W_J = W_{I\wedge J}$: $F \in W_I\cap W_J$ iff $F\cap I \neq \emptyset$ and $F\cap J \neq \emptyset$ iff $F \cap I \cap J \neq \emptyset$ iff $F\in W_{I\cap J}$.
\end{enumerate}

\noindent We will denote the topological space $(X_B, \tau_B)$ by \DEFSYM{S B}{$\Sp B$}.

\begin{lemma}\label{p:SpObjects}
    Let $B$ be a Boolean algebra. Then $\Sp B$ is a Stone space.
\end{lemma}
\begin{proof}
    Compactness follows directly from compactness of $\JO B$. Note that the complemented elements are of the form $W_{\downset a}$ for some $a\in B$: $W_{\downset a}\cup W_{\downset a^c} = W_{\downset (a\vee a^c)} = X_B$ and $W_{\downset a}\cap W_{\downset a^c} = W_{\downset (a\wedge a^c)} = \emptyset$. Since every open set is an union of sets of the form $W_{\downset a}$, the $\Sp B$ is zero--dimensional.

    To show that $\Sp B$ is also Hausdorff, take any ultrafilters $E \neq F$. Without any loss of generality there is some $e \in E \setminus F$. We have $e\vee e^c = 1$ and since $F$ is an ultrafilter we have $e^c \in F$. We also have $e\wedge e^c = 0$ and so $W_{\downset e}$ and $W_{\downset e^c}$ separates $E$ and $F$.
\end{proof}

\begin{lemma}
    Let $f\colon A \to B$ be a Boolean homomorphism and let $F$ be an ultrafilter of $B$. Then $f^{-1}[F]$ is an ultrafilter of $A$.
\end{lemma}
\begin{proof}
    Set $E = f^{-1}[F]$. First, we will show that $E$ is a filter. Trivially, $1 \in E$ and $0 \not\in E$. For $x, y \in E$, there are $x', y' \in F$ such that $x' = f(x)$ and $y' = f(y)$. Hence $x'\wedge y' = f(x) \wedge f(y) = f(x\wedge y)$ and $x\wedge y \in E$. For the upwards closeness, take any $x \in E$ and $y \geq x$. $y \in E$ as $f(y) \geq f(x) \in F$.

    $E$ is also an ultrafilter. For $a, b\in A$ such that $a\vee b \in E$, $f(a)\vee f(b) = f(a\vee b) \in F$ and so $f(a)$ or $f(b)$ is in $F$ and therefore $a$ or $b$ is in $E$.
\end{proof}

\num\label{p:SpMorphisms}Let $f\colon A \to B$ be a Boolean homomorphism. Denote by $\Sp f\colon \Sp B \to \Sp A$ the map defined as
    $$\Sp f\colon F \mapsto f^{-1}[F].$$

\noindent From the previous Lemma, we see that the definition is sound. We will show that $\Sp f$ is also continuous. Take any $W_I \in \tau_B$. We have
\begin{align*}
    (\Sp f)^{-1}[W_I] &= \Set{ (\Sp f)^{-1}(F) | F \cap I \neq \emptyset } \\
                      &= \Set{ E | \exists F \subseteq B \text{ ultrafilter, } (\Sp f)(E)=F, \text{ and } F \cap I \neq \emptyset } \\
                      &= \Set{ E | f^{-1}(E)\cap I \neq \emptyset }
                       = \Set{ E | E\cap f[I] \neq \emptyset } \\
                      &= \Set{ E | E\cap \downset f[I] \neq \emptyset }
                       = W_{(\JO h)(I)}.
\end{align*}

\begin{theorem}
    $\Sp\colon \Bool \to \StoneSp$ is a functor.
\end{theorem}
\begin{proof}
    Follows immediately from~\ref{p:SpObjects} and~\ref{p:SpMorphisms}.
\end{proof}

\num Our situation is as follows

\begin{diagram}
    \StoneFrm \ar[bend left=15]{rr}{\BcO} & & \Bool{} \ar[bend left=15]{ll}{\JO} \ar{ldd}{\Sp} \\
    \\
    & \StoneSp \ar{uul}{\Omega}
\end{diagram}

\begin{proposition}
    The collection of morphisms $\pi_B\colon \JO B \to \Omega\Sp(B)$ defined for $B \in \Bool$ by $I \mapsto W_I$, constitutes a natural equivalence $\JO \cong \Omega\circ\Sp$.\ACP
\end{proposition}
\begin{proof}
    From the definition of $\Sp B$, we can see that $\pi_B$ is an onto frame homomorphism.

    Take any $I \neq J$, $I,J \in \JO B$. Without loss of generality take $e \in I \setminus J$. Now the filter $\upset e$ is disjoint with the ideal $J$ and by Boolean Ultrafilter Theorem there exists an ultrafilter $U \supseteq \upset e$ disjoint with $J$. The intersection $U \cap I$ is not empty, it contains $e$, and therefore $W_I \neq W_J$.

    The naturalness of $\pi$ follows from the commutativity of the following diagram

    \begin{diagram}
        \JO A \ar{r}{\pi_A} \ar{d}[swap]{\JO f} & \Omega\Sp(A) \ar{d}{\Omega\Sp(f)}\\
        \JO B \ar{r}{\pi_B}                    & \Omega\Sp(B)
    \end{diagram}

    \noindent for any Boolean homomorphism $f\colon A \to B$. Indeed, by~\ref{p:SpMorphisms}, we have
    $$ \Omega\Sp(f)(W_I) = (\Sp f)^{-1}[W_I] = W_{(\JO h)(I)},$$

    \noindent for any $I\in \JO B$.
\end{proof}

\begin{conclusion}
    $\BcO\circ\Omega\circ\Sp \cong \Id_\Bool$.\ACP
\end{conclusion}
\begin{proof}
    By the previous Proposition and the Stone correspondence for Stone frames and Boolean algebras (Theorem~\ref{p:stonefrmBoolIso}), we have $\BcO\circ\Omega\circ\Sp \cong \BcO\circ\JO \cong \Id_\Bool$.
\end{proof}

\begin{proposition}
    The collection of morphisms $\rho_X\colon X \to \Sp\BcO\Omega(X)$ defined for $X \in \StoneSp$ by $x \mapsto F_x = \Set{U\text{ clopen} | x\in U }$, constitutes a natural equivalence $\Id_\StoneSp \cong \Sp\circ\BcO\circ\Omega$.
\end{proposition}
\begin{proof}
    We will show that $\rho_X$ is a homeomorphism.
    \begin{itemize}
        \item $\rho_X$ is one--one: For two points $x_1 \neq x_2$ of $X$, from Hausdorff property there are $U_1, U_2$ such that $U_1\cap U_2 = \emptyset$ and $x_i\in U_i$. From zero--dimensionality of $X$ there are two clopen subsets $M_1 \subseteq U_1, M_2 \subseteq U_2$, and $x_i \in M_i$. Hence $F_{x_1} \neq F_{x_2}$.

        \item $\rho_X$ is onto: Take any $F$ ultrafilter. We will prove $F \subseteq F_x$ for some $x\in X$. Suppose it is not the case, then $\bigcap F = \emptyset$. Further, $X = X \setminus \bigcap F = \bigcup_{U\in F} (X\setminus U)$. Therefore $\p C = \Set{X \setminus U | U\in F}$ is a cover and from compactness there exist a finite subcover $\p C'\subseteq \p C$. Then, $X = \bigcup_{X \setminus U\in \p C'} (X\setminus U) = X \setminus \bigcap_{(X \setminus U)\in \p C'} U$. This is a contradiction, since $\bigcap_{(X \setminus U)\in \p C'}$ is empty set.

        And so $F \subseteq F_x$; but $F$ is an ultrafilter, hence maximal, and hence $F = F_x$.

        \item $\rho_X$ and $\rho_X^{-1}$ are continuous: From previous ($\bullet$), we know that each ultrafilter of $X$ is of the form $F_x$, for some $x$. Hence
        \begin{align*}
            \rho_X^{-1}[W_I] &= \Set{ \rho_X^{-1}(F_x) = x | F_x \cap I \neq \emptyset}
                = \Set{ x | x \in M \in I } = \bigcup I \in \tau_X, \text{ and}\\
            \rho_X[W] &= \Set{ F_x | x \in M \subseteq W, M \text{ is clopen}} \\
                      &= \Set{ F | F \cap (\downset W \cap \BcO\Omega(X)) \neq \emptyset} = W_{\downset W \cap \BcO\Omega(X)} \in \tau_{\Sp\BcO\Omega(X)}.
        \end{align*}
    \end{itemize}

    Finally, $\rho = (\rho_X)_X$ is a natural equivalence. The following diagram commutes
    \begin{diagram}
        X \ar{r}{\rho_X} \ar{d}[swap]{f} & \Sp\BcO\Omega(X) \ar{d}{\Sp\BcO\Omega(f)}\\
        Y \ar{r}{\rho_Y}                 & \Sp\BcO\Omega(Y)
    \end{diagram}
    \noindent for any continuous map $f\colon X \to Y$. Indeed,
    \begin{align*}
        \Sp\BcO\Omega(f)(F_x)
            &= (\BcO\Omega(f))^{-1}[F_x] = \Omega(f)^{-1}[F_x] \\
            &= \Set{ \Omega(f)^{-1}(M) | M \text{ clopen}, x \in M } \\
            &= \Set{ N | x \in M, M \text{ clopen}, f[M] = N } \\
            &= \Set{ N | f(x) \in N } = F_{f(x)} = \rho_Y(f(x)).
    \end{align*}
\end{proof}

\num From the previous, we have
\begin{theorem*}
    Functors $\Omega\circ\BcO$ and \Sp{} are mutually inverse. Thus, categories \Bool{} and $\StoneSp^{\text{op}}$ are equivalent.\ACP
\end{theorem*}

\num Similarly to the previous section, we obtained an equivalence of categories. However, the adjoint functors of this equivalence are contravariant, thus we obtained a duality of two categories instead of a correspondence.

Note that we can make the correspondence, from the previous section, into duality by the duality between the category of frames and the category of locales.

\section{Notes on constructivity}

One can check that the whole correspondence between Stone frames and Boolean algebras has been proved constructively. Therefore, the necessity of a choice principle needed in classical Stone duality between Stone spaces and Boolean algebras is necessary only to show that spaces constructed from Boolean algebras have enough points.

Analogously, the compactification of completely regular frames described in the previous chapter was constructive; its counterpart for topological spaces is equivalent to Boolean Ultrafilter Theorem~\cite{banaschewski1984stone}.

To be more precise, in the proof of the compactification we used the fact that compact regular frames are completely regular (Proposition~\ref{p:compRegIsCR}).
We construct an interpolative relation $\crbelow$ using the Axiom of Countable Dependent Choice (CDC), but by Banaschewski and Pultr~\cite{banaschewski2002constructive}, we can avoid using CDC by working with strongly regular frames instead of completely regular. The whole construction can then be made constructive, even in the sense of topos theory.

Strongly regular frames are frames in which, for every element $x$, the following equation holds
$$ x = \bigvee \Set{ y | y\;(\rbelow)_o\;x }, $$

\noindent where $(\rbelow)_o$ is the largest interpolative relation contained in $\rbelow$. Such a relation can be constructed as the union of all interpolative relations contained in $\rbelow$. Under CDC, complete regularity is precisely the same as strong regularity~\cite{banaschewski2002constructive}.
