\chapter{Stone duality}

Stone duality is a pioneer work by/classical result of M. H. Stone, it is duality of two categories the category of Boolean algebras and Boolean homomorphisms and the category of compact, Hausdorff, zero--dimensional topological spaces and continuous maps. At that time, without the language of Category Theory.
% TODO cite

The duality means ... contravariant

The duality at that time was a big surprise, because it shows similarities between a objects with nice algebraic structure with another mathematical objects witch was thought to has no algebraic structure at all -- topological spaces.

In this chapter, we will first reconstruct classical Stone duality for frames. Since the category of frames is in contravariant position to category of topological spaces, the whole construction will be covariant, hence the name Stone correspondence. And in the second part, we will reconstruct the classical Stone duality from this correspondence.

\section{Stone correspondence for \StoneFrm}

\begin{definition}
    We say a frame is \DEF{Stone frame} if it is compact and zero--dimensional.

    By \DEF{\StoneFrm{}} denote the category of Stone frames and frame homomorphisms.
\end{definition}
% TODO shifted text ("Definition")

\begin{definition}
    Let $B$ be a Boolean algebra. Define \DEFSYM{JO}{$\JO{}$}$B$ to be the set of all ideals of $B$.
    For a Boolean homomorphism $f\colon A \to B$, define $\JO f\colon \JO A \to \JO B$ as
    $$(\JO f)(I) = \downset f[I].$$
\end{definition}

\begin{lemma}\label{p:complIdeal}
    Let $B$ be a Boolean algebra and $I \in \JO B$. Then $I$ is complemented iff $I = \downset b$ for some $b \in B$.
\end{lemma}
\begin{proof}
    Let $I$ be a complemented ideal. Since $I \vee I^c = 1_{\JO B}$ there exists $a \in I$ and $b \in I^c$ such that $a \vee b = 1$. From $I \wedge I^c = 0_{\JO B}$ we have $a \wedge b = 0$ and $I \wedge \downset b = 0_{\JO B}$.
    From uniques of complements we get $I^c = \downset b$, indeed $I \vee \downset b = 1_{\JO B}$ and $I \wedge \downset b = 0_{\JO B}$. Using the same argument we get $I = \downset a$.

    Converse implication is trivial since $\downset a \vee \downset a^c = 1_{\JO B}$ and $\downset a \wedge \downset a^c = 0_{\JO B}$.
\end{proof}

\begin{proposition}\label{p:JisFunctor}
    $\JO\colon \Bool \to \StoneFrm$ is a functor.
\end{proposition}
\begin{proof}
    $\JO B$ is trivially a compact frame. Lemma \ref{p:complIdeal} implies that $\downset a$ is complemented and $\downset a \rbelow \downset a$ for all $a \in B$. Thus for any ideal $I \in \JO B$ we obtain
    $$ I = \bigvee \Set{ \downset a | a \in I } \subseteq \bigvee \Set{ J | J \rbelow I } \subseteq I,$$

\noindent hence $\JO B$ is zero--dimensional. The rest is easy.
\end{proof}

\begin{definition}
    Let $L$ be a Stone frame. Define $\DEFSYM{Bc}{$\BcO{}$} L$ to be the set of all complemented elements of $L$ and for a frame homomorphism $f\colon L \to M$ define
    $$\BcO f = \restr{f}{\BcO L}\colon \BcO L \to \BcO M.$$
\end{definition}
% TODO somehow mention that $f$ is between stone spaces

From the fact that a homomorphic image of a complemented element is a complemented element and since join or meet of two complemented elements is again complemented (to see that, let $a$ and $b$ be two complemented elements, then
    \begin{align*}
        (a\wedge b) \vee (a^c \vee b^c)
            &= (a\vee (a^c\vee b^c))\wedge (b\vee (a^c\vee b^c))
            = 1 \wedge 1 = 1, \text{ and} \\
        (a\wedge b) \wedge (a^c \vee b^c)
            &= ((a\wedge b)\wedge a^c)\vee ((a\wedge b)\wedge b^c)
            = 0 \vee 0 = 0;
    \end{align*}
and similarly $a^c\wedge b^c$ is complement of $a\vee b$.), one can see that $\BcO f$ is well--defined Boolean homomorphism.

% TODO some comment's before this statement
\begin{observation}
    $\BcO\colon \StoneFrm \to \Bool$ is a functor.
\end{observation}

\num\label{p:BoolEquivalence} For a Boolean algebra $B$, define \DEFSYM{istar}{$i_B$}$\colon B \to \BcO\JO(B)$ as follows
    $$i_B\colon b \mapsto \downset b.$$
    The definition is sound by Lemma \ref{p:complIdeal} and $i_B$ is a Boolean homomorphism, indeed $\downset a \vee \downset b = \downset (a \vee b)$, $\downset a \wedge \downset b = \downset (a \wedge b)$, $\downset 1 = B$ respectively $\downset 0 = \{0\}$ is top respectively bottom of $\BcO\JO(B)$, and complements are preserved thanks to distributivity of $\BcO\JO(B)$.

From Lemma \ref{p:complIdeal}, we also see, $i_B$ is an isomorphism and therefore the following diagram commutes

\begin{diagram}
    A \ar{r}{i_A} \ar{d}[swap]{f} & \BcO\JO(A) \ar{d}{\BcO\JO(f)}\\
    B \ar{r}{i_B}                 & \BcO\JO(B)
\end{diagram}

\noindent for any Boolean homomorphism $f\colon A \to B$ (for any $a\in A$, $\BcO\JO(f)\,i_A (a) = \BcO\JO(f)(\downset a) = \downset f[\downset a] = \downset f(a) = i_B\,f (a)$). From previous we have

\begin{proposition*}
    The collection $i_*$ of Boolean homomorphisms forms a natural equivalence between $\BcO\JO$ and the identity functor on \Bool.
\end{proposition*}

\num\label{p:StoneFrmEquivalence} Similarly for $L$, a Stone frame, we have a mapping \DEFSYM{vstar}{$v_L$}$\colon \JO\BcO(L) \to L$ defined as
    $$v_L\colon I \mapsto \bigvee I,$$
    and a mapping in the opposite direction $\iota\colon L \to \JO\BcO(L)$:
    $$\iota\colon e \mapsto \downset e \cap \BcO L.$$

We can see that both $v_L$ and $\iota$ are monotone maps, $v_L \iota = \id_L$ (by zero--dimensionality of $L$) and $\id_{\JO\BcO(L)} \subseteq \iota v_L$, therefore $v_L$ is the left Galois adjoint to $\iota$, hence $v_L$ preserves all suprema.
Since $\bigvee I_1 \wedge \bigvee I_2 = \bigvee \Set{a_1 \wedge a_2 | a_i \in I_i} \leq \bigvee \Set{ a | a \in I_1 \cap I_2 } = \bigvee (I_1 \cap I_2) \leq \bigvee I_1 \wedge \bigvee I_2$, $v_L$ also preserves finite infima which makes it a frame homomorphism.

Finally, $\id_{\JO\BcO(L)} = \iota v_L$: take any $x \in \iota v_L(I)$. From the definitions we immediately see $x \leq \bigvee I$ and $x$ is complemented in $L$. By the fact that
    $$1 = x \vee x^c \leq \bigvee I \vee x^c$$
    and by compactness of $L$ there is a finite $F \subseteq I$ such that $\bigvee F \vee x^c = 1$. Since $x = 1 \wedge x = (\bigvee F \vee x^c) \wedge x = (\bigvee F \wedge x) \vee (x^c \wedge x) = \bigvee F \wedge x$ we get that $x \leq \bigvee F$ and therefore $x \in I$.

Previous observations give us the fact that $v_L$ is an isomorphism of $L$ and $\JO\BcO(L)$, also by direct computation, for any homomorphism of Stone frames $f\colon L \to M$, the following diagram commutes

\begin{diagram}
    \JO\BcO(L) \ar{d}[swap]{\JO\BcO(f)} \ar{r}{v_L} & L \ar{d}{f} \\
    \JO\BcO(M) \ar{r}{v_M}    & M
\end{diagram}

\noindent (for any $I\in \JO\BcO(L)$, $v_M\,\JO\BcO(f) (I) = \bigvee (\downset f[I]) = \bigvee f[I] = f(\bigvee I) = f\,v_L (I)$). Again, as conclusion of previous paragraphs we obtain

\begin{proposition*}
    The collection $v_*$ of frame homomorphisms is a natural equivalence between $\JO\BcO$ and the identity functor on \StoneFrm.
\end{proposition*}

\num Using previous facts we get the main result of this section.

\begin{theorem*}\label{p:stonefrmBoolIso}
    Functor \BcO{} is the right adjoint to the functor \JO{} with $i_*$ as unit of adjunction and $v_*$ as counit. Moreover, \BcO{} and \JO{} constitute an isomorphism of categories \StoneFrm{} and \Bool.
\end{theorem*}

As we will see in the next chapter, \JO{} corresponds precisely to compactification of Boolean algebras the same way as the space of ultrafilters is part of Stone duality and compactification for spaces. (TODO better wording)

TODO (note about AC:) As one can check, the whole correspondence is given in constructive setting. No use of Axiom of Choice unlike in classical case, \dots

\section{Stone duality for \StoneSp}

\num Similarly to frames:
\begin{definition*}
    We say a topological space is \DEF{Stone space} if it is compact, Hausdorff and zero--dimensional.
\end{definition*}

\num Let $B$ be a Boolean algebra, define
    $$
    X_B  = \Set{F \subseteq B | F \text{ is an ultrafilter on } B}
    \quad\text{ and }\quad
    \tau_B = \Set{ O_I | I \in \JO B},
    $$
\noindent where \DEFSYM{OI}{$O_I$} is the set $\Set{ F \text{ ultrafilter} | F \cap I \neq \emptyset }$. Now, $(X_B, \tau_B)$ is a topological space:

\begin{enumerate}[label=(T\arabic*)]
    \item $\tau_B$ contains $O_{\downset 0} = \emptyset$ and $O_{\downset 1} = X_B$.

    \item $\bigcup_i O_{J_i} = O_{\bigvee_i J_i}$: $\supseteq$ holds trivially since $\bigcup J_i \subseteq \bigvee J_i$. On the other hand, for an ultrafilter $F$ such that $F\cap \bigvee_i J_i \neq \emptyset$, there is an $e = \bigvee E$ where $E$ is a finite subset of $\bigcup J_i$ and $e \in F$. Since $F$ is prime, there is an $e' \in E$ such that $e' \in F$ and therefore $J_j \ni e'$ and $F\cap J_j \neq \emptyset$ for some $j$.

    \item $O_I \cap O_J = O_{I\wedge J}$: $F \in O_I\cap O_J$ iff $F\cap I \neq \emptyset$ and $F\cap J \neq \emptyset$ iff $F \cap I \cap J \neq \emptyset$ iff $F\in O_{I\cap J}$.
\end{enumerate}

\noindent We will denote \DEFSYM{S B}{$\Sp B$} to be $(X_B, \tau_B)$.

\begin{lemma}\label{p:SpObjects}
    Let $B$ be a Boolean algebra, $\Sp B$ is a Stone space.
\end{lemma}
\begin{proof}
    Compactness follows directly from compactness of $\JO B$. Notice, complemented elements are of the form $O_{\downset a}$ for some $a\in B$: $O_{\downset a}\cup O_{\downset a^c} = O_{\downset (a\vee a^c)} = X_B$ and $O_{\downset a}\cap O_{\downset a^c} = O_{\downset (a\wedge a^c)} = \emptyset$. Since every open set is an union of sets of the form $O_{\downset a}$, the $X_B$ is zero--dimensional.

    To show $X_B$ is also Hausdorff, take any ultrafilters $E \neq F$. Without lost of generality there is some $e \in E \setminus F$. We have $e\vee e^c = 1$, since $F$ is an ultrafilter we have $e^c \in F$. We also have $e\wedge e^c = 0$ and so $O_{\downset e}$ and $O_{\downset e^c}$ separates $E$ and $F$.
\end{proof}

\begin{lemma}
    Let $f\colon A \to B$ be a Boolean homomorphism and let $F$ be an ultrafilter of $B$. Then $f^{-1}[F]$ is an ultrafilter of $A$.
\end{lemma}
\begin{proof}
    Set $E = f^{-1}[F]$. Firstly, we will show $E$ is a filter. Trivially, $1 \in E$ and $0 \not\in E$. For $x, y \in E$, there are $x', y' \in F$ such that $x' = f(x)$ and $y' = f(y)$. Hence $x'\wedge y' = f(x) \wedge f(y) = f(x\wedge y)$ and $x\wedge y \in E$. For the upwards closeness, take any $x \in E$ and $y \geq x$. $y \in E$ as $f(y) \geq f(x) \in F$.

    $E$ is also an ultrafilter. For $a, b\in A$ such that $a\vee b \in E$, $f(a)\vee f(b) = f(a\vee b) \in F$ and so $f(a)$ or $f(b)$ is in $F$ and therefore $a$ or $b$ is in $E$.
\end{proof}

\num\label{p:SpMorphisms} Let $f\colon A \to B$ be a Boolean homomorphism, denote $\Sp f\colon \Sp B \to \Sp A$ to be the map defined as
    $$\Sp f\colon F \mapsto f^{-1}[F].$$

\noindent From the previous Lemma, we see the definition is sound. We will show $\Sp f$ is also continuous. Take any $O_I \in \tau_B$, we have
\begin{align*}
    (\Sp f)^{-1}[O_I] &= \Set{ (\Sp f)^{-1}(F) | F \cap I \neq \emptyset } \\
                      &= \Set{ E | \exists F \subseteq B \text{ ultrafilter, } (\Sp f)(E)=F, \text{ and } F \cap I \neq \emptyset } \\
                      &= \Set{ E | f^{-1}(E)\cap I \neq \emptyset }
                       = \Set{ E | E\cap f[I] \neq \emptyset } \\
                      &= \Set{ E | E\cap \downset f[I] \neq \emptyset }
                       = O_{(\JO h)(I)}.
\end{align*}

\begin{theorem}
    $\Sp\colon \Bool \to \StoneSp$ is a functor.
\end{theorem}
\begin{proof}
    Follows immediately from~\ref{p:SpObjects} and~\ref{p:SpMorphisms}.
\end{proof}

\num Our situation is as follows

\begin{diagram}
    \StoneFrm \ar[bend left=15]{rr}{\BcO} & & \Bool{} \ar[bend left=15]{ll}{\JO} \ar{ldd}{\Sp} \\
    \\
    & \StoneSp \ar{uul}{\Omega}
\end{diagram}

\begin{proposition}
    The collection of morphisms $\pi_B\colon \JO B \to \Omega\Sp(B)$ defined $I \mapsto O_I$, for $B \in \Bool$, constitutes a natural equivalence $\JO \cong \Omega\circ\Sp$.\ACP
\end{proposition}
\begin{proof}
    From the definition of $\Sp B$, we can see $\pi_B$ is an onto frame homomorphism.

    Take any $I \neq J$, $I,J \in \JO B$. Without lost of generality take $e \in I \setminus J$. Now the filter $\upset e$ is disjoint with the ideal $J$ and by Boolean Ultrafilter Theorem there exists an ultrafilter $U \supseteq \upset e$ disjoint with $J$. Also $U \cap I$ is not empty, it contains $e$, and therefore $O_I \neq O_J$.

    To prove naturalness of $\pi_*$, the following diagram have to commute

    \begin{diagram}
        \JO A \ar{r}{\pi_A} \ar{d}[swap]{\JO f} & \Omega\Sp(A) \ar{d}{\Omega\Sp(f)}\\
        \JO B \ar{r}{\pi_B}                    & \Omega\Sp(B)
    \end{diagram}

    \noindent for any Boolean homomorphism $f\colon A \to B$. Which is true, we have
    $$ \Omega\Sp(f)(O_I) = (\Sp f)^{-1}[O_I] = O_{(\JO h)(I)},$$

    \noindent where both equalities follows from~\ref{p:SpMorphisms}.
\end{proof}

\begin{conclusion}
    $\BcO\circ\Omega\circ\Sp \cong \Id_\Bool$.\ACP
\end{conclusion}
\begin{proof}
    By the previous Proposition and the Stone correspondence for Stone frames and Boolean algebras (Theorem~\ref{p:stonefrmBoolIso}), we have $\BcO\circ\Omega\circ\Sp \cong \BcO\circ\JO \cong \Id_\Bool$.
\end{proof}

\begin{proposition}
    The collection of morphisms $\rho_X\colon X \to \Sp\BcO\Omega(X)$ defined $x \mapsto F_x = \Set{U\text{ clopen} | x\in U }$, for $X \in \StoneSp$, constitutes a natural equivalence $\Id_\StoneSp \cong \Sp\circ\BcO\circ\Omega$.
\end{proposition}
\begin{proof}
    First observe $U$ is clopen in $\Sp\BcO\Omega(X)$ iff $U = \downset a$ for some $a \in \BcO\Omega(X)$ iff $a$ is clopen in $\Omega(X)$ iff $a$ is clopen in $\tau_X$.

    We will show $\rho_X$ is a homeomorphism.
    \begin{itemize}
        \item $\rho_X$ is one--one: For two points $x_1 \neq x_2$ of $X$, from Hausdorff property there are $U_1, U_2$ such that $U_1\cap U_2 = \emptyset$ and $x_i\in U_i$. From zero--dimensionality of $X$ there are two clopen subsets $M_1 \subseteq U_1, M_2 \subseteq U_2$, and $x_i \in M_i$. Hence $F_{x_1} \neq F_{x_2}$.

        % TODO avoid using knowledge about compact spacese?
        \item $\rho_X$ is onto: Take any $F$ ultafilter. From the observation above we know, $F$ is also an ultrafilter on $X$. Since $X$/$\Sp\BcO\Omega(X)$ is compact, there exists a convergent point $x$ of $F$. And so $F \subseteq F_x$. But $F$ is an ultrafilter, maximal filter, and therefore $F = F_x$.

        \item $\rho_X$ and $\rho_X^{-1}$ are continuous:
        \begin{align*}
            \rho_X^{-1}[O_I] &= \Set{ \rho_X^{-1}(F_x) = x | F_x \cap I \neq \emptyset} 
                = \Set{ x | x \in M \in I } = \bigcup I \in \tau_X \\
            \rho_X[O] &= \Set{ F_x | x \in M \subseteq O, M \text{ is clopen}} \\
                      &= \Set{ F | F \cap (\downset O \cap \BcO\Omega(X)) \neq \emptyset} = O_{\downset O \cap \BcO\Omega(X)} \in \tau_{\Sp\BcO\Omega(X)}.
        \end{align*}
    \end{itemize}

    And finally $\rho_*$, is a natural equivalence. The following diagram commutes
    \begin{diagram}
        X \ar{r}{\rho_X} \ar{d}[swap]{f} & \Sp\BcO\Omega(X) \ar{d}{\Sp\BcO\Omega(f)}\\
        Y \ar{r}{\rho_Y}                 & \Sp\BcO\Omega(Y)
    \end{diagram}
    \noindent for any continuous map $f\colon X \to Y$. Indeed,
    \begin{align*}
        \Sp\BcO\Omega(f)(F_x)
            &= (\BcO\Omega(f))^{-1}[F_x] = \Omega(f)^{-1}[F_x] \\
            &= \Set{ \Omega(f)^{-1}(M) | M \text{ clopen}, x \in M } \\
            &= \Set{ N | x \in M, M \text{ clopen}, f[M] = N } \\
            &= \Set{ N | f(x) \in N } = F_{f(x)} = \rho_Y(f(x)).
    \end{align*}
\end{proof}

\num When we sum up everything, we obtain
\begin{theorem*}
    Functors $\Omega\circ\BcO$ and \Sp{} are mutually inverse. Specially categories \Bool{} and $\StoneSp^{\text{op}}$ are isomorphic.\ACP
\end{theorem*}

\num As in the previous section, we obtained an isomorphism of categories. The functors creating this adjunction are contravariant, thus we got instead of correspondence between categories, the duality of two categories. However we can make the correspondence from previous section to duality simply using the duality between the category of frames and the category of locales (by taking Galois adjoints of all frame homomorphisms).

\section{Notes on constructivity}
One can check, the whole correspondence between Stone frames and Boolean algebras is given constructively. Therefore, the necessity of choice principle needed in classical Stone duality between Stone spaces and Boolean algebras is only to show that spaces constructed from Boolean algebras have enough points. Similarly to compactification of completely regular frames, we obtained classical result from Set Topology which uses Axiom of Choice in Point--free topology strictly constructively (we also avoided the use excluded middle principle).

To be more precise, in the proof of the compactification we used the fact that compact regular frames are completely regular (\ref{??}) we constructed an interpolative relation $\crbelow$ using the Axiom of Countable Dependent Choice (CDC). We actually do not have to worry about that, by Banaschewski and Pultr [TODO], we can avoid using CDC at all by working with strongly regular frames instead of completely regular. The whole construction can then be made constructive, even in the sense of Topos Theory.

Strongly regular frames are frames, in which for every element $x$ the following holds
$$ x = \bigvee \Set{ y | y\;(\rbelow)_o\;x }. $$

Where $(\rbelow)_o$ is the largest interpolative relation contained in $\rbelow$. Such relation can be constructed as the union of all interpolative relations contained in $\rbelow$. Moreover, under CDC, the complete regularity is precisely the same as strong regularity.
