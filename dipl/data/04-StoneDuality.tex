\chapter{Stone duality}
\section{Stone correspondence for \StoneFrm}

\begin{definition}
    We say a frame is \DEF{Stone frame} if it is compact, regular and zero--dimensional.
\end{definition}
% TODO define category \StoneFrm
% TODO shifted text ("Definition")

\begin{definition}
    Let $B$ be a Boolean algebra. Define \DEFSYM{J}{$\J{}$}$B$ to be the set of all ideals of $B$.
    For a Boolean homomorphism $f\colon A \to B$, define $\J f\colon \J A \to \J B$ as
    $$(\J f)(I) = \downset f[I].$$
\end{definition}

\begin{lemma}\label{p:complIdeal}
    Let $B$ be a Boolean algebra and $I \in \J B$. Then $I$ is complemented iff $I = \downset b$ for some $b \in B$.
\end{lemma}
\begin{proof}
    Let $I$ be a complemented ideal. Since $I \vee I^c = 1_{\J B}$ there exists $a \in I$ and $b \in I^c$ such that $a \vee b = 1$. From $I \wedge I^c = 0_{\J B}$ we have $a \wedge b = 0$ and $I \wedge \downset b = 0_{\J B}$.
    From uniques of complements we get $I^c = \downset b$, indeed $I \vee \downset b = 1_{\J B}$ and $I \wedge \downset b = 0_{\J B}$. Using the same argument we get $I = \downset a$.

    Converse implication is trivial since $\downset a \vee \downset a^c = 1_{\J B}$ and $\downset a \wedge \downset a^c = 0_{\J B}$.
\end{proof}

\begin{proposition}\label{p:JisFunctor}
    $\J\colon \Bool \to \StoneFrm$ is a functor.
\end{proposition}
\begin{proof}
    $\J B$ is trivially a compact frame. Lemma \ref{p:complIdeal} implies that $\downset a$ is complemented and $\downset a \rbelow \downset a$ for all $a \in B$. Thus for any ideal $I \in \J B$ we obtain
    $$ I = \bigvee \Set{ \downset a | a \in I } \subseteq \bigvee \Set{ J | J \rbelow I } \subseteq I,$$

\noindent hence $\J B$ is regular and zero--dimensional. The rest is easy.
\end{proof}

\begin{definition}
    Let $L$ be a Stone frame. Define $\DEFSYM{Bc}{$\Bc{}$} L$ to be the set of all complemented elements of $L$ and for a frame homomorphism $f\colon L \to M$ define
    $$\Bc f = \restr{f}{\Bc L}\colon \Bc L \to \Bc M.$$
\end{definition}
% TODO somehow mention that $f$ is between stone spaces

From the fact that a homomorphic image of a complemented element is a complemented element and since join or meet of two complemented elements is again complemented (to see that, let $a$ and $b$ be two complemented elements, then
    \begin{align*}
        (a\wedge b) \vee (a^c \vee b^c)
            &= (a\vee (a^c\vee b^c))\wedge (b\vee (a^c\vee b^c))
            = 1 \wedge 1 = 1, \text{ and} \\
        (a\wedge b) \wedge (a^c \vee b^c)
            &= ((a\wedge b)\wedge a^c)\vee ((a\wedge b)\wedge b^c)
            = 0 \vee 0 = 0;
    \end{align*}
and similarly $a^c\wedge b^c$ is complement of $a\vee b$.), one can see that $\Bc f$ is well--defined Boolean homomorphism.

% TODO some comment's before this statement
\begin{observation}
    $\Bc\colon \StoneFrm \to \Bool$ is a functor.
\end{observation}

\num\label{p:BoolEquivalence} For $B \in \Bool$ define \DEFSYM[istar]{}{$i_B$}$\colon B \to \Bc\J(B)$ as follows
    $$i_B\colon b \mapsto \downset b.$$
    The definition is sound by Lemma \ref{p:complIdeal} and $i_B$ is a Boolean homomorphism, indeed $\downset a \vee \downset b = \downset (a \vee b)$, $\downset a \wedge \downset b = \downset (a \wedge b)$, $\downset 1 = B$ respectively $\downset 0 = \{0\}$ is top respectively bottom of $\Bc\J(B)$, and complements are preserved thanks to distributivity of $\Bc\J(B)$.

From Lemma \ref{p:complIdeal} we also see that $i_B$ is an isomorphism and therefore the following diagram commutes

\begin{diagram}
    A \ar{r}{i_A} \ar{d}[swap]{f} & \Bc\J(A) \ar{d}{\Bc\J(f)}\\
    B \ar{r}{i_B}                 & \Bc\J(B)
\end{diagram}

\noindent for any Boolean homomorphism $f\colon A \to B$. From previous we have

\begin{proposition*}
    The collection $i_*$ of Boolean homomorphisms forms a natural equivalence between $\Bc\J$ and the identity functor on \Bool.
\end{proposition*}

\num\label{p:StoneFrmEquivalence} Similarly for $L \in \StoneFrm$ we have a mapping \DEFSYM[vstar]{}{$v_L$}$\colon \J\Bc(L) \to L$ defined as
    $$v_L\colon I \mapsto \bigvee I,$$
    and a mapping in the opposite direction $\iota\colon L \to \J\Bc(L)$:
    $$\iota\colon e \mapsto \downset e \cap \Bc L.$$

We can see that both $v_L$ and $\iota$ are monotone maps, $v_L \iota = \id_L$ (by zero--dimensionality and regularity of $L$) and $\id_{\J\Bc(L)} \subseteq \iota v_L$, therefore $v_L$ is the left Galois adjoint to $\iota$, hence $v_L$ preserves all suprema.
Since $\bigvee I_1 \wedge \bigvee I_2 = \bigvee \Set{a_1 \wedge a_2 | a_i \in I_i} \leq \bigvee \Set{ a | a \in I_1 \cap I_2 } = \bigvee (I_1 \cap I_2) \leq \bigvee I_1 \cap \bigvee I_2$, $v_L$ also preserves finite infima which makes it a frame homomorphism.

Finally, $\id_{\J\Bc(L)} = \iota v_L$: take any $x \in \iota v_L(I)$. From the definitions we immediately see that $x \leq \bigvee I$ and $x$ is complemented in $L$. By the fact that
    $$1 = x \vee x^c \leq \bigvee I \vee x^c$$
    and by compactness of $L$ there is a finite $F \subseteq I$ such that $\bigvee F \vee x^c = 1$. Since $x = 1 \wedge x = (\bigvee F \vee x^c) \wedge x = (\bigvee F \wedge x) \vee (x^c \wedge x) = \bigvee F \wedge x$ we get that $x \leq \bigvee F$ and therefore $x \in I$.

Previous observations give us the fact that $v_L$ is an isomorphism of $L$ and $\J\Bc(L)$ and also that for any homomorphism of Stone frames $f\colon L \to M$ the following diagram commutes

\begin{diagram}
    \J\Bc(L) \ar{d}[swap]{\J\Bc(f)} \ar{r}{v_L} & L \ar{d}{f} \\
    \J\Bc(M) \ar{r}{v_M}    & M
\end{diagram}

\noindent Again, as conclusion of previous paragraphs we obtain

\begin{proposition*}
    The collection $v_*$ of frame homomorphisms is a natural equivalence between $\J\Bc$ and the identity functor on \StoneFrm.
\end{proposition*}

\num Using previous facts we get the main result of this section.

\begin{theorem*}
    Functor \Bc{} is the right adjoint to the functor \J{} with $i_*$ as unit of adjunction and $v_*$ as counit. Moreover, \Bc{} and \J{} constitute an isomorphism of categories \StoneFrm{} and \Bool.
\end{theorem*}

As we will see in next chapter, \J{} corresponds precisely to compactification of Boolean algebras the same way as the space of ultrafilters is part of Stone duality and compactification for spaces. (TODO better wording)

TODO (note about AC:) As one can check, the whole correspondence is given in constructive setting. No use of Axiom of Choice unlike in classical case, \dots

\section{Stone duality for \StoneSp}

\num Similarly to Frame case:
\begin{definition*}
    We say a topological space is \DEF{Stone space} if it is compact, Hausdorff and zero--dimensional.
\end{definition*}

\num Let $B$ be a Boolean algebra, define
    $$
    X_B  = \Set{F \subseteq B | F \text{ is an ultrafilter on } B}
    \quad\text{ and }\quad
    \tau_B = \Set{ O_I | I \in \J B},
    $$
\noindent where \DEFSYM{OI}{$O_I$} is the set $\Set{ F \text{ ultrafilter} | F \cap I \neq \emptyset }$. Now $(X_B, \tau_B)$ is a topological space:
\begin{itemize}
    \item $O_I \cap O_J = O_{I\cap J}$: $F \in O_I\cap O_J$ iff $F\cap I \neq \emptyset$ and $F\cap J \neq \emptyset$ iff $F \cap I \cap J \neq \emptyset$ iff $F\in O_{I\cap J}$.
    \item $\bigcup_i O_{J_i} = O_{\bigvee_i J_i}$: $\supseteq$ holds trivially since $\bigcup J_i \subseteq \bigvee J_i$. On the other hand, for an ultrafilter $F$ such that $F\cap \bigvee_i J_i \neq \emptyset$, there is $e = \bigvee E$ where $E \subseteq \bigcup J_i$ finite and $e \in F$. Since $F$ is prime, there is $e' \in E$ such that $e' \in F$ and therefore $F\cap J_j \neq \emptyset$ for some $J_j \ni e'$.
    \item $\tau_B$ contains $O_{\downset 0} = \emptyset$ and $O_{\downset 1} = X_B$.
\end{itemize}

\noindent We will denote \DEFSYM{S B}{$\Sp B$} to be $(X_B, \tau_B)$.

\begin{lemma}\label{p:SpObjects}
    Let $B$ be a Boolean algebra, $\Sp B$ is a Stone space.
\end{lemma}
\begin{proof}
    Compactness and zero--dimensionality follows directly from compactness and zero--dimensionality of $\J B$.

    To show $X_B$ is also Hausdorff, take any ultrafilters $E \neq F$. Without lost of generality there is some $e \in E \setminus F$. We have $e\vee e^c = 1$, since $F$ is an ultrafilter we have $e^c \in F$. We also have $e\wedge e^c = 0$ and so $O_{\downset e}$ and $O_{\downset e^c}$ separates $E$ and $F$.
\end{proof}

\begin{lemma}
    Let $f\colon A \to B$ be a Boolean homomorphism and let $F$ be an ultrafilter on $B$. Then $f^{-1}[F]$ is an ultrafilter on $A$.
\end{lemma}
\begin{proof}
    Set $E = f^{-1}[F]$. Firstly, we will show $E$ is a filter. Trivially $1 \in E$ and $0 \not\in E$. For $x, y \in E$, there are $x', y' \in F$ such that $x' = f(x)$ and $y' = f(y)$. Hence $x'\wedge y' = f(x) \wedge f(y) = f(x\wedge y)$ and $x\wedge y \in E$. For the upwards closeness, take any $x \in E$ and $y \geq x$. $y \in E$ as $f(y) \geq f(x) \in F$.

    $E$ is also an ultrafilter. For $a, b\in A$ such that $a\vee b \in E$, $f(a)\vee f(b) = f(a\vee b) \in F$ and so $f(a)$ or $f(b)$ is in $F$ and therefor $a$ or $b$ is in $E$.
\end{proof}

\num\label{p:SpMorphisms} Let $f\colon A \to B$ be a Boolean homomorphism, denote $\Sp f\colon \Sp B \to \Sp A$ to be the map defined as
    $$\Sp f\colon F \mapsto f^{-1}[F].$$

\noindent From previous Lemma, we see the definition is sound. We will show $\Sp f$ is also continuous. Take any $O_I \in \tau_B$, we have
\begin{align*}
    (\Sp f)^{-1}[O_I] &= \Set{ (\Sp f)^{-1}(F) | F \cap I \neq \emptyset } \\
                      &= \Set{ E | \exists F \subseteq B \text{ ultrafilter, } (\Sp f)(E)=F, \text{ and } F \cap I \neq \emptyset } \\
                      &= \Set{ E | f^{-1}(E)\cap I \neq \emptyset }
                       = \Set{ E | E\cap f[I] \neq \emptyset } \\
                      &= \Set{ E | E\cap \downset f[I] \neq \emptyset }
                       = O_{\J(h)(I)}.
\end{align*}

\begin{theorem}
    $\Sp\colon \Bool \to \StoneSp$ is a functor.
\end{theorem}
\begin{proof}
    Follows immediately from~\ref{p:SpObjects} and~\ref{p:SpMorphisms}.
\end{proof}

\num Our situation is as follows

\begin{diagram}
    \StoneFrm \ar[bend left=15]{rr}{\Bc} & & \Bool{} \ar[bend left=15]{ll}{\J} \ar{ldd}{\Sp} \\
    \\
    & \StoneSp \ar{uul}{\Omega}
\end{diagram}

\begin{proposition}
    The collection of morphisms $\pi_B\colon \J B \to \Omega\Sp(B)$ defined $I \mapsto O_I$, for $B \in \Bool$, constitutes a natural equivalence $\J \cong \Omega\circ\Sp$.\ACP
\end{proposition}
\begin{proof}
    From the definition of $\Sp B$ we can see that $\pi_A$ is onto frame homomorphism.

    Take any $I \neq J$, $I,J \in \J B$. Without lost of generality take $e \in I \setminus J$. Now the filter $\upset e$ is disjoint with the ideal $J$ and by Boolean Ultrafilter Theorem there exists an ultrafilter $U \supseteq \upset e$ disjoint with $J$. Also $U \cap I$ is not empty, it contains $e$, and therefore $O_I \neq O_J$.

    To prove naturalness of $\pi_*$ the following diagram have to commute

    \begin{diagram}
        \J A \ar{r}{\pi_A} \ar{d}[swap]{\J f} & \Omega\Sp(A) \ar{d}{\Omega\Sp(f)}\\
        \J B \ar{r}{\pi_B}                    & \Omega\Sp(B)
    \end{diagram}

    \noindent for any Boolean homomorphism $f\colon A \to B$. We have
    $$ \Omega\Sp(f)(O_I) = (\Sp f)^{-1}[O_I] = O_{\J(h)(I)},$$

    \noindent where both equalities follows from~\ref{p:SpMorphisms}.
\end{proof}

\begin{conclusion}
    $\Bc\circ\Omega\circ\Sp \cong \Id_\Bool$.\ACP
\end{conclusion}
\begin{proof}
    By previous Proposition and the Stone correspondence for Stone frames and Boolean algebras we have $\Bc\circ\Omega\circ\Sp \cong \Bc\circ\J \cong \Id_\Bool$.
\end{proof}

\begin{proposition}
    The collection of morphisms $\rho_X\colon X \to \Sp\Bc\Omega(X)$ defined $x \mapsto F_x = \Set{U\text{ clopen} | x\in U }$, for $X \in \StoneSp$, constitutes a natural equivalence $\Id_\StoneSp \cong \Sp\circ\Bc\circ\Omega$.
\end{proposition}
\begin{proof}
    First observe $U$ is clopen in $\Sp\Bc\Omega(X)$ iff $U = \downset a$ for some $a \in \Bc\Omega(X)$ iff $a$ is clopen in $\Omega(X)$ iff $a$ is clopen in $\tau_X$.

    We will show $\rho_X$ is homeomorphism.
    \begin{itemize}
        \item $\rho_X$ is one--one: For two points $x_1 \neq x_2$ of $X$, from Hausdorff property there are $U_1, U_2$ such that $U_1\cap U_2 = \emptyset$ and $x_i\in U_i$. From zero--dimensionality of $X$ there are two clopen subsets $M_1 \subseteq U_1, M_2 \subseteq U_2$, and $x_i \in M_i$. Hence $F_{x_1} \neq F_{x_2}$.

        % TODO avoid using knowledge about compact spacese?
        \item $\rho_X$ is onto: Take any $F$ ultafilter. From the observation we know, $F$ is also ultrafilter on $X$. Since $\Sp\Bc\Omega(X)$ is compact, there exists convergent point $x$ of $F$. And so $F \subseteq F_x$. But $F$ is an ultrafilter, maximal filter, and therefore $F = F_x$.

        \item $\rho_X$ and $\rho_X^{-1}$ are continuous:
        \begin{align*}
            \rho_X^{-1}[O_I] &= \Set{ \rho_X^{-1}(F_x) = x | F_x \cap I \neq \emptyset} 
                = \Set{ x | x \in M \in I } = \bigcup I \in \tau_X \\
            \rho_X[O] &= \Set{ F_x | x \in M \subseteq O, M \text{ is clopen}} \\
                      &= \Set{ F | F \cap (\downset O \cap \Bc\Omega(X)) \neq \emptyset} = O_{\downset O \cap \Bc\Omega(X)} \in \tau_{\Sp\Bc\Omega(X)}.
        \end{align*}
    \end{itemize}

    And finally $\rho_*$ is natural equivalence. The following diagram commutes
    \begin{diagram}
        X \ar{r}{\rho_X} \ar{d}[swap]{f} & \Sp\Bc\Omega(X) \ar{d}{\Sp\Bc\Omega(f)}\\
        Y \ar{r}{\rho_Y}                 & \Sp\Bc\Omega(Y)
    \end{diagram}
    \noindent for any continuous map $f\colon X \to Y$. Indeed
    \begin{align*}
        \Sp\Bc\Omega(f)(F_x)
            &= \Bc\Omega(f)^{-1}[F_x] = \Omega(f)^{-1}[F_x] \\
            &= \Set{ \Omega(f)^{-1}(M) | M \text{ clopen}, x \in M } \\
            &= \Set{ N | x \in M, M \text{ clopen}, f^{-1}[N] = M } \\
            &= \Set{ N | x \in f^{-1}[N] } = \Set{ N | f(x) \in N } = F_{f(x)} = \rho_Y(f(x))
    \end{align*}
\end{proof}

\begin{theorem}
    Functors $\Omega\circ\Bc$ and \Sp{} are mutually inverse. Specially categories \Bool{} and $\StoneSp^{\text{op}}$ are isomorphic.\ACP
\end{theorem}

\section{Role of Regular filters}
