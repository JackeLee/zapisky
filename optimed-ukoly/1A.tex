\documentclass[11pt,a4paper]{article}
\usepackage{../basic}
\usepackage{../math}

% Nastaveni hlavicek a paticek
\usepackage{fancyhdr}
\pagestyle{fancy} % nebo {empty} pro uplne vypnuti
\lhead{Tomáš Jakl}
\rhead{\today}
\rfoot{\thepage}
\cfoot{}

\renewcommand{\headrulewidth}{1.0pt} % podtrzeni titulku

% Pro zmenu pro konkretni stranku staci nastavit:
% \thispagestyle{plain/empty/...}

% Matematika
\renewcommand{\vec}[1]{\mathbf #1}


\newtheorem*{zadani}{Zadání}
\newtheorem*{reseni}{Řešení}

\begin{document}
\begin{zadani}[Domácí úloha A]
Zapište následující úlohy LP v rovnicovém a nerovnicovém tvaru.
\begin{enumerate}
\item
	\begin{tabbing}
		maximalizovat \= $2x_1 - 3x_2$ \\
		za podmínek \> $4x_1 - 5x_2 \leq 6$ \\
		\> $7x_1 + 8x_2 = 8 $ \\
		\> $x_1 \geq 0$
	\end{tabbing}

\item
	\begin{tabbing}
		maximalizovat \= $c^Tx$ \\
		za podmínek \> $A'x \geq b'$ \\
		\> $A''x = b''$ \\
		\> $x \in \R^n, x \geq \vec 0$ \\
		\> kde $A' \in \R^{m' \times n}, A'' \in \R^{m'' \times n},$\\
		\> $b' \in \R^{m'}, b'' \in \R^{m''}, c \in \R^{n}$
	\end{tabbing}
\end{enumerate}
\end{zadani}

\begin{reseni}
Postupně:
\begin{enumerate}
\item
	Převodem do nerovnicového tvaru dostáváme úlohu lineárního programování ve
	standartním maticovém tvaru, tedy $max_{x \in \R^2} \{c^Tx~|~Ax \leq
	b\}$, kde:
	\begin{align*}
		A = \begin{pmatrix} 4 & -5 \\ 7 & 8 \\ -7 & -8 \\ -1 &  0 \end{pmatrix}\!,~
		b = \begin{pmatrix} 6 \\ 8 \\ -8 \\ 0 \end{pmatrix}\!,~
		c = \begin{pmatrix} 2 \\ -3 \end{pmatrix}
	\end{align*}

	Rovnost jsme nahradili dvěmi nerovnostmi (platí: $a = b \Leftrightarrow a
	\leq b~\& -a \leq -b$) a pro požadavek $x_1 \geq 0$ jsme přidali řádek
	podmiňující $-x_1 \leq 0$.

	Pro nerovnicový tvar a tedy řešení úlohy $-min_{x \in \R^4}
	\{c^Tx~|~Ax=b, x \geq \vec 0\}$ stačí upravit zadání do tvaru:
	\begin{align*}
		A = \begin{pmatrix} 4 & -5 & 5 & 1 \\ 7 & 8 & -8 & 0 \end{pmatrix}\!,~
		b = \begin{pmatrix} 6 \\ 8 \end{pmatrix}\!,~
		c = \begin{pmatrix} -2 \\ 3 \\ -3 \\ 0 \end{pmatrix}
	\end{align*}

	Možnost záporného $x_2$ stačí vyřešit přidáním dalšího sloupce, který bude
	kopií druhého přenásobeného $-1$-čkou (tím se docílí, že pro $A^2x_2 \leq b_2$
	bez požadavku na $x_2$ dostáváme ekvivalentní vztah $A^2x'_2 - A^2x''_2
	\leq b_2$ vyhovovující požadavku $\mtrx{x'_2 \\ x''_2} \geq \vec 0$).
	
	Protože první řádek matice $A$ má reprezentovat nerovnost, přidali jsme
	,,vatu'' v podobě přičtení nové neomezené proměnné pro první rovnici.
	Požadavek na nezápornost této nové proměnné nic nemění. Navíc pro nějaké
	maximální řešení našeho systému podmínek, existuje ,,vata'', tak, že z
	první nerovnosti udělá rovnost, takže tato úprava je korektní.
\item
	Použitím stejných úvah jako v předchozím případě je úprava zadání do
	nerovnicového s řešením $max_{x \in \R^n} \{ c^Tx~| Ax \leq b\}$, kde:
	\begin{align*}
		A = \mtrx{ -A' \\ A'' \\ -A'' \\ -\vec1}\!,~
		b = \mtrx{ -b \\ b'' \\ -b'' \\ \vec 0}
	\end{align*}

	Převod na rovnicový tvar, jehož řešení má tvar $-min \{ -c'^Tx~| Ax = b, x \geq \vec 0 \}$ je následující:
	\begin{align*}
		A = \mtrx{A'' & \vec 1 \\ A'' & \vec 0 }\!,~
		b = \mtrx{b' \\ b''}\!,~
		c' = \mtrx{c \\ \vec 0}
	\end{align*}

\end{enumerate}
\end{reseni}
\end{document}
