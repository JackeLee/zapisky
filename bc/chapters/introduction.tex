% \chapter*{Introduction}
% \addcontentsline{toc}{chapter}{Introduction}

\chapter{Introduction}
After Gasparov was defeated, new challenges has come. Defeat men in Go. Until
now, there were no significant success on standart 13$\times$13 board. But many
useful new way of playing games was invented. In ??? Monte Carlo methods were
successfully used to defeat all other computer players of Go. After that day
all successful programs were using Monte Carlo methods.
% TODO so many ???s

Tomáš Kozelek has shown in his work, that building Arimaa playing program using
MCTS is possible. In this work we will focus on comparing capabilities and
perspectives to the future given by AlphaBeta search and MCTS to game of
Arimaa.

After first computers were created it was always in human target to fight
with/compete human mind in every occasion. First significant result was shown
in 1997. IBM constructed/built computer with just one purpose, to defeat
the best human player in the game of chess. It took few years of development
and ??? milions of dolars to build computer. Fist time they wasn't successful
[???], but after some time and more effort they defeated Gasparov with computer
named Deep blue.
% TODO preformulovat a upresnit/rozsirit
[1] and [???]

We will introduce the game of Arimaa and describe two algorithms \ac{MCTS} and AlphaBeta search ... TODO


\section{Terminology}
\begin{description}
\item[Game bot] is game playing program.
\item[Game tree] for arbitrary game is tree with starting positions as root and
   with children of nodes as all possible consequent positions.
\item[Evaluation function] is function which estimates value of given position of the game. It can be used for example to compare which of two given positions is better.
\item[Minimax tree] for two player game is game tree limited to some depth with
added values in all nodes. Values are defined recursively. In leaf of the tree
is value calculated by evaluation function. In nonleaf node is the value
defined as the best value from nodes children from nodes active player point
of view.
\item[Principal variation] is best sequence of moves for actual player leading
from root of the tree to leaf in minimax if we presume both players play their
best.
\item[Transposition] is 
\item[Branching factor] 
\end{description}

\section{The Game of Arimaa}
The game of Arimaa is pretty new game. It is carefully designed to be hard to
play for computers, but easy to play for humans. Creator of the game Omar
Syed [1]

The game was carefully designed in order not to be possible to use methods
well known from Chess as game-ending tables or opening tables. (TODO
REWRITE:) Also to be significantly harder to precompute moves for huge
number moves to future and to efficiently decide which of two given position
is better.

(TODO REWRITE:) Omar says that gasparov was not oversmarted but overcomputed ...
In Arimaa it is significantly harder to precompute moves ??because?? and to
efficiently decide which of two given position is better ??because??.

Why we cannot use standart methods widely used in chess? (Section 2 in Kozeleks thesis)

\section{Rules of the game~\cite{arimaa.com}}
Arimaa is two-player zero-sum game with perfect information. It is designed to
be possible to play it using the board and pieces from chess set. The starting
player has Gold color, second is Silver. In the zeroth turn Gold and then
Silver player each place all their pieces into first two (for Gold) or last two
(for Silver) lines of the board any way they consider appropriate. Piece set
consist of eight Rabbits, two Cats, two Dogs, two Horses, Camel and Elephant in
order from weakest to strongest.

%% TODO example picture of the starting position

Players are taking turns starting with Gold. In each turn player makes move,
which consist of one to four steps. If less than four steps were made it is
said that player passed. After move ends, position of the board must be
different from position before turn started.

Figure is considered frozen if on one of its neighbouring squares is opponents
stronger piece and on no figure sharing color with it. Adjacent or neighbouring
squares are those squares lying in one position to left, right, front or
backwards.

%% TODO example picture of the frozen piece and possibility to push and pull

To make step player chooses one of its non frozen figures and move it to one of
the free adjacent squares, with one exception -- rabbits cannot step backwards.
Instead of making simple step player can provide pull or push. Pulling is
almost identical to simple step, but after step is done player choose one of
the weaker piece which stayed on neighbouring of the square from which figure
moved, and move it to position from which our piece stepped. Pushing is a kind
of opposite to pulling. At first neighbouring weaker piece is moved to its free
adjacent square and then our piece is placed to weaker pieces former place.
Push and pull counts as 2 steps and can not be combined together.

On the board are four special squares called traps in positions \texttt{c3},
\texttt{c6}, \texttt{f3} and \texttt{f6}. After each step ends if any piece is
in trap and has no adjacent piece sharing color with him, it is trapped and
therefore removed from the board.

At end of the move is checked if game ended and it happens if one of the
following conditions is fulfilled:
\begin{enumerate}
\item one of the players rabbit has reached opponents front line (goalline), we
call it goal,
\item one of the players has no rabbit left, we call it elimination,
\item next active player cannot move, we call it immobilisation,
\item or the same position is repeated in third time in the row.
\end{enumerate}

Player wins if he scores goal or opponent is eliminated, is immobilised or
repeated position for third time.

%TODO figures with examples of pushing/pulling, immobilisations, trapping, ...

\section{Comparison to Go and Chess}
Because Arimaa is played with full chess set it is very natural to ask about
similarities with the game of Chess. As in Chess is in Arimaa also so easy to
ruin good position with just one bad move. For example stepping out of trap an
therefore let another piece to be trapped or unfreezing rabbit near goalline.
Unlike in Chess starting position is not predefined and there are about
$4.207\times10^{15}$ different possible openings which makes it hard for arimaa
bot programmer to use any kind of opening tables~\cite{COX}.

In Arimaa it is also very hard to build good static evaluation function. Even
human players often do not know for example if in certain position it is better
to sacrifice Camel for trapping opponents Horse and Cat and often it depends on
many other factors. Building good evaluation function is very hard in game of
Go also, because it requires a lot of local searching to find territories and
determine their potential owners. On the other hand destroying good position by
making wrong step is in Go a lot harder.

In game of Go if we omit filling eye any random playing sequence leads to end,
which is not so easily achievable in Arimaa and Chess. Christ Cox showed in his
work that in Arimaa branching factor of average position is around 20,000. With
comparison in Chess it is only 35 and in Go 200~\cite{COX}.

% TODO? position in arimaa vs tactic in chess ~\cite{COX}

\section{Challenge}
Omar Syed decided to left a prize 10,000 USD for programmer or group of
programmers who develop Arimaa playing program which win Arimaa Computer
Championship and then defeat three chosen top human players before the year
2020. Omar Syed believes that this motivation will help further improvements in
area of AI game programming~\cite{syed}.

So far computers were not even close to defeat one of the chosen human
players~\cite{arimaa.com}. ... (TODO add facts)

\section{Object of research}
The main part of this work is to develop well documented Arimaa playing program
and try to answer the following questions:

\begin{enumerate}
\item Is MCTS competitive alpha beta search at all?
\item Is MCTS more promising engine than AlphaBeta search in the future with
      increasing number of cpus?
\item How important is evaluation function in MCTS compared to AlphaBeta?
\end{enumerate}
% TODO: which area should be compared? who is more perspective in future?

