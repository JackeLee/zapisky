% \chapter*{Introduction}
% \addcontentsline{toc}{chapter}{Introduction}

\chapter{Introduction}
After Gasparov was defeated, new challenges has come. Defeat men in Go. Until
now, there were no significant success on standart 13$\times$13 board. But many
useful new way of playing games was invented. In ??? Monte Carlo methods were
successfully used to defeat all other computer players of Go. After that day
all successful programs were using Monte Carlo methods.
% TODO so many ???s

Tomáš Kozelek has shown in his work, that building Arimaa playing program using
MCTS is possible. In this work we will focus on comparing capabilities and
perspectives to the future given by AlphaBeta search and MCTS to game of
Arimaa.

After first computers were created it was always in human target to fight
with/compete human mind in every occasion. First significant result was shown
in 1997. IBM constructed/built computer with just one purpose, to defeat
the best human player in the game of chess. It took few years of development
and ??? milions of dolars to build computer. Fist time they wasn't successful
[???], but after some time and more effort they defeated Gasparov with computer
named Deep blue.
% TODO preformulovat a upresnit/rozsirit
[1] and [???]

We will introduce the game of Arimaa and describe two algorithms \ac{MCTS} and AlphaBeta search ... TODO


\section{Terminology}
% TODO organise in stg like table
\begin{description}
\item[game bot] is game playing program
\item[board/game position]
\item[game tree] for arbitrary game is tree with starting positions as root and
   with children of nodes as all possible consequent positions.
\item[evaluation function] is function which estimate value of given position of the game. It can be used for example to compare which of two given positions is better.
\item[minimax tree] for two player game is game tree limited to some depth with
added values in all nodes. Values are defined recursively. In leaf of the tree
is value defined by evaluation function. In node is the best value from nodes
children from nodes player on turn side point of view.
\end{description}

\section{Object of research}
The main part of this work is to develop well documented Arimaa playing program
and try to answer the following questions:

\begin{enumerate}
\item Is MCTS competitive alpha beta search at all?
\item Is MCTS more promising engine than AlphaBeta search in the future with
      increasing number of cpus?
\item How important is eval function in MCTS compared to AlphaBeta?
\end{enumerate}
% TODO: which area should be compared? who is more perspective in future?

Disclaimer: We are not supposed to develop strong arimaa playing program. (We don't try to develop ...)

\section{The Game of Arimaa}
The game of Arimaa is pretty new game. It is carefully designed to be hard to
play for computers, but easy to play for humans. Creator of the game Omar
Syed [1]

The game was carefully designed in order not to be possible to use methods
well known from Chess as game-ending tables or opening tables. (TODO
REWRITE:) Also to be significantly harder to precompute moves for huge
number moves to future and to efficiently decide which of two given position
is better.

(TODO REWRITE:) Omar says that gasparov was not oversmarted but overcomputed ...
In Arimaa it is significantly harder to precompute moves ??because?? and to
efficiently decide which of two given position is better ??because??.

Why we cannot use standart methods widely used in chess? (Section 2 in Kozeleks thesis)

\section{Rules of the game}
Arimaa is two-player zero-sum game with perfect information. It is designed to
be possible to play it using the board and pieces from chess set. The starting
player has Gold color, seconds is Silver. In the zeroth turn Gold and then
Silver player each place all their pieces to into first two (for Gold) or last
two (for Silver) lines of the board. Piece set consist of eight Rabbits, two
Cats, two Dogs, two Horses, Camel and Elephant in order from weakest to
strongest.

%% TODO example picture of the starting position

Players are taking turns starting with Gold. In each turn player makes move,
which consist of one to four steps. If less than four steps were made it is
said that player passed. After move ends, position of the board must be
different from position before turn started.

Figure is considered frozen if on one of its neighbouring squares is opponents
stronger piece and on no figure sharing color with it. Adjacent or neighbouring
squares are those squares lying in one position to left, right, front or
backwards.

%% TODO example picture of the frozen piece and possibility to push and pull

To make step player chooses one of its non frozen figures and move it to one of
the free adjacent squares, with one condition -- rabbits cannot step backwards.
Instead of making simple step player can provide pull or push. Pulling is
almost identical to simple step, but after step is done player choose one of
the weaker piece which stayed on neighbouring of the square from which figure
moved, and move it to position from which our piece stepped. Pushing is a kind
of opposite to pulling. At first neighbouring weaker piece is moved to its free
adjacent square and then our piece is placed to weaker pieces former place.
Push and pull counts as 2 steps and can not be combined together.

On the board are four special squares called traps in positions \texttt{c3},
\texttt{c6}, \texttt{f3} and \texttt{f6}. After each step if any piece is in
trap and has no adjacent piece sharing color with him, it is trapped and
therefore removed from the board.

Game ends when one of the players rabbit reaches opponents front line, one of
the players has no rabbit, active player cannot move.

TODO Third time repetition rule.
TODO examples of pushing/pulling, trapping from figures


\section{Comparison to Go and Chess}

similarities with chess:
- less stable positions
- "infinite sequence of moves is more common"
- however there is no possibility to create opening and ending tables

similarities with go:
- huge branching factor
- it's hard to build evaluation function


\section{Challenge}
Omar Syed decided to left few thousands dollars to
