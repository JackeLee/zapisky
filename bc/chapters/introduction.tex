\chapter{Introduction}
After first computers were created there were always an intention to compete
human mind in many occasions. First significant result has occurred in 1997.
IBM's employees succeeded by constructing a computer named Deep blue built with
just one purpose, to defeat Garry Kasparov, the best human player in the game
of chess. The core of an algorithm used in Deep blue was AlphaBeta search.

Because Kasparov was defeated, new challenges has come. Defeat men in Go. Until
now, there were no significant success on standart 19$\times$19 board. But many
useful new ways of playing games were invented. In 2006 Monte Carlo methods
were used to defeat all other computer players of Go. After that day all
successful programs were using Monte Carlo methods~\cite{MoGo}.

In the world of chess programming the most successful algorithm for game tree
search is still considered AlphaBeta search, however in game of Go it is Monte
Carlo Tree Search. Arimaa has similarities with both Go and Chess, but there
has been no successful program using Monte Carlo Tree Search so far.

On the 1st February 2010 in Arimaa forum, an interesting question was asked,
how would two bots --  one using Monte Carlo Tree Search and the second using
AlphaBeta search -- compete if they both would use the same evaluation
function~\cite{arimaa.com}.

We will introduce rules of the game of Arimaa and describe two algorithms
\ac{MCTS} and the AlphaBeta search with various extensions.


\section{Terminology}
\begin{description}
\item[Game tree] for arbitrary game is tree with starting positions as root and
   with children of nodes as all possible consequent positions.
\item[Evaluation function] is function which estimates value of given position of the game. It can be used for example to compare which of two given positions is better.
\item[Minimax tree] for two player game is game tree limited to some depth with
added values in all nodes. Values are defined recursively. In a leaf of the tree
the value is calculated by evaluation function. In nonleaf node the value
is defined as the best value from nodes children from  point of view of nodes
active player.
\item[Principal variation] is best sequence of moves for actual player leading
from root of the tree to leaf in minimax tree if we presume both players play
their best.
\item[Transposition] is a position which can be reached by more than one
sequence of moves.
\item[Branching factor] of a node is number of children that node has. Usually
branching factor of a tree is an average branching factor of all its nodes.
\item[Game bot] is a program playing the game.
\end{description}

\section{The Game of Arimaa}
\begin{flushright}
\emph{"Even simple rules can lead to interesting games."\\
--- Omar Syed}
\end{flushright}

The game of Arimaa belongs to younger games. It is carefully designed to be
hard to play for computers and easy to play for humans. Arimaa's creator, Omar
Syed says that Kasparov was not oversmarted but overcomputed by Deep blue.
This motivated him to create game with such properties.

In order to make the game easier for humans and harder for computers he brought
an interesting idea: "The pieces should have very simple movements, but the
players should be able to move more than one piece in each
turn"~\cite{arimaa.com}.

Another weakening of computers is achieved by having rules that embarrasses
methods well known from Chess such as game-ending tables or opening tables.
It should be also significantly harder to efficiently decide which of two given
position is better~\cite{arimaa.com}.

In spite of the fact that in Arimaa we have six different kinds of pieces,
almost all of them moves the same way which makes the rules of the game much
easier for human in comparison to chess, furthermore the rules of Arimaa are so
simple that even small kids can play it.


\section{Rules of the game}
Arimaa is a two-player zero-sum game with perfect information. It is designed to
be possible to play it using the board and pieces from chess set. The starting
player has Gold color, second is Silver. In the first turn Gold and then
Silver player each place all their pieces into first two (for Gold) or last two
(for Silver) lines of the board any way they consider appropriate. Piece set
consist of eight Rabbits, two Cats, two Dogs, two Horses, Camel and Elephant in
order from weakest to strongest.

%% TODO example picture of the starting position

Players are taking turns starting with Gold. In each turn a player makes move,
which consist of one to four steps. Player may \emph{pass} and do not use
remaining steps. After move ends, the position of the board must be different
from the position before move started and a player is not allowed to play the
same position for third time.

A piece is touching another piece if it is staying on a neighbouring square of
that piece. We say a piece is lonely if it is not touching another piece
sharing color with it. A lonely piece is frozen if it is touching an opponents
stronger piece. A catchable piece by some stronger piece is an opponents piece
touching that stronger piece. Adjacent or neighbouring squares are those
squares lying in one position to left, right, front or backwards.

%% TODO example picture of the frozen piece and possibility to push and pull

To make step a player chooses one of its non frozen pieces and move it to one of
the free adjacent squares, with one exception -- rabbits cannot step backwards.
Instead of making a simple step player can provide pull or push.

In pulling the player choses his non frozen piece and a piece catchable by it.
Normal step by the player's piece is performed and then the catchable piece is
moved to the stepped's piece previous location. Pushing is a kind of opposite
to pulling. Player on turn chooses a catchable piece and his stronger piece
touching it. Then by the player's choice the opponent's catchable piece is
moved to its free adjacent square and the stronger non frozen piece touching it
is moved to the catchable piece's former location. Push and pull counts as 2
steps and can not be combined together.

On the board are four special squares called traps in positions \texttt{c3},
\texttt{c6}, \texttt{f3} and \texttt{f6}. A lonely piece standing in a trap
square after a step is immediately removed from the board.

Following end-game conditions are checked at the end of each turn:
\begin{enumerate}
\item Scoring \emph{goal:} one of the rabbits has reached goalline (the
furthest row from the starting rows). The player whose rabbits reached the
goalline wins. In the artificial case when rabbits of both players reached the
goalline the player making the move wins.
\item \emph{elimination:} one of the players has no rabbit left. The player
having a rabbit wins. In case where no rabbit remains, the player making the
move wins.
\end{enumerate}

At the start of a turn an existence of a legal step is checked. If such step
does not exist, the player to move loses due to \emph{immobilisation}.

Making illegal move leads to losing the game. This cannot happen to human
player using modern computer interface, but it could happen to buggy computer
programs or in the game played on the real board. The games played using
computer interface also can end by \emph{timeout} when a player to make step do
not make it at time.

There are special rules to define a game result even for games taking too long
time, but these rules will not be important for us.

%TODO figures with examples of pushing/pulling, immobilisations, trapping, ...

More informations about rules can be found in~\cite{arimaa.com}.

\section{Comparison to Go and Chess}
Because Arimaa is played with full chess set it is very natural to ask about
similarities with the game of Chess. In Chess and in Arimaa it is so easy to
ruin good position with just one bad move. For example stepping out of trap an
therefore let another piece to be trapped or unfreezing rabbit near goalline.
Unlike in Chess starting position is not predefined and there are about
$4.207\times10^{15}$ different possible openings which makes it hard for Arimaa
bot programmer to use any kind of opening tables~\cite{COX}.

In Arimaa it is also very hard to build a good static evaluation
function~\cite{WorldChampion}. Event two current best players do not agree in
evaluation of camel for cat with horse exchange, which wildly depends on the
position of other pieces. The simplest example of our lack of knowledge is
evaluation of initial rabbit for nonrabbit exchanges. Some of the top players
evaluate initial exchange of two rabbits for one horse almost equal while
others prefer a cat for two rabbits.

Building good evaluation function is very hard in game of Go also, because it
requires a lot of local searching to find territories and determine their
potential owners. On the other hand destroying good position by making wrong
step is in Go a lot harder.

In the game of Go if we omit filling eyes, then any random playing sequence
leads to end, which is not so easily achievable in Arimaa and Chess. Christ Cox
showed in his work that in Arimaa branching factor of average position is
around 20,000. With comparison in Chess it is only 35 and in Go 200~\cite{COX}.


\section{Challenge}
Omar Syed decided to left a prize 10,000 USD for programmer or group of
programmers who develop Arimaa playing program which win Arimaa Computer
Championship and then defeats three chosen top human players before the year
2020. Omar Syed believes that this motivation will help further improvements in
area of AI game programming. So far computers were not even close to defeat one
of the chosen human players~\cite{arimaa.com}~\cite{syed}.


\section{Object of research}
Tomáš Kozelek has shown in his work, that building Arimaa playing program using
MCTS is possible. In this work we will focus on comparing capabilities and
perspectives to the future given by AlphaBeta search and MCTS in game of
Arimaa.

The main part of this work is to develop Arimaa playing program and try to
answer the following questions:

\begin{enumerate}
\item Is MCTS competitive to alpha beta search at all?
\item Is MCTS more promising engine than AlphaBeta search in the future with
      increasing number of cpus?
\item How important is evaluation function in MCTS compared to AlphaBeta?
\end{enumerate}
% TODO: which area should be compared? who is more perspective in future?
