% \chapter*{Conclusion}
% \addcontentsline{toc}{chapter}{Conclusion}

\chapter{Results}
Writing two programs at once turns out to be more difficult than we thought. In
our development we have not focussed on high time optimisations and
sophisticated goal check and trap control in position evaluation. These three
thinks are considered the most difficult in Arimaa bot development.

\section{Conclusion}
Using MCTS as is or as was provided by Kozelek is not enought for bot
programming in Arimaa, but UCB formula or UCT algorithm itself can be valuable
in yet unknown hybrid algorithm using both AlphaBeta and MCTS algorithms.

??? Combine previous and consequent paragraphs ???

The most prommising variant of the Arimaa playing program using MCTS algorithm
could be some combination of MCTS and AlphaBeta. Using full four depth
AlphaBeta search while expanding leafs of UCT. (after node expansion make
tactiacal lookahead)

REWRITE: We also believe that using more sophisticated board representation
would help MCTS algorithm against AlphaBeta, because the cost of managing
clever representation returns with more times you use the clever
representation, which dominates in MCTS. (From one position MCTS often
generates possible moves more than once in contrast of AlphaBeta).

\section{Further work}
% TODO it should be further work from our point of view (to attack another similar topics to ours)

\begin{enumerate}
\item Find way how to implement some kind of Quiescence search to MCTS and compare with similar approach in AlphaBeta.
\item ...
\item Optimise playouts using more sophisticated way how to generate steps. Interesting ideas are described in Zhong's work.\cite{ZHONG}
\end{enumerate}

The quality of bots can be improved by:
\begin{enumerate}
\item create better evaluation function,
\item try progressive pruning methods\cite{progressive-strategies},
\item use more optimised data structures and random number generator,

Goal check??

\end{enumerate}
