\chapter{Conclusion}
Writing two programs at once turns out to be more difficult than we thought. In
our development we have not focussed on high time optimisations,
sophisticated goal check and trap control in position evaluation. To handle these three topics is considered as the most difficult in Arimaa bot development.

Using MCTS as is or as was provided by Kozelek is not enought for bot
programming in Arimaa, but UCB1 formula or UCT algorithm itself can be valuable
in some combination of AlphaBeta and MCTS algorithms.

Maybe in the future we will see UCT oriented algorithm which after node expansion make short tactical (AlphaBeta) lookahead.

TODO What was done. What we achieved. What we get from it.

We believe that using more sophisticated board representation would help MCTS
algorithm against AlphaBeta, because the cost of managing clever representation
returns with more times you use the clever representation, which dominates in
MCTS. (From one position or similar to it MCTS often generates in playouts
possible moves more than once in contrast of AlphaBeta).

from \ref{generalTuning}, with increasing number of cpus, the requirements for more memory increases too (TODO citation needed), from this picture we see, that working this can help to mcts


\section{Further work}
% TODO it should be further work from our point of view (to attack another similar topics to ours)

A proceeding work could cover the following problems:

\begin{enumerate}
\item Find way how to implement some kind of Quiescence search with Goal check and Trap control to MCTS and compare with similar approach in AlphaBeta.
\item Optimise playouts using a more sophisticated way how to generate steps. Interesting ideas are described in Zhong's work~\cite{ZHONG}.
\item Try patterns heuristics in playout generation (idea taken from ~\cite{PatternsGo,PatternsArimaa}).
\end{enumerate}

\noindent The quality of bots can be improved by:
\begin{enumerate}
\item Creating better evaluation function.
\item Trying progressive pruning methods from ~\cite{progressive-strategies,
MonteCarloGo}.
\item Using more optimised data structures and random number generator.
\item Improving step-evaluation function (as We learned from \ref{pic:generalTuning}).
\end{enumerate}
