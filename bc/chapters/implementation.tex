\chapter{Implementation}
We developed set of libraries for play Arimaa to be used with both MCTS and
AlphaBeta algorithm using haskell as programming language. On top of those
libraries we built mentioned algorithms. The critical parts like bit operations
and evaluate function were written in C.

Haskell differs from other common used languages whith its lazynes. Which means
that only values that are truly needed for computation are evaluated. For
example we may define infinite list of prime numbers and then ask for third
element of this list which causes only the first three prime numbers to be
computed.

This paradigm is applied naturaly in the AlphaBeta search. When we are
exploring some node and in one of the first child's the cut-off occurs we
spared time not generating others.

% In a tree node during MCTS when children hashing mechanism is switched on,
% instead of alocating more space for $C$ cached nodes, actual nodes are lazily
% sorted in decreasing order and only $\mathcal{O}(C\log n)$ sorting realy
% hapends.

In Haskell it is possible to program in much higher level than it is in most
other programming languages. However it is also harder to reason about
performance. We are pretty sure that more experienced Haskell programmer would
write both engines more efficiently.

\subsubsection{Parameter tuning}

We made huge effort to keep our program rather simple and modular as much as
possible. Therefore one can switch on or off a lot of mentioned search
extension. In step generator we let the possibility to generate Pass step
switched off by default.

Aspiration window gave sometimes strange results so we left it unused by default.
Using history heuristic in Alpha Beta program tends to decrease quality of our
program and we believe that causing list of steps from given position to be
evaluated is limiting in comparison to have them evaluated lazily. However it
is known that importance of history heuristic grows when the depth increaseand
hence for mor efficient programs HH is much more important~\cite{ZHONG}.

Our AlphaBeta algorithm lack of standart Quiescence extension with Trap control
or the \ac{Goal check}. We believe that both mentioned extensions could be
somehow included in MCTS.

\section{Parallelization}
In Monte Carlo Tree Search, make bot to be parallel is much more natural than
it is in AlphaBeta search. For each CPU one thread is created and there is only
one shared tree for all of them. Haskell gave us easy threading and simple data
structure locking capabilities thanks to standart MVar.

In a parallelised AlphaBeta search we start one thread per CPU. Every iteration
of Iterative deepening framework creates a shared queue filled with all
possible steps from starting position. Then each thread is taking steps eagerly
from queue and updating the window and killer moves.

\section{Evaluation function}
We wrote evaluation function completely in C to be as fast as possible. It
consist of hand tuned material and position evaluation, simple bonuses for
having stronger piece advantage, controlling trap, freezing opponents piece and
possibility to move oponents weaker piece.

In our work did not focus on creating strong evaluation function even though in
order to have arimaa playing program, we need to have one. Therefore we
developed one not so strong and with possibility to easily replace it.

Our evaluation function lack of a very important part of standart evaluation
function -- the \ac{Goal check}. Also it do not have frame or hostage detection.
