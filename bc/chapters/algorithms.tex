\chapter{Algorithms}
In this chapter we introduce two algorithms for searching minimax trees
AlphaBeta search and Monte Carlo Tree Search (MCTS). Both algorithms became
successful and dominating in some field. AlphaBeta search in game of Chess and
MCTS in game of Go. (TODO reformulate)

The basic algorithm to search game tree is Minimax search. It is searching
minimax tree in depth first search order and therefore the search do not ends
until all nodes at certain depth are explored. It is quite noneffective with
huge branching factor which Arimaa, Chess or Go have.

\section{Description of the AlphaBeta search}
AlphaBeta search has grown in popularity in game of Chess. Until now all
successful arimaa bots were using it with various enchants. AlphaBeta search is
natural optimisation of Minimax search. The main purpose of algorithm is to
reduce number of branches and nodes to be visited.

During the search we are trimming bounds (window) of the best minimax value.
The pseudocode of the algorithm is shown in~\ref{alphabeta}.

\lstset{language=Python, caption=Pseudocode of the AlphaBeta search, label=alphabeta}
\begin{lstlisting}
alphabeta (node, depth, alpha, beta):
    if  depth = 0 or node is a terminal node:
        return evaluate(node)
    if  is maximizing node:
        for each child of node
            score = alphabeta(child, depth-1, alpha, beta,
                              opponent(Player))
            alpha = max(alpha, score)

            if beta <= alpha: break  # Beta cut-off
        return alpha
    else:
        for each child of node:
            score = alphabeta(child, depth-1, alpha, beta,
                              opponent(Player))
            beta = min(beta, score)

            if beta <= alpha: break  # Alpha cut-off
        return beta
\end{lstlisting}

In maximizing nodes we are improving lower estimate (alpha) of the minimax
value and in minimizing nodes the upper estimate (beta). If a value from a
child forces these bounds to meet we know that better approximation of this
node is not possible and so we return our estimate, we say we pruned the search
in node.

As can be proven that if AlphaBeta finds solution for depth $n$, then it is the
best solution in MiniMax tree for depth $n$~\cite{knuth:alphabeta}. In optimal
case instead of examining $\mathcal O(b^d)$ nodes is only examined $\mathcal
O(b^{d/2})$ of them, where $b$ is branching factor of the game and $d$ is
searched depth~\cite{ZHONG}.

A lot of time is spared if the pruning children of nodes are listed as first.
However in the worst case if children are sorted in opposite order to the
optimal, the whole minimax tree is searched. Consequently it is very important
to have nodes well ordered.


\section{Description of the Monte Carlo Tree search}
Monte Carlo methods was formerly used for approximation of mathematical,
physical, biological and others processes where full calculation would be
difficult or even impossible. For example in mathematics it is used for numeric
integration or estimate the $\pi$. In games without perfect information, such
as poker, backgammon, or scrabble, using randomness has shown to be beneficial
too~\cite{MonteCarloMethod, MonteCarloGo}.

In game of Go was always a huge troublesome to build efficient evaluation
function. Unlike poker or backgammon, Go is a game with perfect information and
hence it is not so natural to use Monte Carlo methods. The first attempts to
use random approach as evaluating approximation of the Go position were in
1993. We present generalised Bernd's algorithm~\cite{BERND,KOZELEK}:

\begin{enumerate}
\item Play random game from given position to the end, with one exception,
	choose only steps not filling eyes. At the end of simulation count in the
	result of simulation for the first step played.
\item If there is time left go to 1.
\item Choose move with highest ratio between number times the move won when it
	  was played and the number of times it was played.
\end{enumerate}

However Bernd showed that after some time his algorithm indicates no further
improvement. It was necessary to realize that the main problem was that it
searched all branches with the same probability. (TODO) .... It should give more
promising branches more attention.


\subsection{Bandit Problem}
A $K$-armed bandit, is slot machine with $K$ arms. When arm is drawn it
produces reward. Distribution of each arm reward is independent on other arms
and previous draws of this arm. The task is to choose best strategy to maximize
sum of reward through iterative plays.~\cite{MoGo,MultiarmedBandit}

In~\cite{MultiarmedBandit} is presented thee UCB1 algorithm (where UCB stands for Upper Confidence Bounds) for Bandit Problem:

\begin{enumerate}
\item Play each arm of the bandit once.
\item Play arm maximizing the formula $\overline X_i + \sqrt{2 \log n \over n_i}$,
	  where $\overline X_i$ is average value of the arm $i$, $n$ is number
	  of games that were played by parent of the $i$ and $n_i$ is number of
	  games played with arm $i$.
\end{enumerate}


\subsection{UCT algorithm}
UCT is Upper Confidence bound to Trees or in other words UCB applied to minimax
trees. The main idea is to consider each node of minimax tree as multiarmed
bandit problem and each child of the node as independent arm.

In MoGo they used UCB1 algorithm extended to searching minimax tree called UCT.
MoGo \cite{MoGo}

UCT is algorithm traversing built multi-armed bandit tree using UCB1 formula.

Playing random simulation is also called playout.

We presume the reader is familiar with rules of the game Go. (TODO ???)
	
Standalone UCT needs set of carefully chosen extensions to be competitive. The
MCTS is some variant of UCT algorithm with those extensions.

UCB1 formula itself presumes the random involved variables are identically distributed and independent which is not true in UCT.
As MCTS we call UCT algorithm with various of extensions.

MCTS algorithm starts with a tree containing only the root node representing starting position.
MCTS consist of four steps, which we repeat until time is up:
\begin{enumerate}
\item \emph{Selection}
\item \emph{Expansion}
\item \emph{Simulation}
\item \emph{Backpropagation}
\end{enumerate}
\cite{progressive-strategies}

\lstset{language=Python, caption=Pseudocode of the MonteCarlo Tree Search, label=mcts:alg}
\begin{lstlisting}
playOneSequence(rootNode):
    node[0] = rootNode
    i = 0
    while node[i] is not visited for first time:
        node[i+1] = descendByUCB1(node[i])
        i = i + 1

    createNode(node[i])
    node[i].value = getValueByMC(node[i])

    for j in {0, ..., i-1}:
        updateValue(node[j],-node[i].value)
\end{lstlisting}

We use Monte Carlo Tree Search algorithm as is described in~\cite{MoGo} and also
modified by Kozelek~\cite{KOZELEK}.

We modified Kozelek's UCB1 exploration formula to form:
	$$
	\overline X_i + c \sqrt{\frac{log~n}{n_i}} + \frac{h_i}{n_i} + \frac{hh_i}{\sqrt{n_i}}
	$$
where $\overline X_i + c \sqrt{\frac{log~n}{n_i}}$ is a generalised UCB1
formula, $h_i$ is heuristics evaluated for step leading to this position, and
$hh_i$ is history heuristic value.
