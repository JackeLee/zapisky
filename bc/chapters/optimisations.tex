\chapter{Used optimisations in search engines}

In order to write a strong playing program in given game, it is necessary to
enhance chosen algorithm with various extensions. Tuning a used set of these
extensions and their parameters is one of the key factors in bot development.
In this chapter we will describe the most used extensions for AlphaBeta or MCTS
algorithms. In this chapter we will describe optimisations we tried to
implement in our game engines.

\section{AlphaBeta}
Implementing variety of AlphaBeta extensions has long background in game of
Chess. In~Arimaa, usually some combination of well approved enhances is
used. TODO

\subsection{Transposition Table}\label{AlphaBeta:TT}
If we look at the game tree of Arimaa there is so many repetitions in nodes of
the tree for almost every position on the board. The Transposition table is
used to reduce the number of repetitions.

In Arimaa thanks to Step-Move approach of the game, handling transpositions is
a little harder than it is in chess. We have to worry about number of steps in
move that were made on top of the classic transpositions handling.

When using transposition tables, every time we are about to explore some node,
we look at the transposition table first. If a entry of the same transposition
occurs from previous search, depending on its searched depth we do:

\begin{itemize}
\item If the entry's search is shallower than we need, we prefer steps from
Principal Variation of given entry in succeeding search and adjust also our
bounds with its.
\item However if the entry has been searched to equal or greater depth we stop
searching of this node and use a full Principal Variation and a score of that
entry as result of exploring our node.
\end{itemize}

After arbitrary position is fully explored, its best value and a corresponding
Principal Variation is saved to the Transposition table~\cite{COX}.

TODO reformulate

\subsection{Iterative Deepening Framework}
Because there is no way how to determine how long the AlphaBeta search to
given depth will take, we need to find some time management tool. This is a
method in which we are iteratively starting new deeper and deeper searches.
Thanks to the exponential growth of the minimax tree with increasing depth we
know that a searching one level more will cost us approximately a lot more than
previous shallower searching.

Because we can sort moves using information gained from previous shallower
searches Iterative deepening used with other optimisation methods such as
History Heuristic or Transposition Table often cause a time save if we compare
it to just searching to a maximal possible depth~\cite{COX}.

\subsection{Aspiration Windows}
Iterative Deepening itself can be further improved. The AlphaBeta search
normally starts with $(\alpha,\beta)$ bounds (window) set to
$(-\infty,\infty)$. When an Aspiration window is set, we start a search with
$(prev - window, prev + window)$ window instead, where $prev$ is the value from
previous shallower search and $window$ is certain predefined constant. If the
result of the search fall outside the window, the new search with wider window
must be performed.

With narrower windows, more pruning during AlphaBeta search should occur and
therefore the search to certain depth should end much earlier. However if a
value of a search miss a window so often, the need for repetitions may actually
linger the search~\cite{AspirationWindow}.


\subsection{Move ordering}
The following methods change the order in which branches are selected and then
inspected. It is very important for the AlphaBeta search to have nodes well
ordered, because earlier we find a pruning child of a node the shorter time
we spend in it.

	\subsubsection{History heuristics}
	The main idea behind this extension is that if some step is good enough
	to cause so many pruning anywhere in the search tree and if it is valid in
	given position it could be also good here.

	A table of scores for every combination of player, piece, position and
	direction is stored into memory. During the AlphaBeta search we increase a
	score of an element every time a step with such combination causes pruning
	or became a child with the best score for an actually searched node. It is
	believed that the deeper cut-off happens the more relevant it is and
	therefore the score is incremented by $d^2$ or $2^d$ where $d$ is an actual
	searched depth.

	During searching are nodes of the minimax tree sorted by score from the
	History Heuristic table in decreasing order~\cite{COX}.

	\subsubsection{Killer moves}
	When a step prunes other branches in some node it is very natural to ask,
	if the same step could cause pruning in another branch and the same depth
	of the tree.
	
	In order to answer this question the last pruning step from the same
	position is tried right after Transposition Table's Principal Variation.
	
	To go even further we take two last steps causing pruning to be
	preferred in the search. As Cox found out, three or more Killer moves
	would not help~\cite{COX}. TODO

	\subsubsection{Null move}
	The main idea is that if a player is in bad position and if we skip its
	opponent's turn and the position cannot be strongly improved, then there is
	no chance for this position to be good.

	During searching in a new-turn nodes, either the whole opponent's turn is
	skipped (which is called Null move) or normal search is performed. In
	reality the Null move is searched as first child of a node and it shortens
	the time until cutoff occurs in really bad positions. Performing two Null
	moves in row or in the root node is forbidden~\cite{COX}.


\subsection{Heuristics}
All preceding optimisations do not change result given by the algorithm. They
could only significantly decrease amount of time needed to obtain such result.

There are also other optimisations, which are rather heuristics, because they strongly depends on good children ordering and with 

try to
give you approximately good result in much shorter time. In this work, we
will not use those optimisations.

TODO Introduce  Negascout/PVS and MTD-f or delete this subsection.


\section{Monte Carlo Tree Search}
In AlphaBeta search we made great effort in sorting nodes of the searched tree
properly. In MCTS we need to use optimisations which helps us to gain more and
better informations from each iteration of algorithm on top of that.

We chosed to implement only heuristic Tomáš Kozelek described in~\cite{KOZELEK}
as useful.

\subsection{Transposition Table}
The motivation is the same as is in the AlphaBeta algorithm
(see~\ref{AlphaBeta:TT}). However in MCTS we are increasingly building game
tree instead of just exploring branches to some depth. A natural use of
Transposition Table for MCTS is to share statistics for the same transpositions
in built game tree.

To do so we bind nodes considered the same to one. Bound nodes share their
children nodes, visit count and score statistic. We say that two nodes are the
same if they are in the same depth in minimax tree and if they represent the
same transposition. The game tree became Directed Acyclic Graph.

In implementation during the computation we keep table of all transpositions
and when any node is expanded we bind it to transposition in table if exists.

In Kozelek's work is regarded to be dangerous to bound visit count and score
statistic for children nodes~\cite{KOZELEK}. Nevertheless we believe that if
some step leads us to a position which is proven to be not worth trying in some
branch the same stands for all its transpositions.

\subsection{Progressive bias}
Progressive bias technique is a nice way how to combine offline learned
informations with online learned informations. The more we go through a node
the importance of offline learned information goes down and the importance of
online learned information became superior.

To do so ${H_B \over n_i}$ is added to the UCB formula. Where $H_B$ is
progressive bias coefficient computed by step-evaluation
function~\cite{progressive-strategies}.

In Arimaa such step-evaluation function should appreciate steps with Elephant
moving, killing an opponent's piece, around previous steps, which are pushing or
pulling or making goal. Such function should also handicap player's own piece
sacrifice steps or inverse steps~\cite{KOZELEK}.

\subsection{History heuristics}
Tomáš Kozelek brought this optimisation used in AlphaBeta to MCTS. Similarly to
AlphaBeta's approach, it is used to share informations gained in one branch of
the searched tree.

In order to rate steps (TODO), we keep statistics (score and visit count) for
each combination of player, piece, location and direction. These statistics are
updated during the Backpropagation part of the MCTS.

For each updated node we also update the statistics of the step leading to that
node. (TODO - actually we update only nodes leading from this position to leafs) The $+ {hh_i \over n_i}$ expression is added to UCB1 formula.
Where $hh_i$ is history heuristics coefficient representing the mean value of
step leading to $i$s child of a searched node score. Original idea is described
in Kozelek's work~\cite{KOZELEK}.

\subsection{Best-of-$N$}
In random simulations, we may want to sacrifice true randomness for gaining
more objective results from playouts~\cite{HeavyPlayouts}. $N$ random steps are
generated and a step with the best value given by the same step-evaluation
function as was used in Progressive bias is chosen.

As a consequence, the number of playouts decreases, but the quality of
information learned in playouts improve and therefore the strength of the
program improve also.

\subsection{Children caching}
Kozelek introduced natural method how to decrease an amount of time spent in
node during selection part of the MCTS.

After some number of visits of a node its children caching is switched on. Then
a few best children are chosen and cached and every time the selection part of
the MCTS goes through this node it chooses a descend node from earlier cached
children. After some time it is necessary to discard and fill cache
again~\cite{KOZELEK}.

Using this optimisations should improve the speed of algorithm without negative
impact on quality.

\subsection{Maturity threshold}
Is another technique how to shorten time used in node selection. We
expand only those nodes which had at least $threshold\_maternity +
depth\_of(node)$ visit count.

\subsection{Virtual visits}
In node expansion initialise new node with $v$ virtual visits. This small
change increases the power of the algorithm significantly. Kozelek
experimentally determined the best performance of algorithm for $v \in
[4,5]$~\cite{KOZELEK}.

\section{Independent optimisations}
The following extensions are considered must have in every successful Arimaa
bot and do not depend on used algorithm.

	\subsection{Bitboards}
	Because Arimaa could be played with standart chess set it is also possible
	to use from world of chess well known board representation called
	Bitboards.

	We represent board as twelve 64bit numbers, one for each combination of
	player and piece kind. The $i$th bit of certain piece and player
	combination is set to \texttt{1} if on $i$th square (counting from
	\texttt{a1} to \texttt{h8} by lines) corresponding stands, otherwise it
	is set to \texttt{0}. For example if a Golds Dog is on \texttt{b2} the 10th
	bit in the Golds Dogs number is set to \texttt{1}.

	One may say that the Bitboards suits even better for Arimaa than for Chess.
	For instance computing simple steps is for almost all pieces just taking
	precomputed bit number of all adjacent squares for his coordinates and
	bit-anding it with 64bit number representing all empty
	squares~\cite{BitboardAnalysis}. TODO bit-anding vs XORing

	\subsection{Zobrist keys}
	Zobrist keys are one of the optimisation methods well known from world of
	chess. With using Transposition Table comes the need for fast determination
	mechanism if two positions are the same, and for effective generating
	almost unique hash keys for storing transpositions into hash table.
	
	For every tripes $piece\times player\times position$ the random 64bit
	number is generated and stored to array as
	\texttt{zobrist[piece][player][position]}. A hash value is then computed by
	taking for each piece on board the corresponding number from the
	precomputed \texttt{zobrist} array and \emph{XOR}ing all those numbers
	together.

	Calculating a new hash value for board is shown in pseudocode
	~\ref{alg:zobrist}.

	\lstset{language=Python, caption=Computing hash value from zobrist keys, label=alg:zobrist}
	\begin{lstlisting}
    zorbist_hash (board):
        key = 0x0000000000000000      # key of empty board
        for p in pieces_of(board):
            hash = zobrist[piece(p)][controller(p)][position(p)]
            key = key XOR hash

        return key
    \end{lstlisting}

	Computing such function every time the hash value is needed would not be
	efficient at all. The value can be easily updated after step is made by
	\emph{XOR}ing with:
	\begin{center}
		\texttt{zobrist[piece][player][from] XOR zobrist[piece][player][and]}
	\end{center}
	Or for push/pull steps by doing the same simultaneously with both a
	pushing/pulling piece and a pushed/pulled piece.

	Probability of collision for two zobrist hashes in typical search is less
	than $2.2 \times 10^{-5}$~\cite{COX}.

% TODO
% \section{Comparison of used optimisations}
% 
% \subsection{History heuristic}
% In MCTS we prefer moves that were often chosen in going through tree. In
% AlphaBeta value of move is increasing when branch using selected step is
% pruned. Every time we order branches of the search tree by value given by
% history heuristics. This approach should increase amount of pruning.
% 
% \subsection{Transposition Table}

