\documentclass[11pt,a4paper]{article}
\usepackage[utf8]{inputenc}
\usepackage[czech]{babel}
\usepackage{amsfonts,amsthm,amsfonts,amssymb,amsmath}
\usepackage{a4wide}
\usepackage{enumerate}
\usepackage[IL2]{fontenc}

\newcommand\m[1]{\mathbb { #1 }} % tucne pismeno/text
\newcommand\p[1]{\mathcal{ #1 }} % psaci pismeno/text
\newcommand\IFF{\ensuremath{\iff}}
\newcommand\N{\m N}

\newcommand*{\todo}[1]{\textbf{TODO} #1}

\newcommand*{\ml}[1]{\[\textstyle\let\displaystyle\textstyle#1\]}	% math line in textstyle
\newcommand*{\mld}[1]{\[#1\]} % math line in displaystyle

\DeclareMathOperator{\alt}{alt} % Krullova dimenze
\DeclareMathOperator{\gen}{gen} % nejmenší počet generátorů
\let \icl \overline % celistvý uzávěr

\newenvironment{items}{% nečíslovaný seznam -- zde možno nastavit jednotný styl
	\itemize
	\itemsep = 0pt%
}{%
	\enditemize
}
\newenvironment{itemsn}{% číslovaný seznam
	\enumerate%
	\itemsep = 0pt%
}{%
	\endenumerate%
}

\newcounter{numb}

\theoremstyle{definition}
\newtheorem*{definice}{Definice}
\newtheorem{pozorovani}[numb]{Pozorování}
\newtheorem{poznamka}[numb]{Poznámka}		% pozorování se nečísluje a pounámka ano? to už bych to spíš čekal obráceně

\theoremstyle{plain}
\newtheorem{veta}[numb]{Věta}
\newtheorem{lemma}[numb]{Lemma}
\newtheorem{tvrzeni}[numb]{Tvrzení}
\newtheorem{dusledek}[numb]{Důsledek}

\title{Konstrukce noetherovského okruhu, jehož celistvý uzávěr\\ není noetherovský}
\author{Adam Bartoš, Tomáš Jakl, Jan Kosina, Martin Raška, Jiří Vančura}


\begin{document}
\maketitle

\section{Úvod}
V našem textu budeme pracovat pouze s komutativními okruhy.

\todo{Zhruba popsat obsah práce}

\section{Formální mocninné řady nad tělesem a odvozené obory}

V této kapitole budeme zkoumat vlastnosti oborů odvozených od oboru formálních mocninných řad na tělesem v konečně mnoha neurčitých. Zajímat nás bude noetherovskost, lokalita, regularita a celistvá uzavřenost. Na základě zde prezentovaných obecných poznatků bude vybudován hledaný příklad noetherovského okruhu, jehož celistvý uzávěr není noetherovský.

\begin{definice}
	Nechť $R$ je okruh, $X$ množina neurčitých, pak $R[X]$ značí okruh polynomů nad $R$ v neurčitých $X$ a $R[[X]]$ značí okruh formálních mocninných řad nad $R$ v neurčitých $X$.
\end{definice}

\begin{pozorovani}
	Je-li $K$ těleso, $X$ množina neurčitých, pak $K[[X]]$ je obor integrity, neboť vlastnost \uv{býti oborem} se zřejmě přenáší z výchozího okruhu na okruh formálních mocninných řad.
\end{pozorovani}

\begin{pozorovani} \label{thm:TXK}
	\newcommand*{\I}{_{i = 1}^n}

	Nechť $T \subseteq K$ jsou tělesa, $X$ je množina neurčitých. Uvažme obor $T[[X]][K] \subseteq K[[X]]$. Řada $f \in K[[X]]$ je dle definice prvkem $T[[X]][K]$ právě tehdy, když existují $\{f_i\}\I \subseteq T[[X]]$ a $\{k_i\}\I \subseteq K$, že $f = \sum\I f_i k_i$. Konečnou množinu prvků $k_i$ označme jako $K_f$.

	Vidíme, že koeficienty řady $f$ leží v nadtělese $T(K_f)$. Je-li množina $K_f$ tvořena prvky algebraickými nad $T$, pak $T(K_f)$ je algebraické rozšíření konečného stupně a takové je i rozšíření generované koeficienty řady $f$.

	Naopak, generují-li koeficienty řady $f \in K[[X]]$ tělesové rozšíření $T$ konečného stupně, konečná množina $K_f$ existuje, totiž konečná báze příslušného rozšíření nad $T$, a tedy $f \in T[[X]][K]$.

	Dále, je-li $F \subseteq K$ konečná množina prvků algebraických nad $T$, pak platí
	\ml{
		T[[X]][F] \subseteq T[F][[X]] \subseteq T(F)[[X]] \subseteq T[[X]][K].
	}
\end{pozorovani}


\subsection{Noetherovskost}

\begin{veta}[Zobecněná Hilbertova věta o bázi] \label{thm:GHB}
	Buď $R$ okruh, $X$ neurčitá, pak následující je ekvivalentní.
	\begin{itemsn}
		\item $R$ je noetherovský.
		\item $R[X]$ je noetherovský.
		\item $R[[X]]$ je noetherovský.
	\end{itemsn}
\end{veta}

\begin{tvrzeni} \label{thm:KX_noe}
	Nechť $K$ je těleso, $X$ konečně mnoho neurčitých. Pak $K[[X]]$ je noetherovský obor.

	\begin{proof}
		Těleso $K$ je jistě noetherovský okruh. Postupným použitím zobecněné Hilbertovy věty o bázi (\ref{thm:GHB}) na jednotlivé neurčité, kterých je konečně mnoho, dostaneme požadované.
	\end{proof}
\end{tvrzeni}

\begin{lemma}
	Buď $S$ noetherovský okruh a buď $R \subseteq S$ jeho podokruh. Pokud pro každý konečně generovaný ideál $I \subseteq R$ platí, že $IS \cap R = I$, pak $R$ je noetherovský.

	\begin{proof}
		Budeme postupovat sporem. Ať $R$ není noetherovský okruh, pak tedy existuje ideál $I \subseteq R$, který není konečně generován. Dále existuje nekonečná posloupnost ideálů $\{I_i: i \in \N\}$ tak, že:
		\ml{
			I_1 \subsetneq I_2 \subsetneq \dots \subseteq I.
		}
		Potom, ale pro posloupnost ideálů $\{I_i S: i \in \N\}$ v $S$ z noetherovskosti existuje $n \in \N$ tak, že $I_n S = I_{n + 1} S$. Podle předpokladu je ale $I_n = I_n S \cap R = I_{n + 1} S \cap R = I_{n + 1}$, ale $I_n \not= I_{n+1}$, což je spor.
	\end{proof}
\end{lemma}

\begin{poznamka}
	Inkluze $I \subseteq IS \cap R$ z předpokladu předchozího lemmatu je zřejmě splněna.
\end{poznamka}

\begin{tvrzeni}
	Nechť $T \subseteq K$ je algebraické rozšíření těles, $X$ konečně mnoho neurčitých. Pak obor $T[[X]][K]$ je noetherovský.

	\begin{proof}
		\newcommand*\I{_{i = 1}^n}

		Označme $R = T[[X]][K]$, $S = K[[X]]$. $S$ je dle tvrzení \ref{thm:KX_noe} noetherovský obor. K použití předchozího lemmatu stačí ukázat, že pro každý konečně generovaný ideál $I \subseteq R$ platí, že $IS \cap R \subseteq I$.

		Buď $I = \sum\I r_i R$ konečně generovaný ideál v $R$. Vol $r \in IS \cap R$. Pak
		\ml{
			r \in IS = \sum\I r_i RS = \sum\I r_i S.
		}
		Tedy $r = \sum\I r_i f_i$ pro nějaké řady $\{f_i\}\I \subseteq S$.

		Uvažme $F = K_r \cup \bigcup\I K_{r_i}$ a $U = T(F)$. $F$ je konečná množina prvků $K$, které je algebraické nad $T$, a $U$ je tedy rozšíření $T$ konečného lineárního stupně. Dle pozorování \ref{thm:TXK} máme $r, r_i \in U[[X]] \subseteq R$, $i = 1, \hdots, n$.

		Dále uvažme rozklad $K = U \oplus K'$. Potom i $K[[X]] = U[[X]] \oplus K'[[X]]$. Rozložme $f_i = g_i + h_i \in U \oplus K'$. Potom máme
		\ml{
			% U[[X]] \owns r = \sum\I r_i g_i + \sum\I r_i h_i \in U[[X]] \oplus K'[[X]].
			\underbrace{r}_{\in U[[X]]} = \underbrace{\sum\I r_i g_i}_{\in U[[X]]} + \underbrace{\sum\I r_i h_i}_{\in K'[[X]]}
		}
		Tedy $\sum\I r_i h_i = 0$ a $r = \sum\I r_i g_i \in \sum\I r_i U[[X]] \subseteq \sum\I r_i R = I$.
	\end{proof}
\end{tvrzeni}

\begin{tvrzeni}
	Nechť $R \subseteq S$ jsou okruhy, $R$ je noetherovský, $S = R[F]$ pro nějakou konečnou $F \subseteq S$. Pak $S$ je také noetherovský.

	\begin{proof}
		Nechť $F = \{s_1, \hdots, s_n\}$. Pak
		\ml{
			S \simeq R[x_1, \hdots, x_n] / \{f \in R[x_1, \hdots, x_n]: f(s_1, \hdots, s_n) = 0\},
		}
		což je dle Hilbertovy věty o bázi (\ref{thm:GHB}) faktor noetherovského okruhu, tedy noetherovský okruh.
	\end{proof}
\end{tvrzeni}

\begin{dusledek} \label{thm:noe}
	Nechť $T \subseteq K$ je algebraické rozšíření těles, $X$ konečně mnoho neurčitých, $F \subseteq K[[X]]$ konečná. Pak obory $T[[X]][F]$ a $T[[X]][K][F]$ jsou noetherovské.
\end{dusledek}


\subsection{Lokalita}

\begin{definice}
	Řekneme, že okruh $R$ je \emph{lokální}, existuje-li v něm jediný maximální ideál. Tento budeme značit $M_R$.
\end{definice}

\begin{lemma} \label{thm:KX_inv}
	Nechť $K$ je těleso, $X$ libovolně mnoho neurčitých. Pak v $K[[X]]$ jsou invertibilní právě řady s nenulovým absolutním členem.

	\begin{proof}
		Zvolme $f \in K[[X]]$ s nenulovým absolutním členem. Označme $\p M$ množinu všech monických monočlenů v neurčitých $X$. Pak můžeme psát $f = \sum_{m \in \p M} f_m m$. Bez újmy na obecnosti absolutní člen $f_1 = 1$. Hledáme $g = \sum_{m \in \p M} g_m m \in K[[x]]$, že $f g = 1$. Chceme, aby $f_1 g_1 = 1$, tedy $g_1 = 1$. Dále pro $m \in \p M \setminus \{1\}$ chceme následující:
		\ml{
			0 = (f g)_m = \sum_{\substack{n, n' \in \p M\\ n n' = m}} f_{n'} g_n = f_1 g_m + \sum_{\substack{n, n' \in \p M\\ n n' = m\\ n \not= m}} f_{n'} g_n = g_m + \sum_{\substack{n, n' \in \p M\\ n n' = m\\ \deg n < \deg m}} f_{n'} g_n.
		}
		Takže
		\begin{equation*} \label{eq:inv}
			\textstyle
			g_m = -\sum_{\substack{n, n' \in \p M\\ n n' = m\\ \deg n < \deg m}} f_{n'} g_n \tag{$\ast$}.
		\end{equation*}
		Máme předpis pro $g_m$, které závisí pouze na $g_n$ pro $\deg n < \deg m$, tedy induktivní konstrukce dává výsledek.

		Naopak, má-li $f$ nulový absolutní člen, pak každý jeho násobek má rovněž nulový absolutní člen. Tedy $f$ není invertibilní.
	\end{proof}
\end{lemma}

\begin{tvrzeni}
	Nechť $K$ je těleso, $X$ konečně mnoho neurčitých. Pak $K[[X]]$ je lokální obor s $M_{K[[X]]} = \sum_{x \in X} x K[[X]]$.

	\begin{proof}
		Dle předchozího lemmatu jsou neinvertibilní prvky v $K[[X]]$ právě ty s nulovým absolutním členem. Tyto pak zřejmě tvoří jediný maximální ideál, který je roven $\sum_{x \in X} x K[[X]]$, protože každý monočlen obsahuje nějakou neurčitou $x$, a tedy ho můžeme přidat do členu rozkladu $x K[[X]]$.
	\end{proof}
\end{tvrzeni}

\begin{tvrzeni}
	Nechť $T \subseteq K$ je algebraické rozšíření těles, $X$ konečně mnoho neurčitých. Pak v $R = T[[X]][K]$ jsou invertibilní právě prvky s nenulovým absolutním členem a $R$ je lokální obor s $M_R = \sum_{x \in X} xR$.

	\begin{proof}
		Víme, že tato charakteristika invertibilních prvků platí v celém $K[[X]]$. Je-li $f \in R$ s nenulovým absolutním členem, pak dle lemmatu \ref{thm:KX_inv} jeho invers v $K[[X]]$ získáme vztahem \eqref{eq:inv}. Fixujeme-li konečnou množinu algebraických prvků $K_f$, že $f \in T(K_f)[[X]]$, pak i invers $g \in T(K_f)[[X]] \subseteq T[[X]][K]$, tedy získaný invers zůstane v $R$.

		Neinvertibilní prvky, což jsou právě prvky s nulovým absolutním členem, tvoří jediný maximální ideál $M_R = R \cap \sum_{x \in X} x K[[X]] \supseteq \sum_{x \in X} x R$. Nyní ukážeme, že $M_R \subseteq \sum_{x \in X} xR$. Je-li $f \in M_R$, pak každý jeho člen obsahuje nějakou neurčitou $x \in X$. Vyberme pro každý člen jednu neurčitou a členy rozdělme podle této vybrané neurčité: $f = \sum_{x \in X} f_x \in \sum_{x \in X} x K[[X]]$. Pro každé $f_x$ navíc můžeme položit $K_{f_x} = K_f$, a tedy $f_x \in xR$.
	\end{proof}
\end{tvrzeni}

Nyní od obecné situace algebraického rozšíření těles $T \subseteq K$ přejdeme k situaci, kdy $K$ má kladnou charakteristiku $p$ a $T = K^p$. Jedná se o speciální případ, protože $K$ je zřejmě algebraické nad $K^p$.

\begin{lemma}
	Nechť $R$ je okruh prvočíselné charakteristiky $p$, $X$ jedna neurčitá, pak pro $f = \sum_{i = 0}^\infty f_i X^i \in R[[X]]$ platí $f^p = \sum_{i = 0}^\infty f_i^p X^{ip}$. Tedy mocnění na $p$-tou probíhá člen po členu.

	\begin{proof}
		Máme $f = f_0 + (f - f_0) = f_0 + X g$, $g \in R[[X]]$. Pak
		\ml{
			f^p = \sum_{i = 0}^p \binom{p}{i} f_0^i (X g)^{p - i} = f_0^p + X^p g^p,
		}
		protože ostatní binomické koeficienty jsou dělitelné $p$, a tedy nulové. Dál můžeme pokračovat indukcí.
	\end{proof}
\end{lemma}

\begin{dusledek} \label{thm:p_power}
	Nechť $R$ je okruh prvočíselné charakteristiky $p$, $X$ konečně mnoho neurčitých. Pak i v $R[[X]]$ probíhá mocnění na $p$-tou člen po členu.

	\begin{proof}
		Stačí použít předchozí lemma postupně pro přibývající neurčité. Charakteristika $p$ se zřejmě přenáší i do okruhu formálních mocninných řad.
	\end{proof}
\end{dusledek}

\begin{tvrzeni} \label{thm:R_loc}
	Nechť $K$ je těleso charakteristiky $p > 0$, $X$ konečně mnoho neurčitých, $R$ meziobor: $K^p[[X]] \subseteq R \subseteq K[[X]]$. Pak $f \in R$ je invertibilní, právě když má nenulový absolutní člen, a $R$ je lokální obor s $M_R = R \cap M_{K[[X]]} \supseteq \sum_{x \in X} x R$.

	\begin{proof}
		Nechť $f \in R$ má nenulový absolutní člen. Pak dle předchozího důsledku $f^p \in K^p[[X]]$ a má rovněž nenulový absolutní člen. Dle lemmatu \ref{thm:KX_inv} existuje $g \in K^p[[X]]$, že $f^p g = 1$. Potom ale $f f^{p - 1} g = 1$ a $f^{p - 1} g \in R$, tedy $f$ je invertibilní v $R$.

		Naopak řady s nulovým absolutním členem tvoří vlastní ideál, tedy nemohou být invertibilní. Tento vlastní ideál je díky předchozímu odstavci jediný maximální, a tedy $R$ je lokální. Zbytek tvrzení je zřejmý.
	\end{proof}
\end{tvrzeni}


\subsection{Regularita}

\begin{definice}
	Je-li $R$ okruh, pak \emph{Krullovu dimenzi} definujeme jako
	\ml{
		\alt R = \sup\{|\p P|: \p P \text{ je ostře rostoucí posloupnost prvoideálů}\} - 1,
	}
	tedy jako největší počet skoků v rostoucím řetězci prvoideálů.

	Dále pro $I \subseteq R$ ideál definujme
	\ml{
		\gen I = \min\{|G|: G \subseteq R, G \text{ generuje } I\},
	}
	tedy nejmenší počet generátorů $I$.
\end{definice}

\begin{definice}
	Nechť $R$ je okruh. Řekneme, že $R$ je \emph{regulární lokální okruh}, je-li lokální, neotherovský a platí $\alt M_R = \gen M_R$.
\end{definice}

\begin{poznamka}
	Lze ukázat, že regulární lokální okruh už je oborem integrity.
\end{poznamka}

\begin{veta}[Věta o Krullově dimenzi]
	Buď $R$ lokální noetherovský okruh. Platí
	\ml{
		\alt R \leq \gen M_R.
	}
\end{veta}

\begin{tvrzeni} \label{thm:R_reg}
	Buď $K$ těleso, $X = \{X_1, \hdots, X_n\}$ konečně mnoho neurčitých, $R$ lokální noetherovský podobor $K[[X]]$ takový, že $R \cap \sum_{x \in X} x K[[X]] = \sum_{x \in X} x R$. Pak $R$ je regulární lokální obor Krullovy dimenze $n$.

	\begin{proof}
		Dle předpokladů je $R$ lokální noetherovský obor.
		\begin{items}
			\item $M_R = \sum_{i = 1}^n X_i R$, tedy $\gen M_r \leq n$.
			\item Dle věty o Krullově dimenzi $\alt R \leq \gen M_R$.
			\item $\{\sum_{i = 0}^k X_i R: k = 0, \hdots, n\}$ je ostře rostoucí řetězec prvoideálů, protože obsahuje-li každý člen součinu řad alespoň jednu z neurčitých $X_1, \hdots, X_k$, musí alespoň jednu z nich obsahovat i každý člen obou řad (jinak součin členů nejmenších stupňů, které neobsahují žádnou takovou neurčitou, také žádnou z nich neobsahuje). Tedy $n \leq \alt M_R$.
		\end{items}
		Celkem $n \leq \alt M_R \leq \gen M_R \leq n$.
	\end{proof}
\end{tvrzeni}

\begin{dusledek} \label{thm:reg}
	Buď $T \subseteq K$ algebraické rozšíření těles, $X$ konečně mnoho ($n$) neurčitých. Pak $T[[X]] \subseteq T[[X]][K] \subseteq K[[X]]$ jsou regulární lokální obory dimenze $n$.
\end{dusledek}


\subsection{Celistvá uzavřenost}

\begin{definice}
	Nechť $R$ je obor integrity. Pak $\icl{R}$ značí celistvý uzávěr $R$ v jeho podílovém nadtělese $Q(R)$. Řekneme, že $R$ je \emph{celistvě uzavřený}, pokud $\icl{R} = R$.
\end{definice}

\begin{tvrzeni}	\label{thm:KX_integral}
	Nechť $K$ je těleso charakteristiky $p > 0$, $X$ konečně mnoho neurčitých. Pak $K[[X]]$ je celistvé rozšíření $K^p[[X]]$.

	\begin{proof}
		Vol $f \in K[[X]]$. Dle důsledku $\ref{thm:p_power}$ platí $f^p \in K^p[[X]]$, tedy $x^p - f^p$ je monický polynom s koeficienty v $K^p[[X]]$, jehož je $f$ kořenem.
	\end{proof}
\end{tvrzeni}

\begin{veta}[Auslander-Buchsbaum-Nagata]
	Každý regulární lokální okruh je Gaussův obor.
\end{veta}

\begin{veta}[fakt]
	Každý pseudo-Bézoutův obor (a tím spíše Gaussův obor) je celistvě uzavřen.
\end{veta}

\begin{dusledek} \label{thm:icl}
	Buď $T \subseteq K$ algebraické rozšíření těles, $X$ konečně mnoho neurčitých. Pak obory $T[[X]] \subseteq T[[X]][K] \subseteq K[[X]]$ jsou celistvě uzavřené.
\end{dusledek}


\section{Hlavní příklad}

Následující definované struktury budeme volně používat po zbytek textu se stejným značením/pojmenováním.

\begin{definice} \hfill
	\newcommand*{\I}{_{i = 0}^\infty}
	\newcommand*{\PI}{_{i = 0}^{k - 1}}

	\begin{items}
		\item $K_0$ buď perfektní těleso charakteristiky $p > 0$, tj.\ $K_0^p = K_0$.
		\item $K = Q(K_0[X])$, kde $X = \{X_i: i \in \N_0\}$ jsou neurčité.
		\item $A = K[[Z_1, Z_2, Z_3]]$, kde $Z_1, Z_2, Z_3$ jsou neurčité.
		\item $B = K^p[[Z_1, Z_2, Z_3]][K]$
		\item $c_j = \sum\I X_{2(i + j)} Z_3^i$,\quad $d_j = \sum\I X_{2(i + j) + 1} Z_3^i$,\quad $b_j = Z_1 c_j + Z_2 d_j$,\quad $b = b_0$.
		\item $P_k(c_j) = \sum\PI X_{2(i + j)} Z_3^i$, $P_k(d_j) = \sum\PI X_{2(i + j) + 1} Z_3^i$, $P_k(b_j) = Z_1 P_k(c_j) + Z_2 P_k(d_j)$, což jsou počáteční úseky řad $b_j$, $c_j$, $d_j$ vzhledem k proměnné $Z_3$.
		\item $C = \icl{B[b]}$.
		\item $\tilde{B} = B[b_i: i \in \N_0]$.
	\end{items}
\end{definice}

Ukážeme, že $B[b]$ je hledaný noetherovský obor, jehož celistvý uzávěr $C$ je roven $\tilde{B} = B[b_i: i \in \N_0]$, což není noetherovský obor.

\begin{pozorovani}
	Použitím obecných tvrzení z minulé kapitoly obdržíme následující vlastnosti definovaných struktur:
	\begin{items}
		\item $A$, $B$ jsou regulární lokální obory Krullovy dimenze $3$, a jsou tedy celistvě uzavřeny (\ref{thm:reg}, \ref{thm:icl}).
		\item Obor $A$ je celistvý nad $B$ (\ref{thm:KX_integral}).
		\item $B[b]$ je lokální noetherovský obor (\ref{thm:noe}, \ref{thm:R_loc}).
	\end{items}
\end{pozorovani}

\begin{pozorovani} Přímo z definice plynou následující vztahy mezi řadami $b_k$, $c_k$, $d_k$.
	\begin{items}
		\item $c_k = X_{2k} + Z_3 c_{k + 1}$,\quad $d_k = X_{2k + 1} + Z_3 d_{k + 1}$,\quad $b_k = Z_1 X_{2k} + Z_2 X_{2k + 1} + Z_3 b_{k + 1}$.
		\item $c_k = P_j(c_{k + j}) + Z_3^j c_{k + j}$,\quad $d_k = P_j(d_{k + j}) + Z_3^j d_{k + j}$,\quad $b_k = P_j(b_{k + j}) + Z_3^j b_{k +j}$.
		\item $B[b_i: i \leq n] = B[b_n]$, neboť z předchozího plyne $b_i \in B[b_{i + 1}]$.
	\end{items}
\end{pozorovani}

\begin{tvrzeni}
	Platí inkluze $\tilde{B} \subseteq C$.

	\begin{proof}
		Z předchozího pozorování vidíme, že $b_{k + 1} = \frac{b_k - Z_1 X_{2k} - Z_2 X_{2k + 1}}{Z_3} \in Q(B[b_k])$. Indukcí tedy dostáváme $b_k \in Q(B[b])$ pro každé $k \in \N_0$.

		Protože je celé $A$ celistvé nad $B$, a tedy i nad $B[b]$, je každé $b_k \in Q(B[b]) \cap A \subseteq C$, což dokazuje požadované.
	\end{proof}
\end{tvrzeni}

\begin{dusledek}
	Neplatí $M_{B[b]} = Z_1 B[b] + Z_2 B[b] + Z_3 B[b]$, jinak by totiž $B[b]$ byl dle \ref{thm:R_reg} regulární lokální obor, a tedy celistvě uzavřený, to je ale ve sporu s předchozím tvrzením.
\end{dusledek}

\begin{lemma}[VII.7, jen strana nepožadující ($p = 2$)]
	Platí $A \cap \bigcup_{k = 0}^\infty \frac{B b + b}{Z_3^k} \subseteq \tilde{B}$.

	\begin{proof}
		Nechť $a = \frac{\beta_1 + \beta_2 b}{Z_3r^l} \in A$, pro nějaká $\beta_1, \beta_2 \in B$. Označme $\beth_l$ prvních $l$ členů mocninného rozvoje $Z_3$ v $b$:
		\mld{
			b = b_0 = \underbrace{\sum_{i = 0}^{l - 1} (X_{2i} Z_1 + X_{2i+1} Z_2) Z_3^i}_{\beth_l} + Z_3^l b_l.
		}

		Potom z $b = \beth_l + Z_3^l\,b_l$ a $\beta_1 + \beta_2\,b = Z_3^l a$ platí, že $Z_3^l a - Z_3^l\,\beta_2\,b_l = \beta_1 + \beta_2\,\beth_l$. Protože $\beta_1 + \beta_2\,\beth_l \in B$, je i $Z_3^l (a - \beta_2\,b_l)
		\in B$. Libovolná řada náleží $B$ pouze v závislosti na jejích koeficientech u $Z_1, Z_2, Z_3$. Proto pokud $Z_3 v \in B$, pak už i $v \in B$. A tedy i $(a - \beta_2\,b_l) \in B$.

		Z toho dostáváme požadované $a = (a - \beta_2\,b_l) + \beta_2\,b_l \in B + B b_l \subseteq B[b_l] \subseteq \tilde{B}$.
	\end{proof}
\end{lemma}

\begin{tvrzeni}
	Platí následující inkluze: $C \subseteq \tilde{B}$, resp.\ $\tilde{B}$ je celistvě uzavřený obor.

	\begin{proof}
		\todo
	\end{proof}
\end{tvrzeni}

\begin{tvrzeni}
	Obor $\tilde{B}$ není noetherovský.
	
	\begin{proof}
		\newcommand*{\I}{_{i = 0}^{n - 1}}

		Dokážeme, že posloupnost ideálů $I_n = \sum\I b_i \tilde{B}$ je ostře rostoucí. K tomu stačí ukázat, že $b_n \notin I_{n - 1}$, $\forall n \in \N$.

		Budeme postupovat sporem. Ať $n \in \N$ je nejmenší, že $b_n \in I_{n - 1}$. Existuje tedy posloupnost $\{f_i\}\I \subseteq \tilde{B}$ tak, že $b_n = \sum\I f_i b_i$. Označme $\p M$ množinu všech \uv{monických monočlenů v neurčitých $\{b_i: i \in \N_0\}$}, jako bychom se na $\tilde{B}$ dívali jako na polynomy v těchto neurčitých s koeficienty v $B$. Každý $f_i \in \tilde{B}$, a proto je lze vyjádřit jako $f_i = \sum_{m \in \p M} f_{i, m} m$, pro $f_{i, m} \in B$, kde pouze konečně mnoho $f_{i, m}$ je nenulových.

		Potom
		\mld{
			b_n = \sum\I \sum_{m \in \p M} f_{i, m} m b_i = \sum\I \sum_{m \in \p M \setminus \{1\}} f_{i, m} m b_i + \sum\I f_{i, 1} b_i.
		}
		A protože $f_{i, 1} = g_i + Z_1 g_{i, 1} + Z_2 g_{i, 2}$ pro nějaká $g_i \in K^p[[Z_3]][K]$ a $g_{i, 1}, g_{i, 2} \in B$. Dostáváme
		\mld{
			b_n = \underbrace{\sum\I \sum_{m \in \p M \setminus \{1\}} f_{i, m} m b_i + \sum\I \left(Z_1 g_{i, 1} + Z_2 g_{i, 2}\right)b_i}_{\in Z_1^2 A + Z_2^2 A + Z_1 Z_2 A} + \sum\I g_i b_i.
		}
		První dvě sumy obsahují ve svých členech vždy některé z $Z_1^2, Z_2^2$ nebo
		$Z_1 Z_2$, ale $b_{n + 1}$ neobsahuje žádné z nich. Proto $b_{n + 1} = \sum\I g_i b_i$. Dále máme
		\mld{
			Z_1 c_n + Z_2 d_n = b_n = \sum\I g_i b_i = Z_1 \sum\I g_i c_i + Z_2 \sum\I g_i d_i
		}
		Tedy
		\mld{
			c_n = \sum\I g_i c_i \quad \& \quad d_n = \sum\I g_i d_i,
		}
		protože $g_i \in K^p[[Z_3]][K]$.

		První rovnost (resp.\ obě, postup je analogický) dovedeme ke sporu. Podívejme se na tuto rovnost jako na rovnost řad v proměnné $Z_3$. Označíme-li $g_i = \sum_{j = 0}^\infty g_{i, j} Z_3^j$, $g_{i, j} \in K$, dostaneme porovnáním koeficientů u $Z_3^k$ následující rovnost:
		\mld{
			X_{2(n + k)} = \sum\I \sum_{j = 0}^k g_{i, j} X_{2(i + k - 1)}.
		}

		Z algebraické nezávislosti neurčitých $\{X_i: i \in \N_0\}$ dostáváme, že některý z koeficientů $g_{i, j}$ obsahuje neurčitou $X_{2(n + k)}$ v mocnině nesoudělné s $p$. Tento výrok označíme jako $o(g_{i, j}, X_{2(n + k)})$. Dostáváme tedy
		\ml{
			(\forall k \in \N_0) (\exists i_k < n) (\exists j_k \leq k): o(g_{i_k, j_k}, X_{2(n + k)}).
		}
		Potom ale existuje $i$, že $g_i$ obsahuje ve svých koeficientech nekonečně mnoho neurčitých $X$. Každý koeficient je přitom z podílového tělesa polynomů v neurčitých $X$, tedy obsahuje jich konečně mnoho. Celkem existuje nekonečně mnoho koeficientů řady $g_i$, že každý z nich obsahuje jinou neurčitou $X$ v mocnině nesoudělné s $p$. Potom ale tyto koeficienty nemohou tvořit rozšíření tělesa $K^p$ konečného stupně, a tedy $g_i \notin K^p[[Z_3]][K]$ (viz \ref{thm:TXK}), což je spor.
	\end{proof}
\end{tvrzeni}

\begin{veta}
	Obor $B[b]$ je příkladem noetherovského oboru, jehož celistvý uzávěr v podílovém nadtělese není noetherovský.
	
	\begin{proof}
		Plyne z předchozích tvrzení.
	\end{proof}
\end{veta}


\end{document}
