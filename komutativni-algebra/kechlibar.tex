\documentclass[11pt,a4paper]{article}
\usepackage[utf8]{inputenc}
\usepackage[czech]{babel}
\usepackage{amsfonts,amsthm,amsfonts,amssymb,amsmath}
\usepackage{a4wide}
\usepackage{enumerate}
\usepackage[IL2]{fontenc}

\newcommand\m[1]{\mathbb { #1 }} % tucne pismeno/text
\newcommand\p[1]{\mathcal{ #1 }} % psaci pismeno/text
\newcommand\IFF{\ensuremath{\iff}}
\newcommand\N{\m N}

\newcommand*{\todo}[1]{\textbf{TODO} #1}

\newcommand*{\ml}[1]{\[\textstyle\let\displaystyle\textstyle#1\]}	% math line in textstyle
\newcommand*{\mld}[1]{\[#1\]} % math line in displaystyle

\DeclareMathOperator{\alt}{alt} % Krullova dimenze
\DeclareMathOperator{\gen}{gen} % nejmenší počet generátorů
\let \icl \overline % celistvý uzávěr

\newenvironment{items}{% nečíslovaný seznam -- zde možno nastavit jednotný styl
	\itemize
	\itemsep = 0pt%
}{%
	\enditemize
}
\newenvironment{itemsn}{% číslovaný seznam
	\enumerate%
	\itemsep = 0pt%
}{%
	\endenumerate%
}

\newcounter{numb}

\theoremstyle{definition}
\newtheorem*{definice}{Definice}
\newtheorem{pozorovani}[numb]{Pozorování}
\newtheorem{poznamka}[numb]{Poznámka}

\theoremstyle{plain}
\newtheorem{veta}[numb]{Věta}
\newtheorem{lemma}[numb]{Lemma}
\newtheorem{tvrzeni}[numb]{Tvrzení}
\newtheorem{dusledek}[numb]{Důsledek}

\title{Konstrukce noetherovského okruhu, jehož celistvý uzávěr\\ není noetherovský}
\author{Adam Bartoš, Tomáš Jakl, Jan Kosina, Martin Raška, Jiří Vančura}
\date{18. července 2012}


\begin{document}
\maketitle


\section{Formální mocninné řady nad tělesem a odvozené obory}

V této kapitole budeme zkoumat vlastnosti oborů odvozených od oboru formálních mocninných řad nad tělesem v konečně mnoha neurčitých. Zajímat nás bude noetherovskost, lokalita, regularita a celistvá uzavřenost. Na základě zde prezentovaných obecných poznatků bude vybudován hledaný příklad noetherovského okruhu, jehož celistvý uzávěr není noetherovský.

V našem textu budeme pracovat pouze s komutativními okruhy.

\begin{definice}
	Nechť $R$ je okruh, $X$ množina neurčitých, pak $R[X]$ značí okruh polynomů nad $R$ v neurčitých $X$ a $R[[X]]$ značí okruh formálních mocninných řad nad $R$ v neurčitých $X$.
	
	Nechť $m$ je monočlen v neurčitých $X$, pak jeho stupeň, $\deg m$, je součet mocnin neurčitých, které se v monočlenu vyskytují.
\end{definice}

\begin{pozorovani}
	Je-li $K$ těleso, $X$ množina neurčitých, pak $K[[X]]$ je obor integrity, neboť vlastnost \uv{býti oborem} se zřejmě přenáší z výchozího okruhu na okruh formálních mocninných řad.
\end{pozorovani}

\begin{pozorovani} \label{thm:TXK}
	\newcommand*{\I}{_{i = 1}^n}

	Nechť $T \subseteq K$ jsou tělesa, $X$ je množina neurčitých. Uvažme obor $T[[X]][K] \subseteq K[[X]]$. Řada $f \in K[[X]]$ je dle definice prvkem $T[[X]][K]$ právě tehdy, když existují $\{f_i\}\I \subseteq T[[X]]$ a $\{k_i\}\I \subseteq K$, že $f = \sum\I f_i k_i$. Konečnou množinu prvků $k_i$ označme jako $K_f$.

	Vidíme, že koeficienty řady $f$ leží v nadtělese $T(K_f)$. Je-li množina $K_f$ tvořena prvky algebraickými nad $T$, pak $T(K_f)$ je algebraické rozšíření konečného stupně a takové je i rozšíření generované koeficienty řady $f$.

	Naopak, generují-li koeficienty řady $f \in K[[X]]$ tělesové rozšíření $T$ konečného stupně, konečná množina $K_f$ existuje, totiž konečná báze příslušného rozšíření nad $T$, a tedy $f \in T[[X]][K]$.

	Dále, je-li $F \subseteq K$ konečná množina prvků algebraických nad $T$, pak platí
	\ml{
		T[[X]][F] \subseteq T[F][[X]] \subseteq T(F)[[X]] \subseteq T[[X]][K].
	}
\end{pozorovani}


\subsection{Noetherovskost}

\begin{veta}[Zobecněná Hilbertova věta o bázi] \label{thm:GHB}
	Buď $R$ okruh, $X$ neurčitá, pak následující je ekvivalentní.
	\begin{itemsn}
		\item $R$ je noetherovský.
		\item $R[X]$ je noetherovský.
		\item $R[[X]]$ je noetherovský.
	\end{itemsn}
\end{veta}

\begin{tvrzeni} \label{thm:KX_noe}
	Nechť $K$ je těleso, $X$ konečně mnoho neurčitých. Pak $K[[X]]$ je noetherovský obor.

	\begin{proof}
		Těleso $K$ je jistě noetherovský okruh. Postupným použitím zobecněné Hilbertovy věty o bázi (\ref{thm:GHB}) na jednotlivé neurčité, kterých je konečně mnoho, dostaneme požadované.
	\end{proof}
\end{tvrzeni}

\begin{lemma}
	Buď $S$ noetherovský okruh a buď $R \subseteq S$ jeho podokruh. Pokud pro každý konečně generovaný ideál $I \subseteq R$ platí, že $IS \cap R = I$, pak $R$ je noetherovský.

	\begin{proof}
		Budeme postupovat sporem. Ať $R$ není noetherovský okruh, pak tedy existuje ideál $I \subseteq R$, který není konečně generován. Dále existuje nekonečná posloupnost ideálů $\{I_i: i \in \N\}$ tak, že:
		\ml{
			I_1 \subsetneq I_2 \subsetneq \dots \subseteq I.
		}
		Potom, ale pro posloupnost ideálů $\{I_i S: i \in \N\}$ v $S$ z noetherovskosti existuje $n \in \N$ tak, že $I_n S = I_{n + 1} S$. Podle předpokladu je ale $I_n = I_n S \cap R = I_{n + 1} S \cap R = I_{n + 1}$, ale $I_n \not= I_{n+1}$, což je spor.
	\end{proof}
\end{lemma}

\begin{poznamka}
	Inkluze $I \subseteq IS \cap R$ z předpokladu předchozího lemmatu je zřejmě splněna vždy.
\end{poznamka}

\begin{tvrzeni}
	Nechť $T \subseteq K$ je algebraické rozšíření těles, $X$ konečně mnoho neurčitých. Pak obor $T[[X]][K]$ je noetherovský.

	\begin{proof}
		\newcommand*\I{_{i = 1}^n}

		Označme $R = T[[X]][K]$, $S = K[[X]]$. $S$ je dle tvrzení \ref{thm:KX_noe} noetherovský obor. K použití předchozího lemmatu stačí ukázat, že pro každý konečně generovaný ideál $I \subseteq R$ platí, že $IS \cap R \subseteq I$.

		Buď $I = \sum\I r_i R$ konečně generovaný ideál v $R$. Vol $r \in IS \cap R$. Pak
		\ml{
			r \in IS = \sum\I r_i RS = \sum\I r_i S.
		}
		Tedy $r = \sum\I r_i f_i$ pro nějaké řady $f_i \in S$.

		Uvažme $F = K_r \cup \bigcup\I K_{r_i}$ a $U = T(F)$. $F$ je konečná množina prvků $K$, které je algebraické nad $T$, a $U$ je tedy rozšíření $T$ konečného lineárního stupně. Dle pozorování \ref{thm:TXK} máme $r, r_i \in U[[X]] \subseteq R$, $i = 1, \hdots, n$.

		Dále uvažme rozklad $K = U \oplus K'$. Potom i $K[[X]] = U[[X]] \oplus K'[[X]]$. Rozložme $f_i = g_i + h_i \in U \oplus K'$. Potom máme
		\ml{
			% U[[X]] \owns r = \sum\I r_i g_i + \sum\I r_i h_i \in U[[X]] \oplus K'[[X]].
			\underbrace{r}_{\in U[[X]]} = \underbrace{\sum\I r_i g_i}_{\in U[[X]]} + \underbrace{\sum\I r_i h_i}_{\in K'[[X]]}
		}
		Tedy $\sum\I r_i h_i = 0$ a $r = \sum\I r_i g_i \in \sum\I r_i U[[X]] \subseteq \sum\I r_i R = I$.
	\end{proof}
\end{tvrzeni}

\begin{tvrzeni} \label{thm:RF_noe}
	Nechť $R \subseteq S$ jsou okruhy, $R$ je noetherovský, $S = R[F]$ pro nějakou konečnou $F \subseteq S$. Pak $S$ je také noetherovský.

	\begin{proof}
		Nechť $F = \{s_1, \hdots, s_n\}$. Pak
		\ml{
			S \simeq R[x_1, \hdots, x_n] / \{f \in R[x_1, \hdots, x_n]: f(s_1, \hdots, s_n) = 0\},
		}
		což je dle Hilbertovy věty o bázi (\ref{thm:GHB}) faktor noetherovského okruhu, tedy noetherovský okruh.
	\end{proof}
\end{tvrzeni}

\begin{dusledek} \label{thm:noe}
	Nechť $T \subseteq K$ je algebraické rozšíření těles, $X$ konečně mnoho neurčitých, $F \subseteq K[[X]]$ konečná. Pak obory $T[[X]][F]$ a $T[[X]][K][F]$ jsou noetherovské.
\end{dusledek}


\subsection{Lokalita}

\begin{definice}
	Řekneme, že okruh $R$ je \emph{lokální}, existuje-li v něm jediný maximální ideál. Tento budeme značit $M_R$.
\end{definice}

\begin{lemma} \label{thm:KX_inv}
	Nechť $K$ je těleso, $X$ libovolně mnoho neurčitých. Pak v $K[[X]]$ jsou invertibilní právě řady s nenulovým absolutním členem.

	\begin{proof}
		Zvolme $f \in K[[X]]$ s nenulovým absolutním členem. Označme $\p M$ množinu všech monických monočlenů v neurčitých $X$. Pak můžeme psát $f = \sum_{m \in \p M} f_m m$. Bez újmy na obecnosti, absolutní člen $f_1 = 1$. Hledáme $g = \sum_{m \in \p M} g_m m \in K[[X]]$, že $f g = 1$. Chceme, aby $f_1 g_1 = 1$, tedy $g_1 = 1$. Dále pro $m \in \p M \setminus \{1\}$ chceme následující:
		\ml{
			0 = (f g)_m = \sum_{\substack{n, n' \in \p M\\ n n' = m}} f_{n'} g_n = f_1 g_m + \sum_{\substack{n, n' \in \p M\\ n n' = m\\ n \not= m}} f_{n'} g_n = g_m + \sum_{\substack{n, n' \in \p M\\ n n' = m\\ \deg n < \deg m}} f_{n'} g_n.
		}
		Takže
		\begin{equation*} \label{eq:inv}
			\textstyle
			g_m = -\sum_{\substack{n, n' \in \p M\\ n n' = m\\ \deg n < \deg m}} f_{n'} g_n \tag{$\ast$}.
		\end{equation*}
		Máme předpis pro $g_m$, které závisí pouze na $g_n$ pro $\deg n < \deg m$, tedy induktivní konstrukce dává výsledek.

		Naopak, má-li $f$ nulový absolutní člen, pak každý jeho násobek má rovněž nulový absolutní člen. Tedy $f$ není invertibilní.
	\end{proof}
\end{lemma}

\begin{tvrzeni}
	Nechť $K$ je těleso, $X$ konečně mnoho neurčitých. Pak $K[[X]]$ je lokální obor s $M_{K[[X]]} = \sum_{x \in X} x K[[X]]$.

	\begin{proof}
		Dle předchozího lemmatu jsou neinvertibilní prvky v $K[[X]]$ právě ty s nulovým absolutním členem. Tyto pak zřejmě tvoří jediný maximální ideál, který je roven $\sum_{x \in X} x K[[X]]$, protože každý monočlen obsahuje nějakou neurčitou $x$, a tedy ho můžeme přidat do členu rozkladu $x K[[X]]$.
	\end{proof}
\end{tvrzeni}

\begin{tvrzeni}
	Nechť $T \subseteq K$ je algebraické rozšíření těles, $X$ konečně mnoho neurčitých. Pak v $R = T[[X]][K]$ jsou invertibilní právě prvky s nenulovým absolutním členem a $R$ je lokální obor s $M_R = \sum_{x \in X} xR$.

	\begin{proof}
		Víme, že tato charakteristika invertibilních prvků platí v celém $K[[X]]$. Je-li $f \in R$ s nenulovým absolutním členem, pak dle lemmatu \ref{thm:KX_inv} jeho invers v $K[[X]]$ získáme vztahem \eqref{eq:inv}. Fixujeme-li konečnou množinu algebraických prvků $K_f$, že $f \in T(K_f)[[X]]$, pak i invers $g \in T(K_f)[[X]] \subseteq T[[X]][K]$, tedy získaný invers zůstane v $R$.

		Neinvertibilní prvky, což jsou právě prvky s nulovým absolutním členem, tvoří jediný maximální ideál. Tento je roven $\sum_{x \in X} x R$ dle důsledku \ref{thm:TXK_absOK}.
	\end{proof}
\end{tvrzeni}

Nyní od obecné situace algebraického rozšíření těles $T \subseteq K$ přejdeme k situaci, kdy $K$ má kladnou charakteristiku $p$ a $T = K^p$. Jedná se o speciální případ, protože $K$ je zřejmě algebraické nad $K^p$.

\begin{lemma}
	Nechť $R$ je okruh prvočíselné charakteristiky $p$, $X$ jedna neurčitá, pak pro $f = \sum_{i = 0}^\infty f_i X^i \in R[[X]]$ platí $f^p = \sum_{i = 0}^\infty f_i^p X^{ip}$. Tedy mocnění na $p$-tou probíhá člen po členu.

	\begin{proof}
		Máme $f = f_0 + (f - f_0) = f_0 + X g$, pro nějaké $g \in R[[X]]$. Pak
		\mld{
			f^p = \sum_{i = 0}^p \binom{p}{i} f_0^i (X g)^{p - i} = f_0^p + X^p g^p,
		}
		protože ostatní binomické koeficienty jsou dělitelné $p$, a tedy nulové. Dál můžeme pokračovat indukcí.
	\end{proof}
\end{lemma}

\begin{dusledek} \label{thm:p_power}
	Nechť $R$ je okruh prvočíselné charakteristiky $p$, $X$ konečně mnoho neurčitých. Pak i v $R[[X]]$ probíhá mocnění na $p$-tou člen po členu.

	\begin{proof}
		Stačí použít předchozí lemma postupně pro přibývající neurčité. Charakteristika $p$ se zřejmě přenáší i do okruhu formálních mocninných řad.
	\end{proof}
\end{dusledek}

\begin{tvrzeni} \label{thm:R_loc}
	Nechť $K$ je těleso charakteristiky $p > 0$, $X$ konečně mnoho neurčitých, $R$ meziobor: $K^p[[X]] \subseteq R \subseteq K[[X]]$. Pak $f \in R$ je invertibilní, právě když má nenulový absolutní člen, a $R$ je lokální obor s $M_R = R \cap M_{K[[X]]} \supseteq \sum_{x \in X} x R$.

	\begin{proof}
		Nechť $f \in R$ má nenulový absolutní člen. Pak dle předchozího důsledku $f^p \in K^p[[X]]$ a má rovněž nenulový absolutní člen. Dle lemmatu \ref{thm:KX_inv} existuje $g \in K^p[[X]]$, že $f^p g = 1$. Potom ale $f f^{p - 1} g = 1$, a protože $f^{p - 1} g \in R$, je $f$ invertibilní v $R$.

		Naopak řady s nulovým absolutním členem tvoří vlastní ideál, tedy nemohou být invertibilní. Tento vlastní ideál je díky předchozímu odstavci jediný maximální, a tedy $R$ je lokální. Zbytek tvrzení je zřejmý.
	\end{proof}
\end{tvrzeni}


\subsection{Řady s nulovým absolutním členem}

V minulé podkapitole jsme ukázali, že řady s nulovým absolutním členem tvoří často jediný maximální ideál příslušného oboru formálních mocninných řad, který je díky tomu lokální. V této podkapitole se budeme zabývat otázkou, kdy jsou řady okruhu $R$ s nenulovým absolutním členem právě prvky ideálu $\sum_{x \in X} xR$. Za takové situace totiž snadno dokážeme regularitu lokálního okruhu (viz další podkapitola).

\newcommand*{\absOK}[1]{řady $#1$ bez absolutního členu jsou generovány neurčitými}
\begin{definice}
	Nechť $K$ je těleso, $X$ neurčité, $R \subseteq K[[X]]$. Řekneme, že \emph{\absOK{R}}, pokud je každá taková řada prvkem ideálu $\sum_{x \in X} xR$.
\end{definice}

\begin{pozorovani}
	Nechť $K$ je těleso, $X$ konečně mnoho neurčitých. Pak každé $f \in K[[X]]$ můžeme psát ve tvaru $f = f_1 + \sum_{x \in X} x f_x$, kde $f_1$ je absolutní člen $f$, $f_x \in K[[X]]$. Z toho ihned plyne, že \absOK{K[[X]]}.
\end{pozorovani}

\begin{tvrzeni} \label{thm:RF_absOK}
	Nechť $K$ je těleso, $X$ konečně mnoho neurčitých, $R$ podobor $K[[X]]$ takový, že \absOK{R}, $F$ libovolná podmnožina $K$. Potom \absOK{R[F]}.
	
	\begin{proof}
		\newcommand{\I}{_{i = 1}^n}
		
		Nechť $f \in R[F]$ je řada bez absolutního členu. Pak $f = \sum\I g_i k_i$, $g_i \in R$, $k_i$ jsou prvky podoboru $K$ generovaného množinou $F$. Dle předpokladu každé $g_i = g_{i, 1} + \sum_{x \in X} x g_{i, x}$, kde $g_{i, x} \in R$. Potom $f = \sum\I g_{i, 1} k_i + \sum_{x \in X} x (\sum\I g_{i, x} k_i)$. Přitom absolutní člen $f_1 = \sum\I g_{i, 1} k_i = 0$ a všechna $g_{i, x} k_i \in R[F]$.
	\end{proof}
\end{tvrzeni}

\begin{dusledek} \label{thm:TXK_absOK}
	Nechť $T \subseteq K$ je algebraické rozšíření těles, $X$ konečně mnoho neurčitých. Pak v oboru $T[[X]][K]$ jsou řady bez absolutního členu generovány neurčitými.
\end{dusledek}

\begin{definice}
	Nechť $U$ je libovolná podmnožina nějakého okruhu, definujme
	\begin{items}
		\item $\p M(U) = \{\prod_{u \in U} u^{k_u}: k_u \in \N_0, \{u: k_u > 0\} \text{ je konečná}\}$,
		\item $\p M_p(U) = \{\prod_{u \in U} u^{k_u}: k_u < p, \{u: k_u > 0\} \text{ je konečná}\}$.
	\end{items}
	Jedná se o množiny monických monočlenů v $U$. Ve druhém případě omezené stupněm $p$, což se hodí v charakteristice $p$. Poznamenejme, že $U$ nemusí být neurčité, jde o libovolné prvky okruhu.
	
	Je-li $K$ těleso, $X$ neurčité, $R \subseteq K[[X]]$, pak $R_1$ označme všechny absolutní členy $R$. Je-li $R$ podobor, pak $R_1$ zřejmě tvoří obor integrity.
\end{definice}

\begin{tvrzeni}
	Nechť $K$ je těleso, $X$ konečně mnoho neurčitých, $R \subseteq K[[X]]$ podobor, ve kterém jsou řady bez absolutního členu generovány neurčitými. Nechť $F \subseteq K[[X]]$ je množina řad taková, že $\p M(F_1)$ je lineárně nezávislá nad $R_1$ (což odpovídá algebraické nezávislosti $F_1$ nad $R_1$). Pak řady $R[F]$ bez absolutního členu jsou generovány neurčitými.
	
	\begin{proof}
		\newcommand*{\I}{_{i = 1}^n}
		
		Nechť $f \in R[F]$ má nulový absolutní člen. Pišme $f = \sum\I g_i h_i$, kde $g_i \in R$ a $h_i \in \p M(F)$. Dle předpokladu můžeme psát $g_i = g_{i, 1} + \sum_{x \in X} x g_{i, x}$, kde $g_{i, x} \in R$. Potom máme
		\mld{
			f = \sum\I g_i h_i = \sum\I g_{i, 1} h_i + \sum_{x \in X} x (\sum\I g_{i, x} h_i).
		}
		Přitom absolutní člen $f_1 = \sum\I g_{i, 1} h_{i, 1} = 0$, $g_{i, 1} \in R_1$, $h_{i, 1} \in \p M(F_1)$. Z předpokladu lineární nezávislosti jsou všechna $g_{i, 1} = 0$, a tedy $f \in \sum_{x \in X} x R[F]$. 
	\end{proof}
\end{tvrzeni}

\begin{tvrzeni} \label{thm:RpF_absOK}
	Nechť $K$ je těleso charakteristiky $p > 0$, $X$ konečně mnoho neurčitých, $R$ meziobor: $K^p[[X]] \subseteq R \subseteq K[[X]]$, jehož řady bez absolutního členu jsou generovány neurčitými, $F \subseteq K[[X]]$ libovolná podmnožina taková, že $\p M_p(F_1)$ je lineárně nezávislá nad $R_1$. Potom řady bez absolutního členu v $R[F]$ jsou generovány neurčitými.
	
	\begin{proof}
		Nechť $f \in R[F]$ je řada bez absolutního členu. Lze psát $f = \sum_{i = 1}^n g_i h_i$, kde $g_i \in R$, $h_i \in \p M(F)$. Protože pro každé $h \in F$ platí $h^p \in K^p[[X]] \subseteq R$ (důsledek \ref{thm:p_power}), můžeme dokonce uvažovat, že $h_i \in \p M_p(F)$. Dále postupujeme jako v důkazu předchozího tvrzení.
	\end{proof}
\end{tvrzeni}
	

\subsection{Regularita}

\begin{definice}
	Je-li $R$ okruh, pak \emph{Krullovu dimenzi} definujeme jako
	\ml{
		\alt R = \sup\{n: P_0 \subsetneq \cdots \subsetneq P_n \text{ je ostře rostoucí posloupnost prvoideálů v } R\}.
	}

	Dále pro $I \subseteq R$ ideál definujme
	\ml{
		\gen_R I = \min\{|G|: G \subseteq R, G \text{ generuje } I\},
	}
	tedy nejmenší počet generátorů $I$ v $R$. Pokud je okruh $R$ zřejmý z kontextu, píšeme pouze $\gen I$.
\end{definice}

\begin{definice}
	Nechť $R$ je okruh. Řekneme, že $R$ je \emph{regulární lokální okruh}, je-li lokální, neotherovský a platí $\alt M_R = \gen M_R$.
\end{definice}

\begin{poznamka}
	Lze ukázat, že regulární lokální okruh už je oborem integrity.
\end{poznamka}

\begin{veta}[Věta o Krullově dimenzi]
	Buď $R$ lokální noetherovský okruh. Platí
	\ml{
		\alt R \leq \gen M_R.
	}
\end{veta}

\begin{tvrzeni} \label{thm:R_reg}
	Buď $K$ těleso, $X = \{X_1, \hdots, X_n\}$ konečně mnoho neurčitých, $R$ noetherovský podobor $K[[X]]$ takový, že invertibilní prvky jsou právě řady s nenulovým absolutním členem a řady s nulovým absolutním členem jsou generovány neurčitými. Pak $R$ je regulární lokální obor Krullovy dimenze $n$.

	\begin{proof}
		Dle předpokladů je $R$ lokální noetherovský obor.
		\begin{items}
			\item $M_R = \sum_{i = 1}^n X_i R$, tedy $\gen M_r \leq n$.
			\item Dle věty o Krullově dimenzi $\alt R \leq \gen M_R$.
			\item $\{\sum_{i = 1}^k X_i R\}_{k = 0}^n$ je ostře rostoucí řetězec prvoideálů, protože obsahuje-li každý člen součinu řad alespoň jednu z neurčitých $X_1, \hdots, X_k$, musí alespoň jednu z nich obsahovat i každý člen obou řad (jinak součin členů nejmenších stupňů, které neobsahují žádnou takovou neurčitou, také žádnou z nich neobsahuje). Tedy $n \leq \alt M_R$.
		\end{items}
		Celkem $n \leq \alt M_R \leq \gen M_R \leq n$.
	\end{proof}
\end{tvrzeni}

\begin{dusledek} \label{thm:reg}
	Buď $T \subseteq K$ algebraické rozšíření těles, $X$ konečně mnoho $(n)$ neurčitých. Pak $T[[X]] \subseteq T[[X]][K] \subseteq K[[X]]$ jsou regulární lokální obory (dimenze $n$).
\end{dusledek}


\subsection{Celistvá uzavřenost}

\begin{definice}
	Nechť $R$ je obor integrity. Pak $\icl{R}$ značí celistvý uzávěr $R$ v jeho podílovém nadtělese $Q(R)$. Řekneme, že $R$ je \emph{celistvě uzavřený}, pokud $\icl{R} = R$.
\end{definice}

\begin{tvrzeni}	\label{thm:KX_integral}
	Nechť $K$ je těleso charakteristiky $p > 0$, $X$ konečně mnoho neurčitých. Pak $K[[X]]$ je celistvé rozšíření $K^p[[X]]$.

	\begin{proof}
		Vol $f \in K[[X]]$. Dle důsledku $\ref{thm:p_power}$ platí $f^p \in K^p[[X]]$, tedy $x^p - f^p$ je monický polynom s koeficienty v $K^p[[X]]$, jehož je $f$ kořenem.
	\end{proof}
\end{tvrzeni}

\begin{veta}[Auslander-Buchsbaum-Nagata]
	Každý regulární lokální okruh je Gaussův obor.
\end{veta}

\begin{veta}[fakt]
	Každý pseudo-Bézoutův obor (a tím spíše Gaussův obor) je celistvě uzavřen.
\end{veta}

\begin{dusledek} \label{thm:icl}
	Buď $T \subseteq K$ algebraické rozšíření těles, $X$ konečně mnoho neurčitých. Pak obory $T[[X]] \subseteq T[[X]][K] \subseteq K[[X]]$ jsou celistvě uzavřené.
\end{dusledek}


\section{Hlavní příklad}

Následující definované struktury budeme volně používat po zbytek textu se stejným značením/pojmenováním.

\begin{definice} \hfill
	\newcommand*{\I}{_{i = 0}^\infty}
	\newcommand*{\PI}{_{i = 0}^{k - 1}}

	\begin{items}
		\item $K_0$ buď perfektní těleso charakteristiky $p > 0$, tj.\ $K_0^p = K_0$.
		\item $K = Q(K_0[X])$, kde $X = \{X_i: i \in \N_0\}$ jsou různé neurčité.
		\item $A = K[[Z]]$, kde $Z = \{Z_1, Z_2, Z_3\}$ jsou různé neurčité (a různé od neurčitých v $X$).
		\item $B = K^p[[Z]][K]$
		\item $c_j = \sum\I X_{2(i + j)} Z_3^i$,\quad $d_j = \sum\I X_{2(i + j) + 1} Z_3^i$,\quad $b_j = Z_1 c_j + Z_2 d_j$,\quad $b = b_0$.
		\item $P_k(c_j) = \sum\PI X_{2(i + j)} Z_3^i$, $P_k(d_j) = \sum\PI X_{2(i + j) + 1} Z_3^i$, $P_k(b_j) = Z_1 P_k(c_j) + Z_2 P_k(d_j)$, což jsou počáteční úseky řad $b_j$, $c_j$, $d_j$ vzhledem k proměnné $Z_3$.
		\item $C = \icl{B[b]}$.
	\end{items}
\end{definice}

\newcommand*{\Ball}{B[b_i: i \in \N_0]} % Zkratka, pro nazornost.
\newcommand*{\Bcdall}{B[c_i, d_i: i \in \N_0]}

Ukážeme, že $B[b]$ je hledaný noetherovský obor, jehož celistvý uzávěr $C$ je roven $B[b_i: i \in \N_0]$, což není noetherovský obor.

\begin{pozorovani}
	Použitím obecných tvrzení z minulé kapitoly obdržíme následující vlastnosti definovaných struktur:
	\begin{items}
		\item $A$, $B$ jsou regulární lokální obory Krullovy dimenze $3$, a jsou tedy celistvě uzavřeny (důsledky \ref{thm:reg} a \ref{thm:icl}).
		\item Obor $A$ je celistvý nad $B$ (tvrzení \ref{thm:KX_integral}).
		\item $B[b]$ je lokální noetherovský obor (důsledek \ref{thm:noe} a tvrzení \ref{thm:R_loc}).
	\end{items}
\end{pozorovani}

\begin{pozorovani} \label{thm:bcd_vztahy} Přímo z definice plynou následující vztahy mezi řadami $b_k$, $c_k$, $d_k$.
	\begin{items}
		\item $c_k = X_{2k} + Z_3 c_{k + 1}$,\quad $d_k = X_{2k + 1} + Z_3 d_{k + 1}$,\quad $b_k = Z_1 X_{2k} + Z_2 X_{2k + 1} + Z_3 b_{k + 1}$.
		\item $c_k = P_j(c_{k + j}) + Z_3^j c_{k + j}$,\quad $d_k = P_j(d_{k + j}) + Z_3^j d_{k + j}$,\quad $b_k = P_j(b_{k + j}) + Z_3^j b_{k +j}$.
		\item $B[b_i: i \leq n] = B[b_n]$, neboť z předchozího plyne $b_i \in B[b_{i + 1}]$.
	\end{items}
\end{pozorovani}

\begin{tvrzeni}
	Platí inkluze $\Ball \subseteq C$.

	\begin{proof}
		Z předchozího pozorování vidíme, že $b_{k + 1} = \frac{b_k - Z_1 X_{2k} - Z_2 X_{2k + 1}}{Z_3} \in Q(B[b_k])$. Indukcí tedy dostáváme $b_k \in Q(B[b])$ pro každé $k \in \N_0$.

		Protože je celé $A$ celistvé nad $B$, a tedy i nad $B[b]$, je každé $b_k \in Q(B[b]) \cap A \subseteq C$, což dokazuje požadované.
	\end{proof}
\end{tvrzeni}

\begin{dusledek}
	Neplatí, že \absOK{B[b]}, jinak by totiž $B[b]$ byl dle tvrzení \ref{thm:R_reg} regulární lokální obor, a tedy celistvě uzavřený, to je ale ve sporu s předchozím tvrzením.
\end{dusledek}

\begin{lemma} \label{thm:QRf}
	Nechť $R$ je meziobor mezi $K^p[[Z]]$ a $K[[Z]]$, $F$ libovolná konečná podmnožina $K[[Z]]$. Pak platí $Q(R[F]) = Q(R)[F]$.
	
	\begin{proof}
		\newcommand*{\I}{_{i = 0}^{p - 1}}
		
		Tvrzení stačí dokázat pro jednoprvkovou $F = \{f\}$, indukce přenese výsledek na libovolnou konečnou množinu. Inkluze $Q(R)[f] \subseteq Q(R[f])$ je zřejmá. Libovolný prvek $Q(R[f])$ je tvaru $\frac{\sum\I \alpha_i f^i}{\sum\I \beta_i f^i}$, $\alpha_i, \beta_i \in R$. Chceme tento zlomek vhodně rozšířit (usměrnit), aby bylo jasné, že je prvkem $Q(R)[f]$. Stačí tedy najít prvky $\gamma_i \in Q(R)$ tak, aby $(\sum\I \beta_i f^i)(\sum\I \gamma_i f^i) \in R$. Přitom víme, že $f^p \in K^p[[Z]] \subseteq R$. Roznásobme sumy a porovnejme koeficienty. Chceme, aby koeficienty u $f^i$ pro $i = 1, \hdots, p - 1$ byly nulové, tedy získáváme soustavu $p - 1$ lineárních rovnic nad $Q(R)$ o $p$ neznámých. Taková soustava má vždy netriviální řešení, a zlomek lze tedy vhodně rozšířit.
	\end{proof}
\end{lemma}

\begin{lemma} \label{thm:B_vlastnost}
Pro každé $f \in B$ existuje $n$, že $f \in K^p[[Z]][X_0, \hdots, X_n]$.
	\begin{proof}
		Každý prvek konečné množiny $K_f$ (viz poznámka \ref{thm:TXK}) leží v nějakém $Q(K^p[X_0, \hdots, X_n]) = K^p[X_0, \hdots, X_n]$, kde poslední rovnost plyne z předchozího lemmatu. Stačí tedy vzít největší $n$.
	\end{proof}
\end{lemma}

\begin{lemma} \label{thm:lin_indep} \hfill
	\begin{items}
		\item Nechť $Y \subseteq X$, pak $\p M_p(Y)$ je lineárně nezávislá nad $K^p(X \setminus Y) \subseteq K$.
		\item	Množina $\p M_p(c_k, d_k)$ je lineárně nezávislá nad $K^p[[Z]][X_0, \hdots, X_{2k - 1}]$.
		\item Množina $\p M_p(c_k, d_k)$ je lineárně nezávislá nad $Q(B)$. 
	\end{items}
	
	\begin{proof} \hfill
		\begin{items}
			\newcommand*{\I}{_{i = 1}^m}

			\item Protože $K^p(X \setminus Y) = Q(K^p[X \setminus Y])$ a $K = Q(K_0[X])$, tedy $K^p = Q(K_0[X]^p)$, platí $K^p(X \setminus Y) = Q(K_0[X]^p[X \setminus Y])$. Lineární nezávislost tedy stačí dokázat nad $K_0[X]^p[X \setminus Y]$, protože testovanou lineární kombinaci stačí vynásobit společným jmenovatelem.

			K tomu stačí dokázat lineární nezávislost $\p M_p(X_n)$ nad $K_0[X]^p[X_0, \hdots, X_{n - 1}]$. Máme-li totiž lineární kombinaci $\sum\I \alpha_i m_i$, $\alpha_i \in K_0[X]^p[X \setminus Y]$, $m_i \in \p M_p(Y)$, jsou všechny $\alpha_i \in K_0[X]^p[F_1]$, kde $F_1$ je nějaká konečná podmnožina $X \setminus Y$. Podobně $m_i$ jsou monočleny v konečně mnoha neurčitých z $Y$, označme tedy jednu z nich $y$ a ostatní sdružme do množiny $F_2$. Potom máme $\sum\I \alpha_i m_i = \sum\I \alpha_i m'_i y^{k_i}$, $m'_i \in \p M_p(F_2)$, $k_i \in \N_0$. Jedná se tedy o $K_0[X]^p[F_1, F_2]$-lineární kombinaci prvků $\p M_p(Y)$. Nyní stačí přečíslovat neurčité.

			Množina $\p M_p(X_n)$ je lineárně nezávislá nad $K_0[X]^p[X_0, \hdots, X_{n - 1}]$. Je-li $\sum_{i = 0}^{p - 1} \alpha_i X_n^i = 0$, pak polynom $\alpha_i X_n^i \in K_0[X]$ obsahuje ve všech členech $X_n$ v mocnině, jejíž zbytek po dělení $p$ je roven $i$, tedy všechny $\alpha_i X_n^i = 0$, takže $\alpha_i = 0$.

			\item Nechť $\sum_{i = 1}^n \alpha_i m_i = 0$, kde $\alpha_i \in K^p[[Z]][X_0, \hdots, X_{2k - 1}]$, $m_i \in \p M_p(c_k, d_k)$. Podívejme se na tuto rovnost jako na rovnost řad v $K[[Z]]$ a uvažme $m \in \p M(Z)$ nejmenšího stupně takový, že některý z koeficientů $\alpha_{i, m}$ je nenulový. Pak ale dostáváme $\sum_{i = 1}^n \alpha_{i, m} m_{i, 1} = 0$, kde $\alpha_{i, m} \in K^p[X_0, \hdots, X_{2k - 1}]$ a $m_{i, 1}$ je absolutní člen řady $m_i$. Protože $m_i \in \p M_p(c_k, d_k)$, je $m_{i, 1} \in \p M_p(X_{2k}, X_{2k + 1})$. Dle předchozího bodu $\alpha_{i, m} = 0$, což je spor, a tedy všechny koeficienty řad $\alpha_i$ jsou nulové.

			\newcommand*{\J}{_{i, j = 0}^{p - 1}}
			\item Podobně jako v prvním bodě stačí ukázat nezávislost nad $B$. Nechť tedy $\sum\J \alpha_{i, j} c_k^i d_k^j = 0$, $\alpha_{i, j} \in B$. Dle lemmatu \ref{thm:B_vlastnost} existuje $n > k$, že $\alpha_{i, j} \in K^p[[Z]][X_0, \hdots, X_{2n - 1}]$. Označíme-li $\gimel = P_{n - k}(c_n)$, $\daleth = P_{n - k}(d_n)$, máme $c_k = \gimel + Z_3^{n - k} c_n$; $d_k = \daleth + Z_3^{n - k} d_n$; $\gimel, \daleth \in K^p[[Z]][X_0, \hdots, X_{2n - 1}]$. Dostáváme
			\mld{
				0 = \sum\J \alpha_{i, j} c_k^i d_k^j = \sum\J \alpha_{i, j} (\gimel + Z_3^{n - k} c_n)^i (\daleth + Z_3^{n - k} d_n)^j = \sum\J \beta_{i, j} c_n^i d_n^j,
			}
			pro nějaká $\beta_{i, j} \in K^p[[Z]][X_0, \hdots, X_{2n - 1}]$.

			Z předchozího bodu $\beta_{i, j} = 0$. Platí $\alpha_{p - 1, p - 1} = (Z_3^{n - k})^{2(p - 1)} \beta_{p - 1, p - 1} = 0$. Potom indukcí $\alpha_{i, j} = (Z_3^{n - k})^{i + j} \beta_{i, j} = 0$.
			\qedhere
		\end{items}
	\end{proof}
\end{lemma}

\begin{lemma} \label{thm:Bcdall_icl}
	Obor $\Bcdall$ je celistvě uzavřený.
	
	\begin{proof}
		\newcommand*{\I}{_{i = 0}^{n - 1}}
		
		Označme $W = \Bcdall$. Nechť $g \in Q(W)$ je celistvý nad $W$. Ukážeme, že $g \in W$. Z definice celistvosti $g^n = \sum\I \alpha_i g^i$ pro nějaké $n \in \N$ a nějaká $\alpha_i \in W$. Dále pišme $g = \frac{w_1}{w_2}$, kde $w_1, w_2 \in W$.
		
		Ukážeme, že každé $w \in W$ je prvkem nějakého $H_k = K^p[[Z]][X_0, \hdots, X_{2k - 1}, c_k, d_k]$.	Užitím pozorování \ref{thm:bcd_vztahy} dostaneme $k_0$, že $w \in B[c_{k_0}, d_{k_0}]$. Tedy $w = \sum_{i, j = 0}^{p - 1} \beta_{i, j} c_{k_0}^i d_{k_0}^j$, $\beta_{i, j} \in B$. Dle lemmatu \ref{thm:B_vlastnost} můžeme vzít $k$: $2k - 1 \geq k_0$, že všechna $\beta_{i, j} \in K^p[X_0, \hdots X_{2k - 1}]$. Potom $w \in K^p[X_0, \hdots, X_{2k - 1}, c_{k_0}, d_{k_0}] \subseteq H_k$.
		
		Volme tedy společné $k$ pro všechna $\alpha_i$, $w_1$, $w_2$. Potom $g \in Q(H_k)$ a je celistvý nad $H_k$. Obory $H_k$ jsou regulární lokální. Vskutku, dle tvrzení \ref{thm:RF_noe} jsou $H_k$ noetherovské. Dle tvrzení \ref{thm:R_loc} jsou invertibilní právě řady s nenulovým absolutním členem. Dle tvrzení \ref{thm:RF_absOK}, lemmatu \ref{thm:lin_indep} a tvrzení \ref{thm:RpF_absOK} jsou postupně v oborech $K^p[[Z]][X_0, \hdots, X_{2k - 1}]$ a $H_k$ řady bez absolutního členu generovány neurčitými. Tvrzení \ref{thm:R_reg} dává regularitu. Protože obory $H_k$ jsou regulární, a tedy celistvě uzavřené, je $g \in H_k \subseteq W$.
	\end{proof}
\end{lemma}

\begin{lemma}
	Pro každé $n \in \N_0$ je $\frac{B[b]}{Z_3^n} \cap A \subseteq \Ball$.
	
	\begin{proof}
		\newcommand*{\I}{_{i = 0}^{p - 1}}
		
		Protože $b^p \in B$ (důsledek \ref{thm:p_power}), je $B[b] = \sum\I b^i B$. Nechť tedy $a = \sum\I \beta_i b^i / Z_3^n \in A$, $\beta_i \in B$, je obecný prvek $\frac{B[b]}{Z_3^n} \cap A$. Označme $\beth_n = P_n(b) \in B$ počáteční úsek $b$ délky $n$. Pak $b = \beth_n + Z_3^n b_n$, tedy
		\mld{
			a Z_3^n = \sum\I \beta_i b^i = \sum\I \beta_i (\beth_n + Z_3^n b_n)^i = \sum\I \beta_i \beth_n + Z_3^n \sum\I \beta_i \gamma_i
		}
		pro $\gamma_i \in B[b_n]$. Dále
		\mld{
			Z_3^n (\underbrace{a - \sum\I \beta_i \gamma_i}_{\in A}) = \sum\I \beta_i \beth_n \in B.
		}
		Protože libovolná řada z $A$ je prvkem $B$ pouze v závislosti na množině koeficientů vzhledem k neurčitým $Z_1$, $Z_2$, $Z_3$ (viz poznámka \ref{thm:TXK}), je $a - \sum\I \beta_i \gamma_i \in B$, tudíž $a \in B + B[b_n] \subseteq \Ball$.
	\end{proof}
\end{lemma}

\newpage % pri zmene textu odstranit
\begin{tvrzeni}
	Platí inkluze $C \subseteq A \cap \bigcup_{n = 0}^\infty \frac{B[b]}{Z_3^n}$, a proto také $C \subseteq \Ball$.

	\begin{proof}
		\newcommand*{\I}{_{i = 0}^{p - 1}}
		
		Volme $f \in C$ libovolně. $f \in Q(B[b]) = Q(B)[b]$ dle lemmatu \ref{thm:QRf}, tedy $f = \sum\I \alpha_i b^i$, $\alpha_i \in Q(B)$. Ukážeme, že $\alpha_i \in \frac{B}{Z_3^n}$ a že $f \in A$. Pak dle předchozího lemmatu $f \in \Ball$.
		
		Platí $B[b] \subseteq \Bcdall$, a tedy $C = \icl{B[b]} \subseteq \icl{\Bcdall} = \Bcdall \subseteq A$, kde poslední rovnost zaručuje lemma \ref{thm:Bcdall_icl}. Existuje $n \in \N_0$, že $f \in B[c_i, d_i: i = 0, \hdots, n]$. Protože $c_0 = P_i(c_0) + Z_3^i c_i$, kde $P_i(c_0) \in B$, a totéž platí pro $d_i$, je $f Z_3^n \in B[c_0, d_0]$, tedy můžeme psát $f Z_3^n = \sum_{m \in \p M_p(c_0, d_0)} \beta_m m$, $\beta_m \in B$. Zároveň $f = \sum\I \alpha_i b^i = \sum\I \alpha_i (Z_1 c_0 + Z_2 d_0)^i$, $\alpha_i \in Q(B)$. Z lineární nezávislosti $\p M_p(c_0, d_0)$ nad $Q(B)$ (lemma \ref{thm:lin_indep}) můžeme porovnat členy. Porovnáním koeficientů u $c_0^i$ a $d_0^i$ dostáváme
		\ml{
			\alpha_0 Z_3^n = \beta_1, \quad \alpha_i Z_1^i Z_3^n = \beta_{c_0^i}, \quad \alpha_i Z_2^i Z_3^n = \beta_{d_0^i}, \quad i > 0.
		}
		Protože $B$ je Gaussův obor, můžeme psát $\alpha_i = \frac{w_1}{w_2}$, kde $w_1$, $w_2$ jsou nesoudělné prvky $B$. Potom máme
		\ml{
			w_1 Z_1^i Z_3^n = w_2 \beta_{c_0^i}, \quad w_1 Z_2^i Z_3^n = w_2 \beta_{d_0^i}.
		}
		Kdyby $Z_1$ dělilo $w_2$, pak by (z druhé rovnice) dělilo i $w_1$, což je spor s nesoudělností. Pak, ale $Z_1$ i $w_1$ dělí $\beta_{c_0^i}$ (z první rovnice) a $w_2 = Z_3^l$, pro nějaké $l \leq n$. Celkem pro každé $\alpha_i$ existuje $n$, že $\alpha_i \in \frac{B}{Z_3^n}$ a tvrzení platí. Závěr platí z předchozího lemmatu.
	\end{proof}
\end{tvrzeni}

\begin{dusledek}
	Platí $C = \Ball = A \cap \bigcup_{n = 0}^\infty \frac{B[b]}{Z_3^n}$.
	
	\begin{proof}
		Z předchozích lemmat a tvrzení máme $C \subseteq A \cap \bigcup_{n = 0}^\infty \frac{B[b]}{Z_3^n} \subseteq \Ball \subseteq C$.
	\end{proof}
\end{dusledek}

\begin{tvrzeni}
	Obor $\Ball$ není noetherovský.
	
	\begin{proof}
		\newcommand*{\I}{_{i = 0}^{n - 1}}

		Dokážeme, že posloupnost ideálů $I_n = \sum\I b_i \Ball$ je ostře rostoucí. K tomu stačí ukázat, že $b_n \notin I_{n - 1}$, $\forall n \in \N$.

		Budeme postupovat sporem. Ať $n \in \N$ je nejmenší, že $b_n \in I_{n - 1}$. Existuje tedy posloupnost $\{f_i\}\I \subseteq \Ball$ tak, že $b_n = \sum\I f_i b_i$. Označme $\p M = \p M_p(b_i: i \in \N_0)$. Každý $f_i \in \Ball$, a proto je lze vyjádřit jako $f_i = \sum_{m \in \p M} f_{i, m} m$, pro $f_{i, m} \in B$, kde pouze konečně mnoho $f_{i, m}$ je nenulových.

		Potom
		\mld{
			b_n = \sum\I \sum_{m \in \p M} f_{i, m} m b_i = \sum\I \sum_{m \in \p M \setminus \{1\}} f_{i, m} m b_i + \sum\I f_{i, 1} b_i.
		}
		A protože $f_{i, 1} = g_i + Z_1 g_{i, 1} + Z_2 g_{i, 2}$ pro nějaká $g_i \in K^p[[Z_3]][K]$ a $g_{i, 1}, g_{i, 2} \in B$. Dostáváme
		\mld{
			b_n = \underbrace{\sum\I \sum_{m \in \p M \setminus \{1\}} f_{i, m} m b_i + \sum\I \left(Z_1 g_{i, 1} + Z_2 g_{i, 2}\right)b_i}_{\in Z_1^2 A + Z_2^2 A + Z_1 Z_2 A} + \sum\I g_i b_i.
		}
		První dvě sumy obsahují ve svých členech vždy některé z $Z_1^2, Z_2^2$ nebo
		$Z_1 Z_2$, ale $b_n$ neobsahuje žádné z nich. Proto $b_n = \sum\I g_i b_i$. Dále máme
		\mld{
			Z_1 c_n + Z_2 d_n = b_n = \sum\I g_i b_i = Z_1 \sum\I g_i c_i + Z_2 \sum\I g_i d_i
		}
		Tedy
		\mld{
			c_n = \sum\I g_i c_i \quad \& \quad d_n = \sum\I g_i d_i,
		}
		protože $g_i \in K^p[[Z_3]][K]$.

		První rovnost (resp.\ obě, postup je analogický) dovedeme ke sporu. Podívejme se na tuto rovnost jako na rovnost řad v proměnné $Z_3$. Označíme-li $g_i = \sum_{j = 0}^\infty g_{i, j} Z_3^j$, $g_{i, j} \in K$, dostaneme porovnáním koeficientů u $Z_3^k$ následující rovnost:
		\mld{
			X_{2(n + k)} = c_{n, k} = \sum\I \sum_{j = 0}^k g_{i, j} c_{i, k - j} = \sum\I \sum_{j = 0}^k g_{i, j} X_{2(i + k - j)}.
		}
		
		Podle lemmatu \ref{thm:B_vlastnost} existuje $m \in \N$, že všechna $g_i \in K^p[[Z_3]][X_0,\dots, X_m]$. Pak ale $g_{i, j} \in K^p[X_0,\dots, X_m]$. Existuje $k \in \N$, že $2(n + k) \geq m$, potom dle rovnosti výše $X_{2(n + k)} \in K^p[X_0,\dots, X_{2(n + k)}]$. To je ale spor s prvním bodem lemmatu \ref{thm:lin_indep}, a tedy $b_n \not\in I_{n-1}$.
	\end{proof}
\end{tvrzeni}

\begin{veta}
	Obor $B[b]$ je příkladem noetherovského oboru, jehož celistvý uzávěr v podílovém nadtělese není noetherovský.
	
	\begin{proof}
		Plyne z předchozích tvrzení.
	\end{proof}
\end{veta}


\end{document}
